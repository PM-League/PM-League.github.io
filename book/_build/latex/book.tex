%% Generated by Sphinx.
\def\sphinxdocclass{jupyterBook}
\documentclass[letterpaper,10pt,english]{jupyterBook}
\ifdefined\pdfpxdimen
   \let\sphinxpxdimen\pdfpxdimen\else\newdimen\sphinxpxdimen
\fi \sphinxpxdimen=.75bp\relax
\ifdefined\pdfimageresolution
    \pdfimageresolution= \numexpr \dimexpr1in\relax/\sphinxpxdimen\relax
\fi
%% let collapsible pdf bookmarks panel have high depth per default
\PassOptionsToPackage{bookmarksdepth=5}{hyperref}
%% turn off hyperref patch of \index as sphinx.xdy xindy module takes care of
%% suitable \hyperpage mark-up, working around hyperref-xindy incompatibility
\PassOptionsToPackage{hyperindex=false}{hyperref}
%% memoir class requires extra handling
\makeatletter\@ifclassloaded{memoir}
{\ifdefined\memhyperindexfalse\memhyperindexfalse\fi}{}\makeatother

\PassOptionsToPackage{booktabs}{sphinx}
\PassOptionsToPackage{colorrows}{sphinx}

\PassOptionsToPackage{warn}{textcomp}

\catcode`^^^^00a0\active\protected\def^^^^00a0{\leavevmode\nobreak\ }
\usepackage{cmap}
\usepackage{fontspec}
\defaultfontfeatures[\rmfamily,\sffamily,\ttfamily]{}
\usepackage{amsmath,amssymb,amstext}
\usepackage{polyglossia}
\setmainlanguage{english}



\setmainfont{FreeSerif}[
  Extension      = .otf,
  UprightFont    = *,
  ItalicFont     = *Italic,
  BoldFont       = *Bold,
  BoldItalicFont = *BoldItalic
]
\setsansfont{FreeSans}[
  Extension      = .otf,
  UprightFont    = *,
  ItalicFont     = *Oblique,
  BoldFont       = *Bold,
  BoldItalicFont = *BoldOblique,
]
\setmonofont{FreeMono}[
  Extension      = .otf,
  UprightFont    = *,
  ItalicFont     = *Oblique,
  BoldFont       = *Bold,
  BoldItalicFont = *BoldOblique,
]



\usepackage[Bjarne]{fncychap}
\usepackage[,numfigreset=1,mathnumfig]{sphinx}

\fvset{fontsize=\small}
\usepackage{geometry}


% Include hyperref last.
\usepackage{hyperref}
% Fix anchor placement for figures with captions.
\usepackage{hypcap}% it must be loaded after hyperref.
% Set up styles of URL: it should be placed after hyperref.
\urlstyle{same}

\addto\captionsenglish{\renewcommand{\contentsname}{Project Management}}

\usepackage{sphinxmessages}



        % Start of preamble defined in sphinx-jupyterbook-latex %
         \usepackage[Latin,Greek]{ucharclasses}
        \usepackage{unicode-math}
        % fixing title of the toc
        \addto\captionsenglish{\renewcommand{\contentsname}{Contents}}
        \hypersetup{
            pdfencoding=auto,
            psdextra
        }
        % End of preamble defined in sphinx-jupyterbook-latex %
        

\title{PM League Wiki}
\date{Mar 08, 2024}
\release{}
\author{FMO}
\newcommand{\sphinxlogo}{\vbox{}}
\renewcommand{\releasename}{}
\makeindex
\begin{document}

\pagestyle{empty}
\sphinxmaketitle
\pagestyle{plain}
\sphinxtableofcontents
\pagestyle{normal}
\phantomsection\label{\detokenize{intro::doc}}


\sphinxAtStartPar
References:
\begin{itemize}
\item {} 
\sphinxAtStartPar
Project Management

\begin{itemize}
\item {} 
\sphinxAtStartPar
{\hyperref[\detokenize{PM/pm-concepts::doc}]{\sphinxcrossref{PM Concepts}}}

\item {} 
\sphinxAtStartPar
{\hyperref[\detokenize{PM/pm-processes::doc}]{\sphinxcrossref{PM Processes}}}

\item {} 
\sphinxAtStartPar
{\hyperref[\detokenize{PM/jcsrua::doc}]{\sphinxcrossref{Probabilistic Scheduling}}}

\item {} 
\sphinxAtStartPar
{\hyperref[\detokenize{PM/lob::doc}]{\sphinxcrossref{Line of Balance}}}

\item {} 
\sphinxAtStartPar
{\hyperref[\detokenize{PM/ccm::doc}]{\sphinxcrossref{Critical Chain Method}}}

\item {} 
\sphinxAtStartPar
{\hyperref[\detokenize{PM/evm::doc}]{\sphinxcrossref{Earned Value Management}}}

\item {} 
\sphinxAtStartPar
{\hyperref[\detokenize{PM/rm::doc}]{\sphinxcrossref{Risk Management}}}

\item {} 
\sphinxAtStartPar
{\hyperref[\detokenize{PM/eac::doc}]{\sphinxcrossref{Estimates at Completion}}}

\item {} 
\sphinxAtStartPar
{\hyperref[\detokenize{PM/ppm::doc}]{\sphinxcrossref{Project Portfolio Management}}}

\end{itemize}
\end{itemize}
\begin{itemize}
\item {} 
\sphinxAtStartPar
Agile Project Management

\begin{itemize}
\item {} 
\sphinxAtStartPar
{\hyperref[\detokenize{APM/agile::doc}]{\sphinxcrossref{Agile Domains, Tools, and Techniques}}}

\item {} 
\sphinxAtStartPar
{\hyperref[\detokenize{APM/sbok::doc}]{\sphinxcrossref{Scrum Body of Knowledge}}}

\end{itemize}
\end{itemize}
\begin{itemize}
\item {} 
\sphinxAtStartPar
Systems Engineering

\begin{itemize}
\item {} 
\sphinxAtStartPar
{\hyperref[\detokenize{SE/sebok::doc}]{\sphinxcrossref{SEBoK}}}

\end{itemize}
\end{itemize}
\begin{itemize}
\item {} 
\sphinxAtStartPar
Systems Project Management

\begin{itemize}
\item {} 
\sphinxAtStartPar
{\hyperref[\detokenize{SPM/spm-concepts::doc}]{\sphinxcrossref{SPM Concepts}}}

\item {} 
\sphinxAtStartPar
{\hyperref[\detokenize{SPM/DSM::doc}]{\sphinxcrossref{Design Structure Matrix (DSM)}}}

\end{itemize}
\end{itemize}
\begin{itemize}
\item {} 
\sphinxAtStartPar
Misc

\begin{itemize}
\item {} 
\sphinxAtStartPar
{\hyperref[\detokenize{Misc/pert-mod::doc}]{\sphinxcrossref{Modified PERT Bistribution}}}

\item {} 
\sphinxAtStartPar
{\hyperref[\detokenize{Misc/putnam::doc}]{\sphinxcrossref{Putnam Model}}}

\end{itemize}
\end{itemize}

\sphinxstepscope


\part{Project Management}

\sphinxstepscope


\chapter{PM Concepts}
\label{\detokenize{PM/pm-concepts:pm-concepts}}\label{\detokenize{PM/pm-concepts::doc}}

\section{Overview}
\label{\detokenize{PM/pm-concepts:overview}}
\sphinxAtStartPar
This page provides
\begin{itemize}
\item {} 
\sphinxAtStartPar
the key concepts applicable to most projects, and

\item {} 
\sphinxAtStartPar
the environments in which projects are performed.

\end{itemize}

\sphinxAtStartPar
Figure 1 shows how PM concepts relate to each other;
\begin{itemize}
\item {} 
\sphinxAtStartPar
the \sphinxstyleemphasis{organizational strategy} identifies \sphinxstyleemphasis{opportunities};

\item {} 
\sphinxAtStartPar
\sphinxstyleemphasis{opportunities} are evaluated and should be documented;

\item {} 
\sphinxAtStartPar
selected \sphinxstyleemphasis{opportunities} are further developed in a \sphinxstyleemphasis{business case} or other similar document and can result in one or more projects that provide \sphinxstyleemphasis{deliverables};

\item {} 
\sphinxAtStartPar
those \sphinxstyleemphasis{deliverables} can be used to realize \sphinxstyleemphasis{benefits},

\item {} 
\sphinxAtStartPar
the \sphinxstyleemphasis{benefits} can contribute to realizing and further developing the \sphinxstyleemphasis{organizational strategy}.

\end{itemize}


\begin{savenotes}\sphinxattablestart
\sphinxthistablewithglobalstyle
\centering
\begin{tabulary}{\linewidth}[t]{T}
\sphinxtoprule
\sphinxstyletheadfamily 
\sphinxAtStartPar
\sphinxincludegraphics{{Fig1-1.drawio}.png}
\\
\sphinxmidrule
\sphinxtableatstartofbodyhook
\sphinxAtStartPar
Figure 1 — Example of value creation framework
\\
\sphinxbottomrule
\end{tabulary}
\sphinxtableafterendhook\par
\sphinxattableend\end{savenotes}


\section{Project}
\label{\detokenize{PM/pm-concepts:project}}
\sphinxAtStartPar
A \sphinxstylestrong{project} consists of a unique set of processes comprising coordinated and controlled activities with start and end dates to achieve project objectives. \\
Achievement of the project objectives requires providing deliverables conforming to specific requirements.

\sphinxAtStartPar
A project may be subject to multiple constraints.

\sphinxAtStartPar
Although many projects may be similar, each project is unique. \\
Project differences may occur in the following:
\begin{itemize}
\item {} 
\sphinxAtStartPar
deliverables provided;

\item {} 
\sphinxAtStartPar
stakeholders influencing;

\item {} 
\sphinxAtStartPar
resources used;

\item {} 
\sphinxAtStartPar
constraints;

\item {} 
\sphinxAtStartPar
the way processes are tailored to provide the deliverables.

\end{itemize}

\sphinxAtStartPar
Every project has a definite start and end and is usually divided into phases.


\section{Project Management}
\label{\detokenize{PM/pm-concepts:project-management}}
\sphinxAtStartPar
\sphinxstylestrong{Project management (PM)}
\begin{itemize}
\item {} 
\sphinxAtStartPar
is the application of methods, tools, techniques, and competencies to a project;

\item {} 
\sphinxAtStartPar
includes the integration of the various phases of the project life cycle;

\item {} 
\sphinxAtStartPar
is performed through processes.

\end{itemize}

\sphinxAtStartPar
The processes selected for performing a project should be aligned in a systemic view.

\sphinxAtStartPar
Each phase of the project life cycle should have specific deliverables. \\
These deliverables should be regularly reviewed during the project to meet the requirements of the sponsor, customers, and other stakeholders.


\section{Organizational Strategy and Projects}
\label{\detokenize{PM/pm-concepts:organizational-strategy-and-projects}}

\subsection{Organizational Strategy}
\label{\detokenize{PM/pm-concepts:organizational-strategy}}
\sphinxAtStartPar
Organizations generally establish strategies based on their mission, vision, policies, and factors outside the organizational boundary. \\
Projects are often the means to accomplish \sphinxstyleemphasis{strategic goals}. \\
An example of a value creation framework is shown in Figure 2.


\begin{savenotes}\sphinxattablestart
\sphinxthistablewithglobalstyle
\centering
\begin{tabulary}{\linewidth}[t]{T}
\sphinxtoprule
\sphinxstyletheadfamily 
\sphinxAtStartPar
\sphinxincludegraphics{{Fig1-2.drawio}.png}
\\
\sphinxmidrule
\sphinxtableatstartofbodyhook
\sphinxAtStartPar
Figure 2 — Example of value creation framework
\\
\sphinxbottomrule
\end{tabulary}
\sphinxtableafterendhook\par
\sphinxattableend\end{savenotes}

\sphinxAtStartPar
\sphinxstyleemphasis{Strategic goals} may guide the identification and development of \sphinxstyleemphasis{opportunities}. \\
\sphinxstyleemphasis{Opportunities} selection includes consideration of various factors, such as how benefits can be realized and risks can be managed.

\sphinxAtStartPar
The project aims to provide measurable benefits that contribute to realizing the selected opportunities.

\sphinxAtStartPar
The project objective contributes to the project goal by creating the required deliverables. \\
Project goals
\begin{itemize}
\item {} 
\sphinxAtStartPar
are achieved when the benefits are realized;

\item {} 
\sphinxAtStartPar
might not be achieved until a time after the objectives are achieved.

\end{itemize}


\subsection{Opportunity Evaluation and Project Initiation}
\label{\detokenize{PM/pm-concepts:opportunity-evaluation-and-project-initiation}}
\sphinxAtStartPar
To support informed decision\sphinxhyphen{}making by responsible management, opportunities may be evaluated to identify feasible projects that could transform some or all of these opportunities into realized benefits. \\
These opportunities
\begin{itemize}
\item {} 
\sphinxAtStartPar
may address, for example, a new market demand, a current organizational need, or a new legal requirement; and

\item {} 
\sphinxAtStartPar
are often evaluated through a set of activities that provide formal authorization to start a new project.

\end{itemize}

\sphinxAtStartPar
The organization should identify a project sponsor responsible for project \sphinxstyleemphasis{goals and benefits}. \\
The \sphinxstyleemphasis{goals and benefits} may result in a \sphinxstyleemphasis{justification} for the investment in the project (e.g., in the form of a business case), which may contribute to prioritizing all opportunities. \\
The purpose of the \sphinxstyleemphasis{justification} is usually to obtain organizational commitment and approval for investment in the selected projects.

\sphinxAtStartPar
The evaluation process may include multiple criteria, including
\begin{itemize}
\item {} 
\sphinxAtStartPar
financial investment appraisal techniques, and

\item {} 
\sphinxAtStartPar
qualitative criteria (e.g., strategic alignment, social impact, and environmental impact).
Criteria may differ from one project to another.

\end{itemize}


\subsection{Benefits Realization}
\label{\detokenize{PM/pm-concepts:benefits-realization}}
\sphinxAtStartPar
Benefits realization is generally the responsibility of organizational management, which may use the deliverables of the project to realize benefits in alignment with the organizational strategy. \\
The project manager should consider the benefits and their realization as they influence decision\sphinxhyphen{}making throughout the project life cycle.


\section{Project Environment}
\label{\detokenize{PM/pm-concepts:project-environment}}

\subsection{General}
\label{\detokenize{PM/pm-concepts:general}}
\sphinxAtStartPar
The project environment may impact project performance and success.

\sphinxAtStartPar
The project team should consider the following:
\begin{itemize}
\item {} 
\sphinxAtStartPar
factors outside the organizational boundary, such as socio\sphinxhyphen{}economic, geographical, political, regulatory, technological, and ecological;

\item {} 
\sphinxAtStartPar
factors inside the organizational boundary include strategy, technology, project management maturity, resource availability, organizational culture, and structure.

\end{itemize}


\subsection{Factors Outside the Organizational Boundary}
\label{\detokenize{PM/pm-concepts:factors-outside-the-organizational-boundary}}
\sphinxAtStartPar
Factors outside the organizational boundary may impact the project by imposing constraints or introducing risks affecting the project. \\
Although these factors are often beyond the project manager’s control, they should still be considered.


\begin{savenotes}\sphinxattablestart
\sphinxthistablewithglobalstyle
\centering
\begin{tabulary}{\linewidth}[t]{T}
\sphinxtoprule
\sphinxstyletheadfamily 
\sphinxAtStartPar
\sphinxincludegraphics{{Fig1-3.drawio}.png}
\\
\sphinxmidrule
\sphinxtableatstartofbodyhook
\sphinxAtStartPar
Figure 3 — Example of relationships between portfolios, programmes and projects
\\
\sphinxbottomrule
\end{tabulary}
\sphinxtableafterendhook\par
\sphinxattableend\end{savenotes}


\subsection{Factors Inside the Organizational Boundary}
\label{\detokenize{PM/pm-concepts:factors-inside-the-organizational-boundary}}

\subsubsection{General}
\label{\detokenize{PM/pm-concepts:id1}}
\sphinxAtStartPar
A project usually exists inside a larger organization encompassing other activities. \\
In such cases, relationships exist between
\begin{itemize}
\item {} 
\sphinxAtStartPar
the project and its environment,

\item {} 
\sphinxAtStartPar
business planning, and

\item {} 
\sphinxAtStartPar
operations.

\end{itemize}

\sphinxAtStartPar
Pre\sphinxhyphen{}project and post\sphinxhyphen{}project activities may include business case development, feasibility studies, and transition to operations. \\
Projects may be organized within programmes and project portfolios.


\subsubsection{Programme Management}
\label{\detokenize{PM/pm-concepts:programme-management}}
\sphinxAtStartPar
A \sphinxstylestrong{programme} is generally a group of related projects and other activities aligned with strategic goals.

\sphinxAtStartPar
Programme management consists of centralized and coordinated activities to achieve the goals.


\subsubsection{Project Portfolio Management}
\label{\detokenize{PM/pm-concepts:project-portfolio-management}}
\sphinxAtStartPar
A \sphinxstylestrong{project portfolio} is generally a collection of projects and programmes and other work grouped to facilitate the effective management of that work to meet strategic goals.

\sphinxAtStartPar
Project portfolio management (PPM) is generally the centralized management of one or more project portfolios, which includes identifying, prioritizing,
authorizing, directing, and controlling projects, programmes, and other work to achieve specific strategic goals.

\sphinxAtStartPar
It may be appropriate to conduct the opportunity identification and selection, as well as the approval and management of projects, through a project portfolio management system.


\section{Project Governance}
\label{\detokenize{PM/pm-concepts:project-governance}}
\sphinxAtStartPar
\sphinxstylestrong{Governance} is the framework by which an organization is directed and controlled.

\sphinxAtStartPar
Project governance includes, but is not limited to, those areas of organizational governance specifically related to project activities. \\
Project governance may include subjects such as the following:
\begin{itemize}
\item {} 
\sphinxAtStartPar
defining the management structure;

\item {} 
\sphinxAtStartPar
the policies, processes, and methodologies to be used;

\item {} 
\sphinxAtStartPar
limits of authority for decision\sphinxhyphen{}making;

\item {} 
\sphinxAtStartPar
stakeholder responsibilities and accountabilities;

\item {} 
\sphinxAtStartPar
interactions such as reporting and the escalation of issues or risks.

\end{itemize}

\sphinxAtStartPar
The responsibility for maintaining the appropriate governance of a project is usually assigned either to the project sponsor or to a project steering committee.


\section{Projects and Operations}
\label{\detokenize{PM/pm-concepts:projects-and-operations}}
\sphinxAtStartPar
PM fits within the general framework of management.

\sphinxAtStartPar
PM differs from other management disciplines by the project’s temporary and unique nature.

\sphinxAtStartPar
Organizations perform work to achieve specific goals. \\
Generally, this work may be categorized as either
\begin{itemize}
\item {} 
\sphinxAtStartPar
operations
\begin{itemize}
\item {} 
\sphinxAtStartPar
performed by relatively stable teams through ongoing and repetitive processes,

\item {} 
\sphinxAtStartPar
focused on sustaining the organization;

\end{itemize}

\item {} 
\sphinxAtStartPar
projects
\begin{itemize}
\item {} 
\sphinxAtStartPar
are performed by temporary teams,

\item {} 
\sphinxAtStartPar
are non\sphinxhyphen{}repetitive, and

\item {} 
\sphinxAtStartPar
provide unique deliverables.

\end{itemize}

\end{itemize}


\section{Stakeholders and Project Organization}
\label{\detokenize{PM/pm-concepts:stakeholders-and-project-organization}}
\sphinxAtStartPar
The project stakeholders, including the project organization, should be described in sufficient detail for the project to be successful.

\sphinxAtStartPar
The roles and responsibilities of stakeholders should be defined and communicated based on the organization and project goals.

\sphinxAtStartPar
Stakeholder interfaces should be managed within the project through the PM processes.

\sphinxAtStartPar
The project organization is the temporary structure that includes project roles, responsibilities and levels of authority, and boundaries that must be defined and communicated to all project stakeholders.
The project organization
\begin{itemize}
\item {} 
\sphinxAtStartPar
may depend on project stakeholders’ legal, commercial, interdepartmental, or other arrangements.

\item {} 
\sphinxAtStartPar
may include the following roles and responsibilities:
\begin{enumerate}
\sphinxsetlistlabels{\arabic}{enumi}{enumii}{}{.}%
\item {} 
\sphinxAtStartPar
the \sphinxstyleemphasis{project manager}, who leads and manages project activities and is accountable for project completion;

\item {} 
\sphinxAtStartPar
the \sphinxstyleemphasis{PM team}, which supports the project manager in leading and managing the project activities;

\item {} 
\sphinxAtStartPar
the \sphinxstyleemphasis{project team}, which performs project activities.

\end{enumerate}

\end{itemize}

\sphinxAtStartPar
Project governance may involve the following:
\begin{itemize}
\item {} 
\sphinxAtStartPar
the \sphinxstyleemphasis{project sponsor}, who authorizes the project, makes executive decisions, and solves problems and conflicts beyond the project manager’s authority;

\item {} 
\sphinxAtStartPar
the \sphinxstyleemphasis{project steering committee} or \sphinxstyleemphasis{board} contributes to the project by providing senior\sphinxhyphen{}level guidance.

\end{itemize}


\begin{savenotes}\sphinxattablestart
\sphinxthistablewithglobalstyle
\centering
\begin{tabulary}{\linewidth}[t]{T}
\sphinxtoprule
\sphinxstyletheadfamily 
\sphinxAtStartPar
\sphinxincludegraphics{{Fig1-4.drawio}.png}
\\
\sphinxmidrule
\sphinxtableatstartofbodyhook
\sphinxAtStartPar
Figure 4 — Example of potential project stakeholders
\\
\sphinxbottomrule
\end{tabulary}
\sphinxtableafterendhook\par
\sphinxattableend\end{savenotes}

\sphinxAtStartPar
Figure 4 includes the following additional stakeholders:
\begin{itemize}
\item {} 
\sphinxAtStartPar
customers or customer representatives who contribute to the project by specifying project requirements and accepting the project deliverables;

\item {} 
\sphinxAtStartPar
suppliers, who contribute to the project by supplying resources to the project;

\item {} 
\sphinxAtStartPar
the project management office (PMO) may perform various activities, including governance, standardization, PM training, project planning, and project monitoring.

\end{itemize}


\section{Competencies of Project Personnel}
\label{\detokenize{PM/pm-concepts:competencies-of-project-personnel}}
\sphinxAtStartPar
Project personnel should develop competencies in PM principles and processes to achieve project objectives and goals. \\
Each project team requires competent individuals capable of applying their knowledge and experience to provide the project deliverables.

\sphinxAtStartPar
Any identified gap between the available and required competence levels represented on the project team could introduce risk and should be addressed.

\sphinxAtStartPar
PM competencies can be categorized into, but are not limited to, the following:
\begin{itemize}
\item {} 
\sphinxAtStartPar
technical competencies for delivering projects in a structured way, including the project management terminology, concepts, and processes defined in this International Standard;

\item {} 
\sphinxAtStartPar
behavioral competencies associated with personal relationships inside the defined boundaries of the project;

\item {} 
\sphinxAtStartPar
contextual competencies related to the management of the project inside the organizational and external environment.
Competency levels may be raised through professional development processes such as training, coaching, and mentoring inside or outside an organization.

\end{itemize}


\section{Project Life Cycle}
\label{\detokenize{PM/pm-concepts:project-life-cycle}}
\sphinxAtStartPar
Projects are usually organized into \sphinxstylestrong{phases} determined by governance and control needs. \\
These phases should
\begin{itemize}
\item {} 
\sphinxAtStartPar
follow a logical sequence, with a start and an end, and

\item {} 
\sphinxAtStartPar
use resources to provide deliverables.

\end{itemize}

\sphinxAtStartPar
To manage the project efficiently during the entire project life cycle, a set of activities should be performed in each phase. \\
Project phases are collectively known as the \sphinxstylestrong{project life cycle}. \\
The project life cycle spans the period from the start of the project to its end.

\sphinxAtStartPar
The phases are divided by \sphinxstylestrong{decision points}, varying depending on the organizational environment.
The decision points facilitate project governance. \\
By the end of the last phase, the project should have provided all deliverables. \\
To manage a project throughout its life cycle, PM processes should be used for the project as a whole or individual phase for each team or sub\sphinxhyphen{}project.


\section{Project Constraints}
\label{\detokenize{PM/pm-concepts:project-constraints}}
\sphinxAtStartPar
There are several types of constraints, and as constraints are often interdependent, a project manager needs to balance a particular constraint against the others.
The project deliverables should fulfill the requirements for the project and relate to any given constraints such as
\begin{itemize}
\item {} 
\sphinxAtStartPar
\sphinxstyleemphasis{scope},

\item {} 
\sphinxAtStartPar
\sphinxstyleemphasis{quality},

\item {} 
\sphinxAtStartPar
\sphinxstyleemphasis{schedule},

\item {} 
\sphinxAtStartPar
\sphinxstyleemphasis{resources}, and

\item {} 
\sphinxAtStartPar
\sphinxstyleemphasis{cost}.

\end{itemize}

\sphinxAtStartPar
Constraints are generally interrelated, so a change in one may affect one or more of the other constraints. \\
Hence, the constraints may impact the decisions made within the PM processes. \\
Achievement of consensus among key project stakeholders on the constraints may form a strong foundation for project success.
Some constraints could be the following:
\begin{itemize}
\item {} 
\sphinxAtStartPar
the duration or target date for the project;

\item {} 
\sphinxAtStartPar
the availability of the project budget;

\item {} 
\sphinxAtStartPar
the availability of project resources, such as people, facilities, equipment, materials, infrastructure, tools, and other resources required to carry out the project activities relating to the requirements of the project;

\item {} 
\sphinxAtStartPar
factors related to health and safety of personnel;

\item {} 
\sphinxAtStartPar
the level of acceptable risk exposure;

\item {} 
\sphinxAtStartPar
the potential social or ecological impact of the project;

\item {} 
\sphinxAtStartPar
laws, rules, and other legislative requirements.

\end{itemize}


\section{Relationship between PM Concepts and Processes}
\label{\detokenize{PM/pm-concepts:relationship-between-pm-concepts-and-processes}}
\sphinxAtStartPar
PM is accomplished through processes utilizing the concepts and competencies

\sphinxAtStartPar
A \sphinxstylestrong{process}
\begin{itemize}
\item {} 
\sphinxAtStartPar
is a set of interrelated activities;

\item {} 
\sphinxAtStartPar
is categorized into three major types:
\begin{itemize}
\item {} 
\sphinxAtStartPar
\sphinxstyleemphasis{PM} processes
\begin{itemize}
\item {} 
\sphinxAtStartPar
specific to PM,

\item {} 
\sphinxAtStartPar
determine how the activities selected for the project are managed;

\end{itemize}

\item {} 
\sphinxAtStartPar
\sphinxstyleemphasis{delivery} processes
\begin{itemize}
\item {} 
\sphinxAtStartPar
are not unique to PM,

\item {} 
\sphinxAtStartPar
result in the specification and provision of a particular product, service, or result,

\item {} 
\sphinxAtStartPar
which vary depending on the particular project deliverable;

\end{itemize}

\item {} 
\sphinxAtStartPar
\sphinxstyleemphasis{support} processes
\begin{itemize}
\item {} 
\sphinxAtStartPar
are not unique to PM,

\item {} 
\sphinxAtStartPar
provide relevant and valuable support to product and PM processes in logistics, finance, accounting, and safety disciplines.

\end{itemize}

\end{itemize}

\end{itemize}

\sphinxAtStartPar
Product, support, and PM processes might overlap and interact throughout a project.

\sphinxstepscope


\chapter{PM Processes}
\label{\detokenize{PM/pm-processes:pm-processes}}\label{\detokenize{PM/pm-processes::doc}}

\section{PM Process Application}
\label{\detokenize{PM/pm-processes:pm-process-application}}
\sphinxAtStartPar
The following PM processes
\begin{itemize}
\item {} 
\sphinxAtStartPar
should be used during a project as a whole, for individual phases, or both,

\item {} 
\sphinxAtStartPar
are appropriate for projects in all organizations.

\end{itemize}

\sphinxAtStartPar
PM requires significant coordination and, as such, requires each process to be
\begin{itemize}
\item {} 
\sphinxAtStartPar
appropriately aligned, and

\item {} 
\sphinxAtStartPar
connected with other processes.

\end{itemize}

\sphinxAtStartPar
Some processes may be repeated to
\begin{itemize}
\item {} 
\sphinxAtStartPar
fully define and meet stakeholder requirements, and

\item {} 
\sphinxAtStartPar
reach an agreement on the project objectives.

\end{itemize}

\sphinxAtStartPar
In conjunction with other project stakeholders, project managers are advised to
\begin{itemize}
\item {} 
\sphinxAtStartPar
carefully consider the processes, and

\item {} 
\sphinxAtStartPar
apply them as appropriate to the project and organizational needs.

\end{itemize}

\sphinxAtStartPar
The processes need not be applied uniformly on all projects or phases. \\
Therefore, the project manager should tailor the management processes for each project or phase by determining
\begin{itemize}
\item {} 
\sphinxAtStartPar
what processes are appropriate, and

\item {} 
\sphinxAtStartPar
the degree of rigor for each process.
The relevant organizational policies should accomplish this tailoring.

\end{itemize}

\sphinxAtStartPar
For a project to be successful, the following actions should be accomplished:
\begin{itemize}
\item {} 
\sphinxAtStartPar
select appropriate processes that are required to meet the project objectives;

\item {} 
\sphinxAtStartPar
use a defined approach to develop or adapt the product specifications and plans to meet the project objectives and requirements;

\item {} 
\sphinxAtStartPar
comply with requirements to satisfy the project sponsor, customers, and other stakeholders;

\item {} 
\sphinxAtStartPar
define and manage the project scope within the constraints while considering the project risks and resource needs to provide the project deliverables;

\item {} 
\sphinxAtStartPar
obtain proper support from each performing organization, including a commitment from the customers and project sponsor.

\end{itemize}

\sphinxAtStartPar
The PM processes are defined and described in terms of their purposes, the relationships among the processes, the interactions within the processes, and the primary inputs and outputs associated with each process.


\section{Process Groups and Subject Groups}
\label{\detokenize{PM/pm-processes:process-groups-and-subject-groups}}

\subsection{General}
\label{\detokenize{PM/pm-processes:general}}
\sphinxAtStartPar
The PM processes may be viewed from two different perspectives:
\begin{itemize}
\item {} 
\sphinxAtStartPar
as \sphinxstylestrong{process groups} for the management of the project;

\item {} 
\sphinxAtStartPar
as \sphinxstylestrong{subject groups} for collecting the processes by subject.

\end{itemize}




\subsection{Process Groups}
\label{\detokenize{PM/pm-processes:process-groups}}

\subsubsection{General}
\label{\detokenize{PM/pm-processes:id1}}
\sphinxAtStartPar
Each \sphinxstylestrong{process group} consists of processes that apply to any project phase or project. \\
These processes, further defined in terms of purpose, description, and primary inputs and outputs, are interdependent. \\
The process groups are independent of the application area or industry focus.
Any process may be repeated.


\subsubsection{Initiating}
\label{\detokenize{PM/pm-processes:initiating}}
\sphinxAtStartPar
The \sphinxstyleemphasis{initiating} processes are used to
\begin{itemize}
\item {} 
\sphinxAtStartPar
start a project phase or project,

\item {} 
\sphinxAtStartPar
define the project phase or project objectives, and

\item {} 
\sphinxAtStartPar
authorize the project manager to proceed with the project work.

\end{itemize}


\subsubsection{Planning}
\label{\detokenize{PM/pm-processes:planning}}
\sphinxAtStartPar
The \sphinxstyleemphasis{planning} processes are used to develop planning detail.

\sphinxAtStartPar
This detail should be sufficient to
\begin{itemize}
\item {} 
\sphinxAtStartPar
establish baselines against which project implementation can be managed, and

\item {} 
\sphinxAtStartPar
performance can be measured and controlled.

\end{itemize}


\subsubsection{Implementing}
\label{\detokenize{PM/pm-processes:implementing}}
\sphinxAtStartPar
The \sphinxstyleemphasis{implementing} processes are used to
\begin{itemize}
\item {} 
\sphinxAtStartPar
perform the project management activities, and

\item {} 
\sphinxAtStartPar
support the provision of the project’s deliverables following the project plans.

\end{itemize}


\subsubsection{Controlling}
\label{\detokenize{PM/pm-processes:controlling}}
\sphinxAtStartPar
The \sphinxstyleemphasis{controlling} processes are used to monitor, measure, and control project performance against the project plan. \\
Consequently, preventive and corrective actions may be taken and change requests made to achieve project objectives when necessary.


\subsubsection{Closing}
\label{\detokenize{PM/pm-processes:closing}}
\sphinxAtStartPar
The \sphinxstyleemphasis{closing} processes are used to
\begin{itemize}
\item {} 
\sphinxAtStartPar
formally establish that the project phase or project is finished, and

\item {} 
\sphinxAtStartPar
provide lessons learned to be considered and implemented as necessary.

\end{itemize}


\subsubsection{PM Process Group Interrelationships and Interactions}
\label{\detokenize{PM/pm-processes:pm-process-group-interrelationships-and-interactions}}
\sphinxAtStartPar
The management of a project
\begin{itemize}
\item {} 
\sphinxAtStartPar
starts with the initiating process group, and

\item {} 
\sphinxAtStartPar
finishes with the closing process group.

\end{itemize}

\sphinxAtStartPar
The interdependency between process groups requires the controlling process group to interact with every other process group.


\begin{savenotes}\sphinxattablestart
\sphinxthistablewithglobalstyle
\centering
\begin{tabulary}{\linewidth}[t]{T}
\sphinxtoprule
\sphinxstyletheadfamily 
\sphinxAtStartPar
\sphinxincludegraphics{{Fig2-1.drawio}.png}
\\
\sphinxmidrule
\sphinxtableatstartofbodyhook
\sphinxAtStartPar
Figure 1 — Process groups interactions
\\
\sphinxbottomrule
\end{tabulary}
\sphinxtableafterendhook\par
\sphinxattableend\end{savenotes}

\sphinxAtStartPar
Process groups are
\begin{itemize}
\item {} 
\sphinxAtStartPar
seldom discrete or one\sphinxhyphen{}time in their application.

\item {} 
\sphinxAtStartPar
repeated within each phase to drive the project to completion. \\
All or some processes within the process groups may be required for a project phase. \\
Not all interactions will apply to all project phases or projects. \\
In practice, the processes within the process groups are often

\item {} 
\sphinxAtStartPar
concurrent,

\item {} 
\sphinxAtStartPar
overlapping, and

\item {} 
\sphinxAtStartPar
interacting in ways not shown.

\end{itemize}

\sphinxAtStartPar
Figure 2 elaborates on Figure 1 to show the interactions among the process groups inside the project’s boundaries, including the representative inputs and outputs of processes within the process groups. \\
Except for the controlling process group, linkages between the various process groups are through individual processes within each process group. \\
While linkage is shown in Figure 2 between the controlling process group and other process groups, it may be considered self\sphinxhyphen{}standing because its processes control the overall project and the individual process groups.


\begin{savenotes}\sphinxattablestart
\sphinxthistablewithglobalstyle
\centering
\begin{tabulary}{\linewidth}[t]{T}
\sphinxtoprule
\sphinxstyletheadfamily 
\sphinxAtStartPar
\sphinxincludegraphics{{Fig2-2.drawio}.png}
\\
\sphinxmidrule
\sphinxtableatstartofbodyhook
\sphinxAtStartPar
Figure 2 — Process group interactions showing representative inputs and outputs
\\
\sphinxbottomrule
\end{tabulary}
\sphinxtableafterendhook\par
\sphinxattableend\end{savenotes}


\subsection{Subject Groups}
\label{\detokenize{PM/pm-processes:subject-groups}}

\subsubsection{General}
\label{\detokenize{PM/pm-processes:id2}}
\sphinxAtStartPar
Each \sphinxstylestrong{subject group} consists of processes applicable to any project phase or project. \\
These processes
\begin{itemize}
\item {} 
\sphinxAtStartPar
are defined in terms of
\begin{itemize}
\item {} 
\sphinxAtStartPar
purpose,

\item {} 
\sphinxAtStartPar
description,

\item {} 
\sphinxAtStartPar
primary inputs and outputs, and

\end{itemize}

\item {} 
\sphinxAtStartPar
are interdependent.

\end{itemize}

\sphinxAtStartPar
Subject groups are independent of the application area or industry focus.

\sphinxAtStartPar
The figures in Annex A illustrate the interactions of the individual processes in each process group identified in 4.2.2 mapped against the subject groups. \\
Any process may be repeated.


\subsubsection{Integration}
\label{\detokenize{PM/pm-processes:integration}}
\sphinxAtStartPar
The \sphinxstyleemphasis{integration} subject group includes
\begin{itemize}
\item {} 
\sphinxAtStartPar
the processes required to identify, define, combine, unify, coordinate, control, and close the various activities, and

\item {} 
\sphinxAtStartPar
the processes related to the project.

\end{itemize}


\subsubsection{Stakeholder}
\label{\detokenize{PM/pm-processes:stakeholder}}
\sphinxAtStartPar
The \sphinxstyleemphasis{stakeholder} subject group includes the processes required to identify and manage the project sponsor, customers, and other stakeholders.


\subsubsection{Scope}
\label{\detokenize{PM/pm-processes:scope}}
\sphinxAtStartPar
The \sphinxstyleemphasis{scope} subject group includes
\begin{itemize}
\item {} 
\sphinxAtStartPar
the processes required to identify and define the work and deliverables, and

\item {} 
\sphinxAtStartPar
only the work and deliverables required.

\end{itemize}


\subsubsection{Resource}
\label{\detokenize{PM/pm-processes:resource}}
\sphinxAtStartPar
The \sphinxstyleemphasis{resource} subject group includes the processes required to identify and acquire adequate project resources such as
\begin{itemize}
\item {} 
\sphinxAtStartPar
people,

\item {} 
\sphinxAtStartPar
facilities,

\item {} 
\sphinxAtStartPar
equipment,

\item {} 
\sphinxAtStartPar
materials,

\item {} 
\sphinxAtStartPar
infrastructure, and

\item {} 
\sphinxAtStartPar
tools.

\end{itemize}


\subsubsection{Time}
\label{\detokenize{PM/pm-processes:time}}
\sphinxAtStartPar
The \sphinxstyleemphasis{time} subject group includes the processes required to
\begin{itemize}
\item {} 
\sphinxAtStartPar
schedule the project activities, and

\item {} 
\sphinxAtStartPar
to monitor progress in controlling the schedule.

\end{itemize}


\subsubsection{Cost}
\label{\detokenize{PM/pm-processes:cost}}
\sphinxAtStartPar
The \sphinxstyleemphasis{cost} subject group includes the processes required to
\begin{itemize}
\item {} 
\sphinxAtStartPar
develop the budget, and

\item {} 
\sphinxAtStartPar
monitor progress to control costs.

\end{itemize}


\subsubsection{Risk}
\label{\detokenize{PM/pm-processes:risk}}
\sphinxAtStartPar
The \sphinxstyleemphasis{risk} subject group includes the processes required to identify and manage threats and opportunities.


\subsubsection{Quality}
\label{\detokenize{PM/pm-processes:quality}}
\sphinxAtStartPar
The \sphinxstyleemphasis{quality} subject group includes the processes required to plan and establish quality assurance and control.


\subsubsection{Procurement}
\label{\detokenize{PM/pm-processes:procurement}}
\sphinxAtStartPar
The \sphinxstyleemphasis{procurement} subject group includes the processes required to
\begin{itemize}
\item {} 
\sphinxAtStartPar
plan and acquire products, services or results, and

\item {} 
\sphinxAtStartPar
manage supplier relationships.

\end{itemize}


\subsubsection{Communication}
\label{\detokenize{PM/pm-processes:communication}}
\sphinxAtStartPar
The \sphinxstyleemphasis{communication} subject group includes the processes required to plan, manage and distribute information relevant to the project.

\sphinxstepscope


\chapter{Probabilistic Scheduling}
\label{\detokenize{PM/jcsrua:probabilistic-scheduling}}\label{\detokenize{PM/jcsrua::doc}}
\sphinxAtStartPar
Objective:
\begin{enumerate}
\sphinxsetlistlabels{\arabic}{enumi}{enumii}{}{.}%
\item {} 
\sphinxAtStartPar
Describe techniques to model cost and schedule uncertainty

\item {} 
\sphinxAtStartPar
Provide an introduction to building a fully integrated cost and schedule uncertainty analysis

\end{enumerate}


\section{Scheduling Methods}
\label{\detokenize{PM/jcsrua:scheduling-methods}}\begin{itemize}
\item {} 
\sphinxAtStartPar
Deterministic
\begin{itemize}
\item {} 
\sphinxAtStartPar
Critical Path Method (CPM)

\item {} 
\sphinxAtStartPar
Precedence Diagram Method (PDM)

\item {} 
\sphinxAtStartPar
Critical Resource Diagram (CRD)

\item {} 
\sphinxAtStartPar
Line of Balance (LOB)

\item {} 
\sphinxAtStartPar
Critical Chain Method (CCM)

\item {} 
\sphinxAtStartPar
Design Structure Matrix (DSM)

\end{itemize}

\item {} 
\sphinxAtStartPar
Probabilistic
\begin{itemize}
\item {} 
\sphinxAtStartPar
Program Evaluation Review Technique (PERT)

\item {} 
\sphinxAtStartPar
Monte Carlo

\item {} 
\sphinxAtStartPar
Latin Hypercube Sampling (LHS)

\end{itemize}

\end{itemize}


\section{Deterministics vs Probabilistic}
\label{\detokenize{PM/jcsrua:deterministics-vs-probabilistic}}

\begin{savenotes}\sphinxattablestart
\sphinxthistablewithglobalstyle
\centering
\begin{tabulary}{\linewidth}[t]{T}
\sphinxtoprule
\sphinxstyletheadfamily 
\sphinxAtStartPar
\sphinxincludegraphics{{PE_to_distribution}.svg}
\\
\sphinxmidrule
\sphinxtableatstartofbodyhook
\sphinxAtStartPar
From Deterministic to Probabilistic
\\
\sphinxbottomrule
\end{tabulary}
\sphinxtableafterendhook\par
\sphinxattableend\end{savenotes}


\section{Statistics}
\label{\detokenize{PM/jcsrua:statistics}}

\subsection{Descriptive Statistics}
\label{\detokenize{PM/jcsrua:descriptive-statistics}}

\begin{savenotes}\sphinxattablestart
\sphinxthistablewithglobalstyle
\centering
\begin{tabulary}{\linewidth}[t]{TT}
\sphinxtoprule
\sphinxstyletheadfamily 
\sphinxAtStartPar
Statistic
&\sphinxstyletheadfamily 
\sphinxAtStartPar
Formula
\\
\sphinxmidrule
\sphinxtableatstartofbodyhook
\sphinxAtStartPar
Random variable
&
\sphinxAtStartPar
\(X\)
\\
\sphinxhline
\sphinxAtStartPar
Mean
&
\sphinxAtStartPar
\( \mu = \frac{1}{N} \sum_{i=1}^{N} x_i\)
\\
\sphinxhline
\sphinxAtStartPar
Median
&
\sphinxAtStartPar
\( x : P[X \leq x] = .5\)
\\
\sphinxhline
\sphinxAtStartPar
Mode
&
\sphinxAtStartPar
\( Mo  : P[X = Mo] = \max[P(\bf{x})]\)
\\
\sphinxhline
\sphinxAtStartPar
Skewness
&
\sphinxAtStartPar
\( Skewness(X) = \frac{\mu_3}{\mu_2^{3/2}}\) where \(\mu_3\) and \(\mu_2\) = third and second moments about \(\mu\)
\\
\sphinxhline
\sphinxAtStartPar
Standard Deviation
&
\sphinxAtStartPar
\( \sigma = \sqrt{\sum_{i=0}^n (x_i - \bar{x})^2/N}\)
\\
\sphinxhline
\sphinxAtStartPar
Variance
&
\sphinxAtStartPar
\( \sigma^2 = \sum_{i=0}^n (x_i - \bar{x})^2/N\)
\\
\sphinxhline
\sphinxAtStartPar
Probability Distribution
&
\sphinxAtStartPar
\(P(X)\)
\\
\sphinxhline
\sphinxAtStartPar
PDF
&
\sphinxAtStartPar
\(f_X(x) = P(X = x)\)
\\
\sphinxhline
\sphinxAtStartPar
CDF
&
\sphinxAtStartPar
\(F_X(x) = P(X \leq x) = \int_{-\infty}^{x} f(x) \text{d}x\)
\\
\sphinxbottomrule
\end{tabulary}
\sphinxtableafterendhook\par
\sphinxattableend\end{savenotes}


\section{Point Estimate}
\label{\detokenize{PM/jcsrua:point-estimate}}

\subsection{Base}
\label{\detokenize{PM/jcsrua:base}}

\begin{savenotes}\sphinxattablestart
\sphinxthistablewithglobalstyle
\centering
\begin{tabulary}{\linewidth}[t]{TT}
\sphinxtoprule
\sphinxstyletheadfamily 
\sphinxAtStartPar
Base
&\sphinxstyletheadfamily 
\sphinxAtStartPar
Description
\\
\sphinxmidrule
\sphinxtableatstartofbodyhook
\sphinxAtStartPar
Program of Record
&
\sphinxAtStartPar
Defined in the requirements documents
\\
\sphinxhline
\sphinxAtStartPar
Technical Baseline
&
\sphinxAtStartPar
Alternative that reflects a technical assessment
\\
\sphinxhline
\sphinxAtStartPar
What\sphinxhyphen{}If Case
&
\sphinxAtStartPar
For a specific sensitivity analysis
\\
\sphinxbottomrule
\end{tabulary}
\sphinxtableafterendhook\par
\sphinxattableend\end{savenotes}


\subsection{Cost vs Schedule PE}
\label{\detokenize{PM/jcsrua:cost-vs-schedule-pe}}\begin{itemize}
\item {} 
\sphinxAtStartPar
Cost PE \(\leftarrow\) approved WBS structure
\begin{itemize}
\item {} 
\sphinxAtStartPar
Direct cost
\begin{itemize}
\item {} 
\sphinxAtStartPar
Labor

\item {} 
\sphinxAtStartPar
Material

\item {} 
\sphinxAtStartPar
Equipment

\item {} 
\sphinxAtStartPar
Subcontract

\end{itemize}

\item {} 
\sphinxAtStartPar
Indirect cost
\begin{itemize}
\item {} 
\sphinxAtStartPar
Taxes

\item {} 
\sphinxAtStartPar
Gen. Cond.

\item {} 
\sphinxAtStartPar
Risk
\begin{itemize}
\item {} 
\sphinxAtStartPar
Profit

\item {} 
\sphinxAtStartPar
Contingency

\end{itemize}

\item {} 
\sphinxAtStartPar
Overhead

\end{itemize}

\end{itemize}

\item {} 
\sphinxAtStartPar
Schedule \(\leftarrow\) integrated network of activities containing all the detailed discrete work packages and planning packages (or lower\sphinxhyphen{}level tasks of activities) necessary to support the events, accomplishments, and criteria of the project plan.

\end{itemize}


\subsection{Estimating Method}
\label{\detokenize{PM/jcsrua:estimating-method}}\begin{itemize}
\item {} 
\sphinxAtStartPar
Analogy

\item {} 
\sphinxAtStartPar
Parametric

\item {} 
\sphinxAtStartPar
Engineering

\item {} 
\sphinxAtStartPar
Extrapolation

\end{itemize}


\subsection{Implementation Method}
\label{\detokenize{PM/jcsrua:implementation-method}}

\begin{savenotes}\sphinxattablestart
\sphinxthistablewithglobalstyle
\centering
\begin{tabulary}{\linewidth}[t]{TT}
\sphinxtoprule
\sphinxstyletheadfamily 
\sphinxAtStartPar
Modeling Approach
&\sphinxstyletheadfamily 
\sphinxAtStartPar
Description
\\
\sphinxmidrule
\sphinxtableatstartofbodyhook
\sphinxAtStartPar
Function of technical parameters
&
\sphinxAtStartPar
Uncertainty is assigned to the equation itself and to its inputs
\\
\sphinxhline
\sphinxAtStartPar
Factor of another estimate
&
\sphinxAtStartPar
The factor is one form of a parametric equation
\\
\sphinxhline
\sphinxAtStartPar
Level of Effort (LOE)
&
\sphinxAtStartPar
Quantity times the cost per unit, burn rate times a duration, etc.
\\
\sphinxhline
\sphinxAtStartPar
Throughputs
&
\sphinxAtStartPar
Analogies, quotes, subject matter expert opinion, etc.
\\
\sphinxhline
\sphinxAtStartPar
Third Party tool results
&
\sphinxAtStartPar
When moving results from another model or tool into another, it is not enough to import the point estimate; the uncertainty needs to be imported as well
\\
\sphinxbottomrule
\end{tabulary}
\sphinxtableafterendhook\par
\sphinxattableend\end{savenotes}


\section{Sensitivity Analysis}
\label{\detokenize{PM/jcsrua:sensitivity-analysis}}

\subsection{Definition}
\label{\detokenize{PM/jcsrua:definition}}
\sphinxAtStartPar
Systematic approach used to identify the \sphinxstylestrong{impact of potential changes} to one or more of an estimate’s major input parameters to the PE.


\subsection{Objective}
\label{\detokenize{PM/jcsrua:objective}}\begin{enumerate}
\sphinxsetlistlabels{\arabic}{enumi}{enumii}{}{.}%
\item {} 
\sphinxAtStartPar
Vary input parameters over a range of probable values,

\item {} 
\sphinxAtStartPar
Recalculate the PE, and

\item {} 
\sphinxAtStartPar
Determine how sensitive the PE is to changes in the input parameters.

\end{enumerate}


\subsection{Output}
\label{\detokenize{PM/jcsrua:output}}

\begin{savenotes}\sphinxattablestart
\sphinxthistablewithglobalstyle
\centering
\begin{tabulary}{\linewidth}[t]{T}
\sphinxtoprule
\sphinxstyletheadfamily 
\sphinxAtStartPar
\sphinxincludegraphics{{sensitivity_analysis}.svg}
\\
\sphinxmidrule
\sphinxtableatstartofbodyhook
\sphinxAtStartPar
Sensitivity Analysis Output
\\
\sphinxbottomrule
\end{tabulary}
\sphinxtableafterendhook\par
\sphinxattableend\end{savenotes}


\section{Uncertainty Distributions}
\label{\detokenize{PM/jcsrua:uncertainty-distributions}}

\subsection{Overview}
\label{\detokenize{PM/jcsrua:overview}}
\sphinxAtStartPar
The PE:
\begin{itemize}
\item {} 
\sphinxAtStartPar
\(=\) \sphinxstylestrong{one possible estimate} based upon a given set of characteristics

\item {} 
\sphinxAtStartPar
\(\rightarrow\) \sphinxstylestrong{reference point} on which the cost uncertainty analysis is \sphinxstylestrong{anchored}

\item {} 
\sphinxAtStartPar
\(\leftarrow\) approach:
\begin{itemize}
\item {} 
\sphinxAtStartPar
\sphinxstylestrong{objective} (statistical analysis of relevant historical data),

\item {} 
\sphinxAtStartPar
\sphinxstylestrong{subjective} (expert opinion), or

\item {} 
\sphinxAtStartPar
\sphinxstylestrong{third\sphinxhyphen{}party tools} (separate models).

\end{itemize}

\end{itemize}


\subsubsection{Objective}
\label{\detokenize{PM/jcsrua:id1}}

\subsubsection{Definition}
\label{\detokenize{PM/jcsrua:id2}}
\sphinxAtStartPar
\sphinxstylestrong{Objective uncertainty} is an assessment of the uncertainty based on well\sphinxhyphen{}defined statistical processes.


\subsubsection{Methods}
\label{\detokenize{PM/jcsrua:methods}}\begin{itemize}
\item {} 
\sphinxAtStartPar
Developing parametric equations through \sphinxstylestrong{regression analysis}

\item {} 
\sphinxAtStartPar
\sphinxstylestrong{Fitting} theoretical distributions to historical data

\end{itemize}


\paragraph{Regression Analysis}
\label{\detokenize{PM/jcsrua:regression-analysis}}

\paragraph{Curve Fitting}
\label{\detokenize{PM/jcsrua:curve-fitting}}

\subparagraph{Descriptive Statistics}
\label{\detokenize{PM/jcsrua:id3}}
\sphinxAtStartPar
\(\min\) < \(LB\) < \(Mo\) / \(Median\) / \(\mu\)  < \(UB\) < \(\max\) where
\begin{itemize}
\item {} 
\sphinxAtStartPar
\(\min = p_0\)

\item {} 
\sphinxAtStartPar
\(\max = p_{100}\)

\item {} 
\sphinxAtStartPar
\(LB = p_\alpha\)

\item {} 
\sphinxAtStartPar
\(UB = p_{1-\alpha}\)

\end{itemize}


\paragraph{Guidelines}
\label{\detokenize{PM/jcsrua:guidelines}}

\begin{savenotes}\sphinxattablestart
\sphinxthistablewithglobalstyle
\centering
\begin{tabulary}{\linewidth}[t]{TTTT}
\sphinxtoprule
\sphinxstyletheadfamily 
\sphinxAtStartPar
Distribution
&\sphinxstyletheadfamily 
\sphinxAtStartPar
Application
&\sphinxstyletheadfamily 
\sphinxAtStartPar
Parameters
&\sphinxstyletheadfamily 
\sphinxAtStartPar
Recommendation
\\
\sphinxmidrule
\sphinxtableatstartofbodyhook
\sphinxAtStartPar
Lognormal
&
\sphinxAtStartPar
Default
&
\sphinxAtStartPar
2
&
\sphinxAtStartPar
\(Median,UB\)
\\
\sphinxhline
\sphinxAtStartPar
Log\sphinxhyphen{}t
&
\sphinxAtStartPar
Lognormal\(\in n \leq 30\)
&
\sphinxAtStartPar
3
&
\sphinxAtStartPar

\\
\sphinxhline
\sphinxAtStartPar
Triangular
&
\sphinxAtStartPar
Expert opinionFinite \(\min,\max\)Possible skew
&
\sphinxAtStartPar
3
&
\sphinxAtStartPar
\(LB,Mo,UB\)
\\
\sphinxhline
\sphinxAtStartPar
PERT
&
\sphinxAtStartPar
Between Triangular and Beta
&
\sphinxAtStartPar
3
&
\sphinxAtStartPar
\(LB,Mo,UB\)
\\
\sphinxhline
\sphinxAtStartPar
Beta
&
\sphinxAtStartPar
\(\min,\max\) region > \(Mo\)
&
\sphinxAtStartPar
4
&
\sphinxAtStartPar
\(\min,LB,UB,\max\)
\\
\sphinxhline
\sphinxAtStartPar
Normal
&
\sphinxAtStartPar
Equal chance\(LB,UB\)
&
\sphinxAtStartPar
2
&
\sphinxAtStartPar
\(\mu,Median,Mo,UB\)
\\
\sphinxhline
\sphinxAtStartPar
t
&
\sphinxAtStartPar
Normal\(\in n\leq30\)
&
\sphinxAtStartPar
3
&
\sphinxAtStartPar
\(LB,UB\)
\\
\sphinxhline
\sphinxAtStartPar
Uniform
&
\sphinxAtStartPar

&
\sphinxAtStartPar
2
&
\sphinxAtStartPar

\\
\sphinxbottomrule
\end{tabulary}
\sphinxtableafterendhook\par
\sphinxattableend\end{savenotes}


\subsubsection{Subjective}
\label{\detokenize{PM/jcsrua:subjective}}

\subsubsection{Definition}
\label{\detokenize{PM/jcsrua:id4}}
\sphinxAtStartPar
\sphinxstylestrong{Subjective uncertainty} is an assessment of the uncertainty based on expert judgment.


\subsubsection{Conditions}
\label{\detokenize{PM/jcsrua:conditions}}\begin{itemize}
\item {} 
\sphinxAtStartPar
Applies where objective uncertainty distributions are not available

\item {} 
\sphinxAtStartPar
Distributions are characterized by parameters describing their
\begin{itemize}
\item {} 
\sphinxAtStartPar
dispersion (\(LB\), \(UB\))

\item {} 
\sphinxAtStartPar
skewness

\end{itemize}

\item {} 
\sphinxAtStartPar
Elicitation is subject to \sphinxstylestrong{Motivational} and \sphinxstylestrong{Cognitive} biases

\end{itemize}


\paragraph{Biases}
\label{\detokenize{PM/jcsrua:biases}}

\begin{savenotes}\sphinxattablestart
\sphinxthistablewithglobalstyle
\centering
\begin{tabulary}{\linewidth}[t]{TT}
\sphinxtoprule
\sphinxstyletheadfamily 
\sphinxAtStartPar
Category
&\sphinxstyletheadfamily 
\sphinxAtStartPar
Bias
\\
\sphinxmidrule
\sphinxtableatstartofbodyhook
\sphinxAtStartPar
Motivational
&
\sphinxAtStartPar
Social pressure (face\sphinxhyphen{}to\sphinxhyphen{}face)Impression (not face\sphinxhyphen{}to)Group ThinkWishful thinkingCareer goalsMisunderstandingProject AdvocacyCompetitive Pressures
\\
\sphinxhline
\sphinxAtStartPar
Cognitive
&
\sphinxAtStartPar
Representativeness (small sample)Availability (most recent)Anchoring and AdjustmentInconsistency (opinion changes over time)Relating to irrelevant analogiesUnderestimationHuman Nature
\\
\sphinxhline
\sphinxAtStartPar
Other
&
\sphinxAtStartPar
Confirmation BiasPremature TerminationInertiaSelective PerceptionOptimism BiasRecencyRepetition BiasEscalating CommitmentAttribution Asymmetry
\\
\sphinxbottomrule
\end{tabulary}
\sphinxtableafterendhook\par
\sphinxattableend\end{savenotes}


\paragraph{Best Practices}
\label{\detokenize{PM/jcsrua:best-practices}}\begin{itemize}
\item {} 
\sphinxAtStartPar
Have historical \sphinxstylestrong{\(\min\)}, \sphinxstylestrong{\(\max\)}, and \sphinxstylestrong{averages} on hand for discussion

\item {} 
\sphinxAtStartPar
Ask for \sphinxstylestrong{\(LB\)} and \sphinxstylestrong{\(UB\)}

\item {} 
\sphinxAtStartPar
Seek the \sphinxstylestrong{\(Mo\)} value

\item {} 
\sphinxAtStartPar
Select a distribution shape based on \sphinxstylestrong{skew} and \sphinxstylestrong{firmness} of the bounds

\end{itemize}


\paragraph{Adjustment}
\label{\detokenize{PM/jcsrua:adjustment}}
\sphinxAtStartPar
Unless there is no evidence to do otherwise,
\begin{enumerate}
\sphinxsetlistlabels{\arabic}{enumi}{enumii}{}{.}%
\item {} 
\sphinxAtStartPar
Treat subjective bounds (expert opinion) as 70\% range, and

\item {} 
\sphinxAtStartPar
Adjust for skew when using triangular, uniform, or PERT.

\end{enumerate}


\section{Special Considerations}
\label{\detokenize{PM/jcsrua:special-considerations}}\begin{itemize}
\item {} 
\sphinxAtStartPar
Truncate distributions at zero unless there is evidence to do otherwise

\item {} 
\sphinxAtStartPar
Sunk costs should not have uncertainty distributions associated with them

\end{itemize}


\section{PERT}
\label{\detokenize{PM/jcsrua:pert}}

\subsection{Assumptions}
\label{\detokenize{PM/jcsrua:assumptions}}\begin{itemize}
\item {} 
\sphinxAtStartPar
Task.Duration \(\sim\) Beta

\item {} 
\sphinxAtStartPar
CLT applies @ path\sphinxhyphen{}level \(\rightarrow\) Tasks samples are independent and identically distributed (iid)

\end{itemize}


\subsection{Algorithm}
\label{\detokenize{PM/jcsrua:algorithm}}\begin{enumerate}
\sphinxsetlistlabels{\arabic}{enumi}{enumii}{}{.}%
\item {} 
\sphinxAtStartPar
For each task \(i\):
\begin{enumerate}
\sphinxsetlistlabels{\arabic}{enumii}{enumiii}{}{.}%
\item {} 
\sphinxAtStartPar
Define \(LB,Mo,UB\)

\item {} 
\sphinxAtStartPar
Evaluate Task mean: \(\mu = (LB+4Mo + UB)/6\)

\item {} 
\sphinxAtStartPar
Evaluate Task variance: \(\sigma^2 = (UB-LB)/36\)
\begin{itemize}
\item {} 
\sphinxAtStartPar
\(\pm 1\sigma \rightarrow 68.26\% \)

\item {} 
\sphinxAtStartPar
\(\pm 2\sigma \rightarrow 95.46\% \)

\item {} 
\sphinxAtStartPar
\(\pm 3\sigma \rightarrow 99.73\% \)

\end{itemize}

\end{enumerate}

\item {} 
\sphinxAtStartPar
For each path \(j\) apply:
\begin{enumerate}
\sphinxsetlistlabels{\arabic}{enumii}{enumiii}{}{.}%
\item {} 
\sphinxAtStartPar
Evaluate Path mean: \(\mu_j = \sum \mu_i\)

\item {} 
\sphinxAtStartPar
Evaluate Path variance via Variance Sum Law (VSL): \(\sigma^2_j = \sum \sigma^2_i\)

\item {} 
\sphinxAtStartPar
Evaluate Path standard deviation: \(\sigma_j = \sqrt{\sigma^2_i}\)

\end{enumerate}

\item {} 
\sphinxAtStartPar
Apply Central Limit Theorem (CLT): \(P(T <= x) = N^{-1}[(x-\mu)/\sigma]\)

\end{enumerate}


\section{Monte Carlo}
\label{\detokenize{PM/jcsrua:monte-carlo}}

\subsection{Principle}
\label{\detokenize{PM/jcsrua:principle}}
\sphinxAtStartPar
Rrandom sampling = new sample points are generated without taking into account the previously generated sample points.


\subsection{Algorithm}
\label{\detokenize{PM/jcsrua:id5}}\begin{enumerate}
\sphinxsetlistlabels{\arabic}{enumi}{enumii}{}{.}%
\item {} 
\sphinxAtStartPar
For each task:
\begin{enumerate}
\sphinxsetlistlabels{\arabic}{enumii}{enumiii}{}{.}%
\item {} 
\sphinxAtStartPar
Define \sphinxstylestrong{cost} and \sphinxstylestrong{duration} distributions
\begin{itemize}
\item {} 
\sphinxAtStartPar
\(Task.cost \sim C(x)\)

\item {} 
\sphinxAtStartPar
\(Task.duration \sim T(x)\)

\end{itemize}

\item {} 
\sphinxAtStartPar
Set precedence relationships

\end{enumerate}

\item {} 
\sphinxAtStartPar
For \( s=1 \rightarrow S\):
\begin{enumerate}
\sphinxsetlistlabels{\arabic}{enumii}{enumiii}{}{.}%
\item {} 
\sphinxAtStartPar
For each task:
\begin{enumerate}
\sphinxsetlistlabels{\arabic}{enumiii}{enumiv}{}{.}%
\item {} 
\sphinxAtStartPar
Randomize \sphinxstylestrong{cost} and \sphinxstylestrong{duration} (\(Task.cost, Task.duration\))
\begin{itemize}
\item {} 
\sphinxAtStartPar
\(Task.cost \leftarrow C^{-1}[U(0,1)]\)

\item {} 
\sphinxAtStartPar
\(Task.duration \leftarrow T^{-1}[U(0,1)]\)

\end{itemize}

\item {} 
\sphinxAtStartPar
Determine \sphinxstylestrong{start} and \sphinxstylestrong{finish} (\(Task.start, Task.finish\))
\begin{itemize}
\item {} 
\sphinxAtStartPar
\(Task.start = \max \left[ Predecessor.finish \, \text{for} \, Predecessor \, \text{in} \,  \bm{Task.predecessor} \right]\)

\item {} 
\sphinxAtStartPar
\(Task.finish = Task.start\) + \(Task.duration\)

\end{itemize}

\end{enumerate}

\item {} 
\sphinxAtStartPar
Calculate project \sphinxstylestrong{duration} and \sphinxstylestrong{cost}
\begin{itemize}
\item {} 
\sphinxAtStartPar
\(Project.duration = \max \left[ Task.finish \, \text{for} \, Task \, \text{in} \, \bm{Task} \right]\)

\item {} 
\sphinxAtStartPar
\(Project.cost = \sum_{Task} Task.cost\)

\end{itemize}

\end{enumerate}

\end{enumerate}

\sphinxstepscope


\chapter{Line of Balance}
\label{\detokenize{PM/lob:line-of-balance}}\label{\detokenize{PM/lob::doc}}
\sphinxstepscope


\chapter{Critical Chain Method}
\label{\detokenize{PM/ccm:critical-chain-method}}\label{\detokenize{PM/ccm::doc}}
\sphinxstepscope


\chapter{Earned Value Management}
\label{\detokenize{PM/evm:earned-value-management}}\label{\detokenize{PM/evm::doc}}

\section{Terms, Definitions, and Abbreviations}
\label{\detokenize{PM/evm:terms-definitions-and-abbreviations}}

\begin{savenotes}\sphinxattablestart
\sphinxthistablewithglobalstyle
\centering
\begin{tabulary}{\linewidth}[t]{TT}
\sphinxtoprule
\sphinxstyletheadfamily 
\sphinxAtStartPar
Terms
&\sphinxstyletheadfamily 
\sphinxAtStartPar
Description
\\
\sphinxmidrule
\sphinxtableatstartofbodyhook
\sphinxAtStartPar
Activity
&
\sphinxAtStartPar
Identified piece of work that is required to be undertaken to complete a project or programme
\\
\sphinxhline
\sphinxAtStartPar
Actual Cost (of Work Performed)
&
\sphinxAtStartPar
Cost incurred for work performed
\\
\sphinxhline
\sphinxAtStartPar
Budget at Completion
&
\sphinxAtStartPar
Total forecasted cost for accomplishing the work related to a work package, activity, or control account
\\
\sphinxhline
\sphinxAtStartPar
Control Account
&
\sphinxAtStartPar
Management control point where scope, budget, actual cost, and schedule of a project or programme, work package, or activity are integrated
\\
\sphinxhline
\sphinxAtStartPar
Earned Value (Budgeted cost of Work Performed)
&
\sphinxAtStartPar
Value of completed work expressed in terms of the budget assigned to that work
\\
\sphinxhline
\sphinxAtStartPar
Earned Value Management
&
\sphinxAtStartPar
The method that integrates project or programme scope, actual cost, budget, and schedule for assessment of progress and performance
\\
\sphinxhline
\sphinxAtStartPar
Estimate at Completion
&
\sphinxAtStartPar
Forecasted total cost to accomplish the work on project, programme, work package, or activity
\\
\sphinxhline
\sphinxAtStartPar
Estimate to Complete
&
\sphinxAtStartPar
Forecasted cost of the work remaining of a project, programme, work package, or activity
\\
\sphinxhline
\sphinxAtStartPar
Integrated Baseline Review
&
\sphinxAtStartPar
Assessment to establish a common understanding of the performance measurement baseline for verification of the technical content of the project or programme
\\
\sphinxhline
\sphinxAtStartPar
Management Reserve
&
\sphinxAtStartPar
Amount of budget external to the performance measurement baseline, withheld for management control in response to unforeseen events or activities that are a part of the scope
\\
\sphinxhline
\sphinxAtStartPar
Network Schedule (Diagram)
&
\sphinxAtStartPar
Graphical representation indicating the logic sequencing and interdependencies of the work elements of a project or programme
\\
\sphinxhline
\sphinxAtStartPar
Performance Measurement
&
\sphinxAtStartPar
Quantitative units of measure that are placed to track progress
\\
\sphinxhline
\sphinxAtStartPar
Performance Measurement Baseline
&
\sphinxAtStartPar
Total time\sphinxhyphen{}phased scope of work and budget plan against which project or programme performance is measured, not including management reserve
\\
\sphinxhline
\sphinxAtStartPar
Planned Value (Budgeted Cost of Work Scheduled)
&
\sphinxAtStartPar
Time\sphinxhyphen{}phased budget authorized for the work scheduled
\\
\sphinxhline
\sphinxAtStartPar
Technical Performance
&
\sphinxAtStartPar
The measure of the results of functionalities or capabilities achieved for the project or programme during implementation
\\
\sphinxhline
\sphinxAtStartPar
Time\sphinxhyphen{}phased Budget
&
\sphinxAtStartPar
Allocation of the cost to accomplish the work over established periods of time or phases
\\
\sphinxhline
\sphinxAtStartPar
Undistributed Budget
&
\sphinxAtStartPar
Cost for authorized work that has not been distributed to a control account
\\
\sphinxhline
\sphinxAtStartPar
Work Breakdown Structure
&
\sphinxAtStartPar
Decomposition of the defined scope of the project or programme into progressively lower levels consisting of elements of work
\\
\sphinxhline
\sphinxAtStartPar
Work Package
&
\sphinxAtStartPar
One or more groups of related activities that are within the control account
\\
\sphinxbottomrule
\end{tabulary}
\sphinxtableafterendhook\par
\sphinxattableend\end{savenotes}


\section{Overview}
\label{\detokenize{PM/evm:overview}}

\subsection{Earned Value Management}
\label{\detokenize{PM/evm:id1}}
\sphinxAtStartPar
Earned value management (EVM)
\begin{itemize}
\item {} 
\sphinxAtStartPar
is a structured method used to provide a performance measurement system for review of past and forecasted performance of a project or programme;

\item {} 
\sphinxAtStartPar
is a method of performance management.

\end{itemize}

\sphinxAtStartPar
Performance management should provide for the planning, implementing, and controlling of the performance of a project or programme in accomplishing the scope of work of the project or programme.


\subsection{Purpose and Benefits of EVM}
\label{\detokenize{PM/evm:purpose-and-benefits-of-evm}}
\sphinxAtStartPar
The purpose of EVM is to control and analyze the project or programme.

\sphinxAtStartPar
EVM facilitates analysis and decision\sphinxhyphen{}making for but is not limited to budget, schedule, human resources, and materials.

\sphinxAtStartPar
The EVM system may include communication of the status from metrics established for the project or programme, improvements, corrective action development, and a common framework and vocabulary. \\
An EVM is a set of procedures, tools, and methods for establishing and maintaining project or programme control.

\sphinxAtStartPar
The application of EVM should result in three overall benefits:
\begin{enumerate}
\sphinxsetlistlabels{\arabic}{enumi}{enumii}{}{.}%
\item {} 
\sphinxAtStartPar
developing objective measurement techniques;

\item {} 
\sphinxAtStartPar
availability of data for project or programme management decisions;

\item {} 
\sphinxAtStartPar
providing a system to monitor the project or programme.

\end{enumerate}

\sphinxAtStartPar
Specific benefits may include, but are not limited to, the following:
\begin{itemize}
\item {} 
\sphinxAtStartPar
forecasting of future performance and EAC based on past performance;

\item {} 
\sphinxAtStartPar
objective metrics for comparison of project or programme performance across an organization and between or among organizations;

\item {} 
\sphinxAtStartPar
development of budgets and baselines;

\item {} 
\sphinxAtStartPar
compilation of estimates;

\item {} 
\sphinxAtStartPar
objective measurement of completion of work packages that is done consistently;

\item {} 
\sphinxAtStartPar
comparison of work performed against actual performance and budget;

\item {} 
\sphinxAtStartPar
highlighting inconsistencies in the measures in EV reports;

\item {} 
\sphinxAtStartPar
consistency of the reporting and performance measurement framework by regular earned value reporting.

\end{itemize}


\subsection{Guidelines for an EVM System}
\label{\detokenize{PM/evm:guidelines-for-an-evm-system}}
\sphinxAtStartPar
An EVM  system should provide consistent performance metrics. \\
To achieve a consistent view of performance metrics, the system should integrate the baselines established for the project or programme including the scope of work defined through the work breakdown structure and performance measurement baseline. \\
The system should also allow for formal, controlled incorporation of changes in baselines, authorized users, and procedures.

\sphinxAtStartPar
To implement an EVM system,
\begin{itemize}
\item {} 
\sphinxAtStartPar
the project or programme control metrics and processes should be documented and understood by the organization or organizations doing the work;

\item {} 
\sphinxAtStartPar
the system should be established to allow systematic review of the data, common assessment methodologies, targeted levels of performance, and an assessment feedback process;

\item {} 
\sphinxAtStartPar
the system may be tailored to accommodate different project or programme subject area integration, more than one organization reporting, and another tailoring as deemed necessary to control the project or programme or provide an integrated programme view.

\end{itemize}

\sphinxAtStartPar
The core data contained in an earned value management system should be the earned value, actual cost, the planned value, estimate to complete, and budget at completion. \\
The EVM system should be able to show the planned status and the actual status of the project or programme.

\sphinxAtStartPar
To implement an earned value management system, the system requires a common agreement on the assignment of “value” and “performance”, which may be tailored for projects or programmes based on organizational considerations.

\sphinxAtStartPar
The review of metrics of performance should be accomplished on a regular, scheduled basis to allow for comparison and analysis of performance.

\sphinxAtStartPar
An EVM system should be able to do the following:
\begin{enumerate}
\sphinxsetlistlabels{\arabic}{enumi}{enumii}{}{.}%
\item {} 
\sphinxAtStartPar
determine what work is to be accomplished, by whom and when;

\item {} 
\sphinxAtStartPar
establish resource requirements;

\item {} 
\sphinxAtStartPar
measure work achievement and record associated costs;

\item {} 
\sphinxAtStartPar
report deviations from the plan for which metrics have been established;

\item {} 
\sphinxAtStartPar
forecast the completion date and cost;

\item {} 
\sphinxAtStartPar
plan and implement corrective and preventive action plans;

\item {} 
\sphinxAtStartPar
authorize scope changes; any approved changes to the prior approved baselines contained in the earned value management system should be controlled, traceable, and documented.

\end{enumerate}


\subsection{EVM Planning}
\label{\detokenize{PM/evm:evm-planning}}
\sphinxAtStartPar
EVM planning should enable:
\begin{itemize}
\item {} 
\sphinxAtStartPar
establishing project or programme objectives, as well as the integrated view of the planning of the overall projects or programmes,

\item {} 
\sphinxAtStartPar
monitoring of project or programme progress to measure deviations from the plan, and

\item {} 
\sphinxAtStartPar
planning by the users of the performance management system for project or programme, objective assessment of progress, and use of resources.

\end{itemize}


\subsection{Using EVM Measurements and Performance Metrics}
\label{\detokenize{PM/evm:using-evm-measurements-and-performance-metrics}}
\sphinxAtStartPar
EV should be used to determine performance metrics to assess the status of a project or programme at a selected point in time. \\
These metrics should enable informed decisions about the management of the project or programme. \\
The metrics derived may be used to compare actual project or programme cost and schedule performance with the PMB.

\sphinxAtStartPar
The PMB should establish variance thresholds for cost and schedule that, when exceeded, identify significant variances for further analysis and management attention.

\sphinxAtStartPar
The information acquired by using the earned value performance measurements should be used to determine:
\begin{itemize}
\item {} 
\sphinxAtStartPar
progress of a project or programme,

\item {} 
\sphinxAtStartPar
progress towards work accomplished,

\item {} 
\sphinxAtStartPar
completion of the deliverables, and

\item {} 
\sphinxAtStartPar
progress towards delivery of a project or programme.
These measurements, combined with the agreed\sphinxhyphen{}upon variance threshold, should be used to determine the cost and schedule variances and cost and schedule performance indices. \\
The information should also be used to forecast the future performance of the project or programme.

\end{itemize}


\section{EVM Process Steps}
\label{\detokenize{PM/evm:evm-process-steps}}

\subsection{General}
\label{\detokenize{PM/evm:general}}
\sphinxAtStartPar
The earned value management process steps are shown in Figure 1.


\begin{savenotes}\sphinxattablestart
\sphinxthistablewithglobalstyle
\centering
\begin{tabulary}{\linewidth}[t]{T}
\sphinxtoprule
\sphinxstyletheadfamily 
\sphinxAtStartPar
\sphinxincludegraphics{{FigEVM-1.drawio}.png}
\\
\sphinxmidrule
\sphinxtableatstartofbodyhook
\sphinxAtStartPar
Figure 1 — EVM process steps
\\
\sphinxbottomrule
\end{tabulary}
\sphinxtableafterendhook\par
\sphinxattableend\end{savenotes}


\section{Cost and Schedule Performance Analysis}
\label{\detokenize{PM/evm:cost-and-schedule-performance-analysis}}

\subsection{General}
\label{\detokenize{PM/evm:id2}}
\sphinxAtStartPar
Analysis of performance metrics is an important approach to measuring and understanding the current period and overall project or programme performance.
The use of performance metrics should allow corrective and preventive actions to be implemented, thereby improving the use of the EVM system data.


\subsection{Performance Measurement Indicators and Predictors}
\label{\detokenize{PM/evm:performance-measurement-indicators-and-predictors}}
\sphinxAtStartPar
To understand performance measurement analysis, it is important to know the metrics of indicators and predictors earned value uses.
Table 1 summarizes the key EV cost performance indicators and predictors.


\begin{savenotes}\sphinxattablestart
\sphinxthistablewithglobalstyle
\centering
\begin{tabulary}{\linewidth}[t]{TTT}
\sphinxtoprule
\sphinxstyletheadfamily 
\sphinxAtStartPar
Acronym
&\sphinxstyletheadfamily 
\sphinxAtStartPar
Name
&\sphinxstyletheadfamily 
\sphinxAtStartPar
Formula
\\
\sphinxmidrule
\sphinxtableatstartofbodyhook
\sphinxAtStartPar
\(t\)
&
\sphinxAtStartPar
Time
&
\sphinxAtStartPar

\\
\sphinxhline
\sphinxAtStartPar
\(PD\)
&
\sphinxAtStartPar
Planned Duration
&
\sphinxAtStartPar
\(\sum_{t} t\)
\\
\sphinxhline
\sphinxAtStartPar
\(WS{(t)}\)
&
\sphinxAtStartPar
Percentage of Work Scheduled
&
\sphinxAtStartPar

\\
\sphinxhline
\sphinxAtStartPar
\(WP{(t)}\)
&
\sphinxAtStartPar
Percentage of Work Performed
&
\sphinxAtStartPar

\\
\sphinxhline
\sphinxAtStartPar
\(AC{(t)}\)
&
\sphinxAtStartPar
Actual Cost of Work Performed
&
\sphinxAtStartPar

\\
\sphinxhline
\sphinxAtStartPar
\(EV{(t)}\)
&
\sphinxAtStartPar
Earned Value
&
\sphinxAtStartPar
\(BAC \cdot WP{(t)}\)
\\
\sphinxhline
\sphinxAtStartPar
\(PV{(t)}\)
&
\sphinxAtStartPar
Planned Value
&
\sphinxAtStartPar
\(BAC \cdot WS{(t)}\)
\\
\sphinxhline
\sphinxAtStartPar
\(BAC\)
&
\sphinxAtStartPar
Budget at Completion
&
\sphinxAtStartPar
\(\sum_{t} PV{(t)}\)
\\
\sphinxhline
\sphinxAtStartPar
\(CV{(t)}\)
&
\sphinxAtStartPar
Cost Variance
&
\sphinxAtStartPar
\(EV{(t)} - AC{(t)}\)
\\
\sphinxhline
\sphinxAtStartPar
\(SV{(t)}\)
&
\sphinxAtStartPar
Schedule Variance
&
\sphinxAtStartPar
\(EV{(t)} - SV{(t)}\)
\\
\sphinxhline
\sphinxAtStartPar
\(CPI{(t)}\)
&
\sphinxAtStartPar
Cost Performance Index
&
\sphinxAtStartPar
\(EV{(t)} / AC{(t)}\)
\\
\sphinxhline
\sphinxAtStartPar
\(SPI{(t)}\)
&
\sphinxAtStartPar
Schedule Performance Index
&
\sphinxAtStartPar
\(EV{(t)} / PV{(t)}\)
\\
\sphinxhline
\sphinxAtStartPar
\(EAC{(t)}\)
&
\sphinxAtStartPar
Estimate at Completion
&
\sphinxAtStartPar
\(EAC{(t)} = AC{(t)} + ETC{(t)}\)
\\
\sphinxhline
\sphinxAtStartPar
\(ETC{(t)}\)
&
\sphinxAtStartPar
Estimate to Complete
&
\sphinxAtStartPar
\(ETC{(t)} = [BAC - EV{(t)}]/PF{(t)}\)
\\
\sphinxhline
\sphinxAtStartPar
\(TCPI{(t)}\)
&
\sphinxAtStartPar
To\sphinxhyphen{}Complete Performance Index
&
\sphinxAtStartPar
\(TCPI{(t)} = [BAC - EV{(t)}]/[BAC - AC{(t)}]\)
\\
\sphinxhline
\sphinxAtStartPar

&
\sphinxAtStartPar

&
\sphinxAtStartPar
\(TCPI{(t)} = [EAC{(t)} - EV{(t)}]/[BAC - AC{(t)}]\)
\\
\sphinxhline
\sphinxAtStartPar
\(ES{(t)}\)
&
\sphinxAtStartPar
Earned Schedule
&
\sphinxAtStartPar
\(ES{(t)} = \begin{Bmatrix} t^* : PV(t^*) = EV{(t)}\end{Bmatrix}\)
\\
\sphinxhline
\sphinxAtStartPar
\(SV^{t}{(t)}\)
&
\sphinxAtStartPar
Earned Schedule Variance
&
\sphinxAtStartPar
\(SV^\text{t}{(t)} = ES{(t)} - t\)
\\
\sphinxhline
\sphinxAtStartPar
\(SPI^\text{t}{(t)}\)
&
\sphinxAtStartPar
Earned Schedule Performance Index
&
\sphinxAtStartPar
\(SPI^\text{t}{(t)} = ES{(t)} / t\)
\\
\sphinxhline
\sphinxAtStartPar
\(TEAC\)
&
\sphinxAtStartPar
Time Estimate at Completion
&
\sphinxAtStartPar
\(TEAC{(t)} = AT + ETC^\text{t}{(t)} / PF^\text{t}{(t)}\)
\\
\sphinxhline
\sphinxAtStartPar
\(ETC^\text{t}{(t)}\)
&
\sphinxAtStartPar
Time Estimate to Complete
&
\sphinxAtStartPar
\(ETC^\text{t}{(t)} = [PD - ES{(t)}]/PF^\text{t}{(t)}\)
\\
\sphinxhline
\sphinxAtStartPar
\(TCPI^\text{t}{(t)}\)
&
\sphinxAtStartPar
To\sphinxhyphen{}Complete \(ES\) Performance Index
&
\sphinxAtStartPar
\(TCPI^\text{t}{(t)} = [PD - ES{(t)}]/(PD - t)\)
\\
\sphinxhline
\sphinxAtStartPar

&
\sphinxAtStartPar

&
\sphinxAtStartPar
\(TCPI^\text{t}{(t)} = [TEAC{(t)} - ES{(t)}]/(PD - t)\)
\\
\sphinxbottomrule
\end{tabulary}
\sphinxtableafterendhook\par
\sphinxattableend\end{savenotes}


\subsection{Benefits of Performance Measurement Analysis}
\label{\detokenize{PM/evm:benefits-of-performance-measurement-analysis}}
\sphinxAtStartPar
Performance measurement analysis contributes to trend analysis over time; to highlight trends in cost over\sphinxhyphen{} or under\sphinxhyphen{}run; and be superimposed with contracted cost outcomes, risk confidence limits, and benefit realization data to provide a more comprehensive picture of overall project or programme cost performance.

\sphinxAtStartPar
Large, early, unfavorable schedule variances, irrespective of cost performance, may be a reliable warning of a project experiencing significant performance issues that should be investigated to determine the causes and implement corrective and preventive recovery actions.


\subsection{Cost Performance Measurements}
\label{\detokenize{PM/evm:cost-performance-measurements}}
\sphinxAtStartPar
A negative \(CV\) and \(CPI\) of less than one indicates an unfavorable over\sphinxhyphen{}budget condition, as the \(EV\) accrued is less than the costs incurred. \\
Conversely, a positive \(CV\) and \(CPI\) greater than one indicates an under\sphinxhyphen{}spend condition.

\sphinxAtStartPar
Significant negative or positive CVs should be investigated to determine the causes and initiate discussions as to whether to implement corrective and preventive actions to prevent further deterioration and implement possible cost recovery.

\sphinxAtStartPar
The \(CPI\) is a ratio that provides a measure of the cost efficiency achieved by the project to date. \\
The \(TCPI\) is a forward\sphinxhyphen{}looking predictor that calculates the future cost efficiency required to achieve specific cost objectives at project or programme completion. \\
The usual cost objectives analyzed should be the project or programme approved \(BAC\) and, if different, the project or programme manager’s claimed cost estimate at completion.

\sphinxAtStartPar
The \(EAC\) is a predictive measure that calculates predicted completion costs based on the historical performance of the project to date. \\
One should produce a range of calculated completion cost outcomes using the \(CPI\), \(SPI\), or both indices in combination as performance factors. \\
These outcomes may be used to assess the likelihood of the project or programme achieving the approved BAC, and the project or programme manager’s claimed cost \(EAC\).


\subsection{Schedule Performance Measurements}
\label{\detokenize{PM/evm:schedule-performance-measurements}}
\sphinxAtStartPar
The traditional earned value schedule performance measurements are also cost\sphinxhyphen{}based measures that compare the volume of work performed to the volume of work planned.

\sphinxAtStartPar
A negative \(SV\) and \(SPI\) of less than one indicates an unfavorable behind\sphinxhyphen{}schedule condition, subject to confirmatory analysis of the network schedule, as the volume of work accomplished, \(EV\) accrued, is less than the volume of work planned, PV. \\
Conversely, a positive \(SV\) and \(SPI\) greater than one may indicate an ahead\sphinxhyphen{}of\sphinxhyphen{}schedule condition, subject to confirmatory analysis of the project network schedule.

\sphinxAtStartPar
Significant SVs should be investigated to understand the causes and to implement corrective and preventive actions. \\
Analysis of the earned value schedule metrics should be undertaken in conjunction with analyzing the network schedule, which remains the primary source of time\sphinxhyphen{}based information. \\
The critical path impact of negative \(EV\) schedule variances should be analyzed in conjunction with the network schedule.

\sphinxAtStartPar
The predictive utility of the \(SV\) will be lost in the final third of the project.

\sphinxAtStartPar
As the project or programme gets closer to completion, the \(SV\) of the project or programme moves closer to zero, otherwise known as reversion to zero.


\subsection{Earned Schedule}
\label{\detokenize{PM/evm:earned-schedule}}
\sphinxAtStartPar
\sphinxstylestrong{Earned Schedule (ES)} is an extension of EVM.

\sphinxAtStartPar
ES calculates schedule metrics and indicators on the time axis rather than on the cost axis traditionally utilized by the EVM schedule metrics.

\sphinxAtStartPar
The basis of ES is to identify the time increment at which the amount of \(EV\) accrued should have been earned.

\sphinxAtStartPar
The time increment can be any selected unit of accrual for the \(EV\), such as a week, month, or any other time period. \\
Once this value has been determined, a series of time\sphinxhyphen{}based metrics can be calculated, which emulates the \(EV\) cost\sphinxhyphen{}based counterparts.

\sphinxAtStartPar
The ES metric, \(SV^\text{t}\), should be calculated by reference to the actual time, that is, the time increment at the project or programme status date. \\
The actual time, \(t\), is an unconstrained measure, in the same way as \(AC\).

\sphinxAtStartPar
The ES metrics should be reliable for early and late finish projects or programmes, as \(SV^\text{t}\) only equals zero and \(SPI^\text{t}\) only equals one at the project or programme completion if on\sphinxhyphen{}time completion has been achieved.

\sphinxAtStartPar
Analysis of the ES metrics should be undertaken in conjunction with an analysis of the network schedule, which remains the primary source of time\sphinxhyphen{}based information.


\begin{savenotes}\sphinxattablestart
\sphinxthistablewithglobalstyle
\centering
\begin{tabulary}{\linewidth}[t]{T}
\sphinxtoprule
\sphinxstyletheadfamily 
\sphinxAtStartPar
\sphinxincludegraphics{{FigEVM-2.drawio}.png}
\\
\sphinxmidrule
\sphinxtableatstartofbodyhook
\sphinxAtStartPar
Figure 2 — Cost, Schedule, and Earned Schedule Variances
\\
\sphinxbottomrule
\end{tabulary}
\sphinxtableafterendhook\par
\sphinxattableend\end{savenotes}


\subsection{Summary}
\label{\detokenize{PM/evm:summary}}


\sphinxstepscope


\chapter{Risk Management}
\label{\detokenize{PM/rm:risk-management}}\label{\detokenize{PM/rm::doc}}

\section{Principles}
\label{\detokenize{PM/rm:principles}}
\sphinxAtStartPar
The purpose of \sphinxstylestrong{risk management (RM)} is the creation and protection of value. \\
It improves performance, encourages innovation, and supports the achievement of objectives.

\sphinxAtStartPar
The principles
\begin{itemize}
\item {} 
\sphinxAtStartPar
guide the characteristics of effective and efficient RM, communicating its value and explaining its intention and purpose. \textbackslash{}

\item {} 
\sphinxAtStartPar
are the foundation for managing risk and should be considered when establishing the organization’s RM framework and processes.

\end{itemize}

\sphinxAtStartPar
These principles should enable an organization to manage the effects of uncertainty on its objectives.

\sphinxAtStartPar
Effective RM requires the following elements and can be further explained as follows.


\begin{savenotes}\sphinxattablestart
\sphinxthistablewithglobalstyle
\centering
\begin{tabulary}{\linewidth}[t]{TT}
\sphinxtoprule
\sphinxstyletheadfamily 
\sphinxAtStartPar
Principle
&\sphinxstyletheadfamily 
\sphinxAtStartPar
Description
\\
\sphinxmidrule
\sphinxtableatstartofbodyhook
\sphinxAtStartPar
Integrated
&
\sphinxAtStartPar
RM is an integral part of all organizational activities
\\
\sphinxhline
\sphinxAtStartPar
Structured and Comprehensive
&
\sphinxAtStartPar
A structured and comprehensive approach to RM contributes to consistent and comparable results
\\
\sphinxhline
\sphinxAtStartPar
Customized
&
\sphinxAtStartPar
The RM framework and process are customized and proportionate to the organization’s external and internal context related to its objectives
\\
\sphinxhline
\sphinxAtStartPar
Inclusive
&
\sphinxAtStartPar
Appropriate and timely involvement of stakeholders enables their knowledge, views, and perceptions to be considered. This results in improved awareness and informed RM
\\
\sphinxhline
\sphinxAtStartPar
Dynamic
&
\sphinxAtStartPar
Risks can emerge, change or disappear as an organization’s external and internal context changes. RM anticipates, detects, acknowledges, and responds to those changes and events in an appropriate and timely manner
\\
\sphinxhline
\sphinxAtStartPar
Best available information
&
\sphinxAtStartPar
The inputs to RM are based on historical and current information and future expectations. RM considers any limitations and uncertainties associated with such information and expectations. Information should be timely, transparent, and available to relevant stakeholders
\\
\sphinxhline
\sphinxAtStartPar
Human and cultural factors
&
\sphinxAtStartPar
Human behavior and culture significantly influence all aspects of RM at each level and stage
\\
\sphinxhline
\sphinxAtStartPar
Continual improvement
&
\sphinxAtStartPar
RM is continually improved through learning and experience
\\
\sphinxbottomrule
\end{tabulary}
\sphinxtableafterendhook\par
\sphinxattableend\end{savenotes}


\section{Framework}
\label{\detokenize{PM/rm:framework}}

\section{Process}
\label{\detokenize{PM/rm:process}}

\subsection{General}
\label{\detokenize{PM/rm:general}}
\sphinxAtStartPar
The RM process involves
\begin{itemize}
\item {} 
\sphinxAtStartPar
systematically applying policies, procedures, and practices to communicating and consulting,

\item {} 
\sphinxAtStartPar
establishing the context, and

\item {} 
\sphinxAtStartPar
assessing, treating, monitoring, reviewing, recording, and reporting risk.

\end{itemize}

\sphinxAtStartPar
This process is illustrated in Figure 1.


\begin{savenotes}\sphinxattablestart
\sphinxthistablewithglobalstyle
\centering
\begin{tabulary}{\linewidth}[t]{T}
\sphinxtoprule
\sphinxstyletheadfamily 
\sphinxAtStartPar
\sphinxincludegraphics{{FigRisk-1.drawio}.png}
\\
\sphinxmidrule
\sphinxtableatstartofbodyhook
\sphinxAtStartPar
Figure 1 — Process
\\
\sphinxbottomrule
\end{tabulary}
\sphinxtableafterendhook\par
\sphinxattableend\end{savenotes}


\subsection{Communication \& Consultation}
\label{\detokenize{PM/rm:communication-consultation}}
\sphinxAtStartPar
The purpose of communication and consultation is to assist relevant stakeholders in understanding
\begin{itemize}
\item {} 
\sphinxAtStartPar
risk,

\item {} 
\sphinxAtStartPar
the basis on which decisions are made, and

\item {} 
\sphinxAtStartPar
why particular actions are required.

\end{itemize}

\sphinxAtStartPar
Communication promotes awareness and understanding of risk, whereas consultation involves obtaining feedback and information to support decision\sphinxhyphen{}making.

\sphinxAtStartPar
Close coordination between the two should facilitate the factual, timely, relevant, accurate, and understandable exchange of information, taking into account
\begin{itemize}
\item {} 
\sphinxAtStartPar
the confidentiality and integrity of information, and

\item {} 
\sphinxAtStartPar
the privacy rights of individuals.

\end{itemize}

\sphinxAtStartPar
Communication and consultation with appropriate external and internal stakeholders should occur within and throughout all steps of the RM process.

\sphinxAtStartPar
Communication and consultation aim to:
\begin{itemize}
\item {} 
\sphinxAtStartPar
bring different areas of expertise together for each step of the RM process;

\item {} 
\sphinxAtStartPar
ensure that different views are appropriately considered when defining risk criteria and when evaluating risks;

\item {} 
\sphinxAtStartPar
provide sufficient information to facilitate risk oversight and decision\sphinxhyphen{}making;

\item {} 
\sphinxAtStartPar
build a sense of inclusiveness and ownership among those affected by risk.

\end{itemize}


\subsection{Scope, Context, and Criteria}
\label{\detokenize{PM/rm:scope-context-and-criteria}}

\subsubsection{General}
\label{\detokenize{PM/rm:id1}}
\sphinxAtStartPar
The purpose of establishing the scope, context, and criteria is to customize the RM process, enabling effective risk assessment and appropriate risk treatment.

\sphinxAtStartPar
Scope, context, and criteria involve
\begin{itemize}
\item {} 
\sphinxAtStartPar
defining the scope of the process, and

\item {} 
\sphinxAtStartPar
understanding the external and internal context.

\end{itemize}


\subsubsection{Defining the Scope}
\label{\detokenize{PM/rm:defining-the-scope}}
\sphinxAtStartPar
The organization should define the scope of its RM activities.

\sphinxAtStartPar
As the RM process may be applied at different levels (e.g. strategic, operational, programme, project, or other activities), it is important to be clear about
\begin{itemize}
\item {} 
\sphinxAtStartPar
the scope under consideration,

\item {} 
\sphinxAtStartPar
the relevant objectives to be considered, and

\item {} 
\sphinxAtStartPar
their alignment with organizational objectives.

\end{itemize}

\sphinxAtStartPar
When planning the approach, considerations include:
\begin{itemize}
\item {} 
\sphinxAtStartPar
objectives and decisions that need to be made;

\item {} 
\sphinxAtStartPar
outcomes expected from the steps to be taken in the process;

\item {} 
\sphinxAtStartPar
time, location, specific inclusions, and exclusions;

\item {} 
\sphinxAtStartPar
appropriate risk assessment tools and techniques;

\item {} 
\sphinxAtStartPar
resources required, responsibilities and records to be kept;

\item {} 
\sphinxAtStartPar
relationships with other projects, processes, and activities.

\end{itemize}


\subsubsection{External and Internal Context}
\label{\detokenize{PM/rm:external-and-internal-context}}
\sphinxAtStartPar
The external and internal context is the environment in which the organization seeks to define and achieve its objectives.

\sphinxAtStartPar
The context of the RM process
\begin{itemize}
\item {} 
\sphinxAtStartPar
should be established by understanding the external and internal environment in which the organization operates, and

\item {} 
\sphinxAtStartPar
should reflect the specific environment of the activity to which the RM process is to be applied.

\end{itemize}

\sphinxAtStartPar
Understanding the context is important because:
\begin{itemize}
\item {} 
\sphinxAtStartPar
RM takes place in the context of the objectives and activities of the organization;

\item {} 
\sphinxAtStartPar
organizational factors can be a source of risk;

\item {} 
\sphinxAtStartPar
the purpose and scope of the RM process may be interrelated with the organization’s objectives as a whole.

\end{itemize}

\sphinxAtStartPar
The organization should establish the RM processes in external and internal contexts by considering the factors mentioned earlier.


\subsubsection{Defining Risk Criteria}
\label{\detokenize{PM/rm:defining-risk-criteria}}
\sphinxAtStartPar
The organization should
\begin{itemize}
\item {} 
\sphinxAtStartPar
specify the amount and type of risk it may or may not take relative to objectives,

\item {} 
\sphinxAtStartPar
define criteria to evaluate risk’s significance, and support decision\sphinxhyphen{}making processes. \textbackslash{}

\end{itemize}

\sphinxAtStartPar
Risk criteria should
\begin{itemize}
\item {} 
\sphinxAtStartPar
be aligned with the RM framework and customized to the specific purpose and scope of the activity under consideration,

\item {} 
\sphinxAtStartPar
reflect the organization’s values, objectives, and resources,

\item {} 
\sphinxAtStartPar
be consistent with policies and statements about RM,

\item {} 
\sphinxAtStartPar
be defined considering the organization’s obligations and stakeholders’ views, and

\item {} 
\sphinxAtStartPar
be dynamic and continually reviewed and amended, if necessary.

\end{itemize}

\sphinxAtStartPar
To set risk criteria, the following should be considered:
\begin{itemize}
\item {} 
\sphinxAtStartPar
the nature and type of uncertainties that can affect outcomes and objectives (both tangible and intangible);

\item {} 
\sphinxAtStartPar
how consequences (both positive and negative) and likelihood will be defined and measured;

\item {} 
\sphinxAtStartPar
time\sphinxhyphen{}related factors;

\item {} 
\sphinxAtStartPar
consistency in the use of measurements;

\item {} 
\sphinxAtStartPar
how the level of risk is to be determined;

\item {} 
\sphinxAtStartPar
how combinations and sequences of multiple risks will be taken into account;

\item {} 
\sphinxAtStartPar
the organization’s capacity.

\end{itemize}


\subsection{Risk Assessment}
\label{\detokenize{PM/rm:risk-assessment}}
\sphinxAtStartPar
Risk assessment is the overall process of risk identification, analysis, and evaluation.

\sphinxAtStartPar
Risk assessment should be conducted systematically, iteratively, and collaboratively, drawing on the knowledge and views of stakeholders. \\
It should use the best available information, supplemented by further inquiry.


\subsubsection{General}
\label{\detokenize{PM/rm:id2}}

\subsubsection{Risk Identification}
\label{\detokenize{PM/rm:risk-identification}}
\sphinxAtStartPar
The purpose of risk identification is to find, recognize and describe risks that might help or prevent an organization from achieving its objectives. \\
Relevant, appropriate and up\sphinxhyphen{}to\sphinxhyphen{}date information is important in identifying risks.

\sphinxAtStartPar
The organization can use various techniques to identify uncertainties affecting one or more objectives. \\
The following factors, and the relationship between these factors, should be considered:
\begin{itemize}
\item {} 
\sphinxAtStartPar
tangible and intangible sources of risk;

\item {} 
\sphinxAtStartPar
causes and events;

\item {} 
\sphinxAtStartPar
threats and opportunities;

\item {} 
\sphinxAtStartPar
vulnerabilities and capabilities;

\item {} 
\sphinxAtStartPar
changes in the external and internal context;

\item {} 
\sphinxAtStartPar
indicators of emerging risks;

\item {} 
\sphinxAtStartPar
the nature and value of assets and resources;

\item {} 
\sphinxAtStartPar
consequences and their impact on objectives;

\item {} 
\sphinxAtStartPar
limitations of knowledge and reliability of information;

\item {} 
\sphinxAtStartPar
time\sphinxhyphen{}related factors;

\item {} 
\sphinxAtStartPar
biases, assumptions, and beliefs of those involved.

\end{itemize}

\sphinxAtStartPar
The organization should identify risks, whether or not their sources are under its control. \\
Consideration should be given that there may be more than one type of outcome, resulting in various tangible or intangible consequences.


\subsubsection{Risk Analysis}
\label{\detokenize{PM/rm:risk-analysis}}
\sphinxAtStartPar
Risk analysis aims to comprehend the nature of risk and its characteristics, including, where appropriate, the level of risk. \\
Risk analysis involves a detailed consideration of uncertainties, risk sources, consequences, likelihood, events, scenarios, controls, and their effectiveness. \\
An event can have multiple causes and consequences and can affect multiple objectives.

\sphinxAtStartPar
Risk analysis can be undertaken with varying degrees of detail and complexity, depending on the purpose of the analysis, the availability and reliability of the information, and the resources available. \\
Analysis techniques can be qualitative, quantitative, or a combination of these, depending on the circumstances and intended use.

\sphinxAtStartPar
Risk analysis should consider factors such as:
\begin{itemize}
\item {} 
\sphinxAtStartPar
the likelihood of events and consequences;

\item {} 
\sphinxAtStartPar
the nature and magnitude of consequences;

\item {} 
\sphinxAtStartPar
complexity and connectivity;

\item {} 
\sphinxAtStartPar
time\sphinxhyphen{}related factors and volatility;

\item {} 
\sphinxAtStartPar
the effectiveness of existing controls;

\item {} 
\sphinxAtStartPar
sensitivity and confidence levels.

\end{itemize}

\sphinxAtStartPar
The risk analysis may be influenced by any divergence of opinions, biases, perceptions of risk, and judgments. \\
Additional influences are the quality of the information used, the assumptions and exclusions made, any limitations of the techniques, and how they are executed. \\
These influences should be considered, documented, and communicated to decision\sphinxhyphen{}makers.

\sphinxAtStartPar
Highly uncertain events can be difficult to quantify. \\
This can be an issue when analyzing events with severe consequences. \\
In such cases, using a combination of techniques generally provides greater insight.

\sphinxAtStartPar
Risk analysis provides input to risk evaluation, decisions on whether risk needs to be treated and how, and the most appropriate risk treatment strategy and methods. \\
The results provide insight into decisions, where choices are being made, and the options involve different types and levels of risk.


\subsubsection{Risk Evaluation}
\label{\detokenize{PM/rm:risk-evaluation}}
\sphinxAtStartPar
The purpose of risk evaluation is to support decisions. \\
Risk evaluation involves comparing the risk analysis results with the established risk criteria to determine where additional action is required. \\
This can lead to a decision to:
\begin{itemize}
\item {} 
\sphinxAtStartPar
do nothing further;

\item {} 
\sphinxAtStartPar
consider risk treatment options;

\item {} 
\sphinxAtStartPar
undertake further analysis to better understand the risk;

\item {} 
\sphinxAtStartPar
maintain existing controls;

\item {} 
\sphinxAtStartPar
reconsider objectives.

\end{itemize}

\sphinxAtStartPar
Decisions should take account of the wider context and the actual and perceived consequences to external and internal stakeholders.

\sphinxAtStartPar
The outcome of risk evaluation should be recorded, communicated, and then validated at appropriate levels of the organization.


\subsection{Risk Treatment}
\label{\detokenize{PM/rm:risk-treatment}}

\subsubsection{General}
\label{\detokenize{PM/rm:id3}}
\sphinxAtStartPar
The purpose of risk treatment is to select and implement options for addressing risk. \\
Risk treatment involves an iterative process of:
\begin{itemize}
\item {} 
\sphinxAtStartPar
formulating and selecting risk treatment options;

\item {} 
\sphinxAtStartPar
planning and implementing risk treatment;

\item {} 
\sphinxAtStartPar
assessing the effectiveness of that treatment;

\item {} 
\sphinxAtStartPar
deciding whether the remaining risk is acceptable;

\item {} 
\sphinxAtStartPar
if not acceptable, take further treatment.

\end{itemize}


\subsubsection{Selection of Risk Treatment Options}
\label{\detokenize{PM/rm:selection-of-risk-treatment-options}}
\sphinxAtStartPar
Selecting the most appropriate risk treatment option(s) involves balancing the potential benefits derived from the achievement of the objectives against costs, effort, or disadvantages of implementation. \\
Risk treatment options are not necessarily mutually exclusive or appropriate in all circumstances. Options for treating risk may involve one or more of the following:
— avoiding the risk by deciding not to start or continue with the activity that gives rise to the risk;
— taking or increasing the risk to pursue an opportunity;
— removing the risk source;
— changing the likelihood;
— changing the consequences;
— sharing the risk (e.g. through contracts, buying insurance);
— retaining the risk by informed decision.

\sphinxAtStartPar
Justification for risk treatment is broader than solely economic considerations and should consider all of the organization’s obligations, voluntary commitments, and stakeholder views. \\
The selection of risk treatment options should be made per the organization’s objectives, risk criteria, and available resources.

\sphinxAtStartPar
When selecting risk treatment options, the organization should consider stakeholders’ values, perceptions, and potential involvement and the most appropriate ways to communicate and consult with them. \\
Though equally effective, some risk treatments can be more acceptable to stakeholders than others.

\sphinxAtStartPar
Even if carefully designed and implemented, risk treatments might not produce the expected outcomes and could produce unintended consequences. \\
Monitoring and review need to be an integral part of the risk treatment implementation to give assurance that the different forms of treatment become and remain effective.

\sphinxAtStartPar
Risk treatment can also introduce new risks that need to be managed. \\
If there are no treatment options available or if treatment options do not sufficiently modify the risk, the risk should be recorded and kept under ongoing review.

\sphinxAtStartPar
Decision makers and other stakeholders should be aware of the nature and extent of the remaining risk after risk treatment. \\
The remaining risk should be documented and subjected to monitoring, review, and, where appropriate, further treatment.


\subsubsection{Preparing and Implementing Risk Treatment Plans}
\label{\detokenize{PM/rm:preparing-and-implementing-risk-treatment-plans}}
\sphinxAtStartPar
The purpose of risk treatment plans is to specify how the chosen treatment options will be implemented so that arrangements are understood by those involved and progress against the plan can be monitored. \\
The treatment plan should identify the order in which risk treatment should be implemented.

\sphinxAtStartPar
Treatment plans should be integrated into the management plans and processes of the organization in consultation with appropriate stakeholders.

\sphinxAtStartPar
The information provided in the treatment plan should include:
\begin{itemize}
\item {} 
\sphinxAtStartPar
the rationale for selection of the treatment options, including the expected benefits to be gained;

\item {} 
\sphinxAtStartPar
those who are accountable and responsible for approving and implementing the plan;

\item {} 
\sphinxAtStartPar
proposed actions;

\item {} 
\sphinxAtStartPar
resources required, including contingencies;

\item {} 
\sphinxAtStartPar
performance measures;

\item {} 
\sphinxAtStartPar
constraints;

\item {} 
\sphinxAtStartPar
required reporting and monitoring;

\item {} 
\sphinxAtStartPar
when actions are expected to be undertaken and completed.

\end{itemize}


\subsection{Monitoring \& Review}
\label{\detokenize{PM/rm:monitoring-review}}
\sphinxAtStartPar
The purpose of monitoring and review is to assure and improve the quality and effectiveness of process design, implementation, and outcomes. \\
Ongoing monitoring and periodic review of the RM process and its outcomes should be a planned part of the RM process, with responsibilities clearly defined.

\sphinxAtStartPar
Monitoring and review should take place in all stages of the process. \\
Monitoring and review include planning, gathering and analyzing information, recording results, and providing feedback.

\sphinxAtStartPar
The results of monitoring and review should be incorporated throughout the organization’s performance management, measurement, and reporting activities.


\subsection{Recording \& Reporting}
\label{\detokenize{PM/rm:recording-reporting}}
\sphinxAtStartPar
The RM process and its outcomes should be documented and reported through appropriate mechanisms. \\
Recording and reporting aim to:
\begin{itemize}
\item {} 
\sphinxAtStartPar
communicate risk management activities and outcomes across the organization;

\item {} 
\sphinxAtStartPar
provide information for decision\sphinxhyphen{}making;

\item {} 
\sphinxAtStartPar
improve risk management activities;

\item {} 
\sphinxAtStartPar
assist interaction with stakeholders, including those with responsibility and accountability for risk management activities.

\end{itemize}

\sphinxAtStartPar
Decisions concerning the creation, retention, and handling of documented information should consider but not be limited to their use, information sensitivity, and the external and internal context.

\sphinxAtStartPar
Reporting is an integral part of the organization’s governance and should enhance the quality of dialogue with stakeholders and support top management and oversight bodies in meeting their responsibilities. \\
Factors to consider for reporting include, but are not limited to:
\begin{itemize}
\item {} 
\sphinxAtStartPar
differing stakeholders and their specific information needs and requirements;

\item {} 
\sphinxAtStartPar
cost, frequency, and timeliness of reporting;

\item {} 
\sphinxAtStartPar
method of reporting;

\item {} 
\sphinxAtStartPar
relevance of the information to organizational objectives and decision\sphinxhyphen{}making.

\end{itemize}

\sphinxstepscope


\chapter{Estimates at Completion}
\label{\detokenize{PM/eac:estimates-at-completion}}\label{\detokenize{PM/eac::doc}}

\section{Overview}
\label{\detokenize{PM/eac:overview}}
\sphinxAtStartPar
The following page describes the approaches that can be adopted to forecast a project’s cost or duration at completion, which we will refer to as cEAC and sEAC, respectively.

\sphinxAtStartPar
Figure 1 provides the flowchart that prescribes the suggested approach by investigating a series of aspects related to the project control.


\begin{savenotes}\sphinxattablestart
\sphinxthistablewithglobalstyle
\centering
\begin{tabulary}{\linewidth}[t]{T}
\sphinxtoprule
\sphinxstyletheadfamily 
\sphinxAtStartPar
\sphinxincludegraphics{{FigEAC-1.drawio}.png}
\\
\sphinxmidrule
\sphinxtableatstartofbodyhook
\sphinxAtStartPar
Figure 1 — EAC approaches flowchart
\\
\sphinxbottomrule
\end{tabulary}
\sphinxtableafterendhook\par
\sphinxattableend\end{savenotes}

\sphinxAtStartPar
The first decision concerns the level at which the estimates are made (\sphinxstyleemphasis{analysis level}). \\
If enough information is available for each control account, work package, or activity level, a detailed analysis can be made. \\
The project cEAC is given by the sum of the individual control account, work package, or activity forecast. \\
Instead, the project sEAC corresponds to the duration of the revised critical path.
Instead, if the project size does not allow, on a practical level, for a detailed one\sphinxhyphen{}by\sphinxhyphen{}one analysis or no information is available, forecasts can only be done at the project level.

\sphinxAtStartPar
In both cases, the next decision involves the availability and quality of \sphinxstyleemphasis{historical data} upon which infer the required information about project costs/durations. \\
When the quality of data is deemed insufficient or no data is available at all, it is not possible to treasure the information that comes from the project’s external environment to improve the reliability of the cost and schedule forecasts. \\
In the opposite case, there can be considerations about the project outcome without taking assumptions that may be questionable or may prove either right or wrong throughout the project execution.

\sphinxAtStartPar
From the project perspective, if no data is available or its quality is insufficient, the suggested approach is the analytical application of the EVM concepts. \\
Instead, if data is available, a statistical approach (e.g., regression analysis) can be adopted.

\sphinxAtStartPar
From the activity perspective, if no data is available, the next decision involves the \sphinxstyleemphasis{level} of reliability one wants to achieve. \\
If assumptions can be made about the past performance reflecting the future one, the analytical approach by EVM can be used to quantify the cost and schedule performance indexes to date and compute the cEAC or sEAC. \\
When such an assumption does not hold, one can make use of Bayesian probability. \\
When historical data is present, one must further investigate its horizon of reliability. \\
For near future activities, information can be inferred from sources external to the project system. \\
For long\sphinxhyphen{}term activities, cost and duration distributions can be fit on the past data and used to simulate the project outcomes.


\section{Project\sphinxhyphen{}level}
\label{\detokenize{PM/eac:project-level}}

\subsection{Analytical Approach}
\label{\detokenize{PM/eac:analytical-approach}}
\sphinxAtStartPar
The project\sphinxhyphen{}level analytical approach consists of the EVM methodology applied to the accrued project \(PV\), \(EV\), and \(AC\).

\sphinxAtStartPar
If cost and schedule overruns are considered to be recoverable,
\begin{equation*}
\begin{split}
cEAC{(t)} = BAC
\end{split}
\end{equation*}
\sphinxAtStartPar
and
\begin{equation*}
\begin{split}
sEAC{(t)} = PD
\end{split}
\end{equation*}
\sphinxAtStartPar
.

\sphinxAtStartPar
Otherwise, the physical formulation for the EAC is
\begin{equation*}
\begin{split}
xEAC{(t)} = \text{Actual} + xETC{(t)}
\end{split}
\end{equation*}
\sphinxAtStartPar
where
\begin{equation*}
\begin{split}
xETC{(t)} = [\text{Planned} - \text{Earned{(t)}}]/xPF{(t)}
\end{split}
\end{equation*}
\sphinxAtStartPar
If cost,
\begin{equation*}
\begin{split}
x = c \rightarrow xPF = cPF \rightarrow cEAC{(t)} = AC{(t)} + [BAC - EV{(t)}]/cPF{(t)}
\end{split}
\end{equation*}
\sphinxAtStartPar
If schedule,
\begin{equation*}
\begin{split}
x = s \rightarrow xPF = sPF \rightarrow sEAC{(t)} = t + [PD - ES{(t)}]/sPF{(t)}
\end{split}
\end{equation*}
\sphinxAtStartPar
\(cPF{(t)}\) and \(sPF{(t)}\) both depend on the assumptions made, as follows.


\begin{savenotes}\sphinxattablestart
\sphinxthistablewithglobalstyle
\centering
\begin{tabulary}{\linewidth}[t]{TT}
\sphinxtoprule
\sphinxstyletheadfamily 
\sphinxAtStartPar
Formula
&\sphinxstyletheadfamily 
\sphinxAtStartPar
Assumption
\\
\sphinxmidrule
\sphinxtableatstartofbodyhook
\sphinxAtStartPar
\(xPF{(t)} = 1\)
&
\sphinxAtStartPar
A one\sphinxhyphen{}time cost/schedule overrun cannot be recovered
\\
\sphinxhline
\sphinxAtStartPar
\(xPF{(t)} = f \left( t, PV{(t)}, EV{(t)}, AC{(t)} \right)\)
&
\sphinxAtStartPar
The future performance will reflect, somehow, the actual performance
\\
\sphinxbottomrule
\end{tabulary}
\sphinxtableafterendhook\par
\sphinxattableend\end{savenotes}

\sphinxAtStartPar
Any combinations of the EVM variables can be used to compute the \(f\).
Examples include, but are not limited to:
\begin{itemize}
\item {} 
\sphinxAtStartPar
punctual values
\begin{itemize}
\item {} 
\sphinxAtStartPar
\(CPI{(t)}\) for \(cPF{(t)}\)

\item {} 
\sphinxAtStartPar
\(SPI^\text{t}{(t)}\) for \(sPF{(t)}\)

\item {} 
\sphinxAtStartPar
\(CPI{(t)} \cdot SPI{(t)}\) for both \(xPF{(t)}\)

\item {} 
\sphinxAtStartPar
\(w_{CPI} \cdot CPI{(t)} + w_{SPI} \cdot SPI{(t)}\) where \(w_{CPI} + w_{SPI} = 1\)

\item {} 
\sphinxAtStartPar
etc.

\end{itemize}

\item {} 
\sphinxAtStartPar
calculated values
\begin{itemize}
\item {} 
\sphinxAtStartPar
weighted/moving average

\item {} 
\sphinxAtStartPar
smoothing

\item {} 
\sphinxAtStartPar
etc.

\end{itemize}

\end{itemize}


\subsection{Statistical Approach}
\label{\detokenize{PM/eac:statistical-approach}}
\sphinxAtStartPar
The project\sphinxhyphen{}level statistical approach proposed is based on regression modeling. \\
A regression model can be represented through the following equation,
\begin{equation*}
\begin{split}
y = f(X) + \varepsilon,
\end{split}
\end{equation*}
\sphinxAtStartPar
where \(y\) is the dependent variable, \(X\) is the matrix of explanatory variables, and \(\varepsilon\) is the random additive error. \\
Regression analysis aims to evaluate the function \(\widehat{f}\) that best approximates \(f\).


\subsubsection{Methods}
\label{\detokenize{PM/eac:methods}}
\sphinxAtStartPar
Two alternative methods to evaluate the EACs are proposed.
\begin{enumerate}
\sphinxsetlistlabels{\arabic}{enumi}{enumii}{}{.}%
\item {} 
\sphinxAtStartPar
The dependent variable is set to the EAC,

\end{enumerate}
\begin{equation*}
\begin{split}
\widehat{y} = \widehat{xEAC}{(t)}
\end{split}
\end{equation*}
\sphinxAtStartPar
It is difficult for regression models to understand how different combinations of the \(X\) variables lead to the same value of \(xEAC\), since it’s unique for each project in the dataset. \\
2. The dependent variable is set to the PF that would provide, at time \(t\), the exact EAC,
\begin{equation*}
\begin{split}
\widehat{y} = \widehat{xPF}{(t)} : xEAC{(t)} = \text{Actual} + [\text{Planned} - \text{Earned}]/\widehat{xPF}{(t)}
\end{split}
\end{equation*}
\sphinxAtStartPar
Setting the problem in such a way allows for evaluating a different \(xPF{(t)}\) for each time \(t\), which is the result of a combination of the project control metrics.


\section{Activity\sphinxhyphen{}level}
\label{\detokenize{PM/eac:activity-level}}

\subsection{Analytical Approach}
\label{\detokenize{PM/eac:id1}}
\sphinxAtStartPar
The activity\sphinxhyphen{}level analytical approach consists of the EVM methodology applied to the accrued activity \(PV\), \(EV\), and \(AC\).

\sphinxAtStartPar
To compute the cEAC, one must sum the individual cEACs, as follows:
\begin{equation*}
\begin{split}
cEAC{(t)} = \sum_{i=1}^{I} cEAC{(t)}_i
\end{split}
\end{equation*}
\sphinxAtStartPar
where
\begin{equation*}
\begin{split}
cEAC{(t)}_i = AC{(t)}_i + [BAC_i - EV{(t)}_i]/cPF{(t)}_i \quad \forall i \in I
\end{split}
\end{equation*}
\sphinxAtStartPar
To compute the sEAC, one must sum the individual sEACs of the activities belonging to the different paths and evaluate the new critical path, as follows:
\begin{equation*}
\begin{split}
sEAC{(t)} = \text{max}_i \begin{Bmatrix} sEAC{(t)}_i : i = 1..I\end{Bmatrix} \end{split}
\end{equation*}
\sphinxAtStartPar
where
\begin{equation*}
\begin{split}
sEAC{(t)}_i = t + [PD_i - ES{(t)}_i]/sPF{(t)}_i \quad \forall i \in I.
\end{split}
\end{equation*}

\subsection{Statistical Approach}
\label{\detokenize{PM/eac:id2}}
\sphinxAtStartPar
The activity\sphinxhyphen{}level statistical approach is based on the assumed distributions of the activities’ cost and duration probability density functions (PDF). \\
The Bayesian inference can be called in case such distributions are hypothesized a priori without historical data to be fitted to. \\
Otherwise, regular Monte Carlo simulations can be performed.

\sphinxAtStartPar
The Monte Carlo simulation applied to the project cost and duration evaluation is performed as follows.


\begin{savenotes}\sphinxattablestart
\sphinxthistablewithglobalstyle
\centering
\begin{tabulary}{\linewidth}[t]{T}
\sphinxtoprule
\sphinxstyletheadfamily 
\sphinxAtStartPar
\sphinxincludegraphics{{FigEAC-2.drawio}.png}
\\
\sphinxmidrule
\sphinxtableatstartofbodyhook
\sphinxAtStartPar
Figure 2 — (Bayesian) simulation approach flowchart
\\
\sphinxbottomrule
\end{tabulary}
\sphinxtableafterendhook\par
\sphinxattableend\end{savenotes}

\sphinxAtStartPar
The cost/duration PDF for each activity must be assumed or inferred from the data. \\
An arbitrary number of simulations is then conducted. \\
Within each simulation, all activities’ cost/duration are randomized. \\
The total project cost is the sum of all the (remaining) activities costs. \\
Instead, the total project duration is the maximum duration among all the different paths of activities revised durations. \\
The simulation stopping criterion is given by the convergence of the standard deviation of the resulting project cost/duration. \\
Plotting the histogram of the resulting project cost/duration allows for the identification of its PDF.


\subsection{Long\sphinxhyphen{}Term Statistical Approach}
\label{\detokenize{PM/eac:long-term-statistical-approach}}
\sphinxAtStartPar
The activity\sphinxhyphen{}level long\sphinxhyphen{}term statistical approach consists of the same (Bayesian) Monte Carlo simulation framework. \\
The only difference is in the PDF of the activities cost/duration, which is not hypothesized a priori but rather inferred from the historical data.


\section{Activity\sphinxhyphen{}level Short\sphinxhyphen{}Term Resource Analysis}
\label{\detokenize{PM/eac:activity-level-short-term-resource-analysis}}
\sphinxAtStartPar
A bottom\sphinxhyphen{}up approach is suggested when the forecast of the project cost/duration involves activities that are currently or will be carried out shortly, rather than projecting their outcome through analytical or statistical means.


\begin{savenotes}\sphinxattablestart
\sphinxthistablewithglobalstyle
\centering
\begin{tabulary}{\linewidth}[t]{T}
\sphinxtoprule
\sphinxstyletheadfamily 
\sphinxAtStartPar
\sphinxincludegraphics{{FigEAC-3.drawio}.png}
\\
\sphinxmidrule
\sphinxtableatstartofbodyhook
\sphinxAtStartPar
Figure 3 — Bottom\sphinxhyphen{}up cost/durations considerations
\\
\sphinxbottomrule
\end{tabulary}
\sphinxtableafterendhook\par
\sphinxattableend\end{savenotes}

\sphinxAtStartPar
In this regard, we recall that the activities for which it is required to compute the revised cost/duration are either labor intensive or material intensive. \\
In the former case, if work resources are internal to the organization, the activity cost will reflect its revised duration; if work resources are external to the organization, the cost depends on the contract scheme for that specific activity. \\
In the latter case, the cost of the material resources may depend on two factors. \\
A market analysis must be conducted if the same amount of resources is needed, but their price is changed. \\
When the initial estimate of the amount of materials required proves inaccurate, the variation in cost/time will reflect the new estimate for the quantity of material.

\sphinxstepscope


\chapter{Project Portfolio Management}
\label{\detokenize{PM/ppm:project-portfolio-management}}\label{\detokenize{PM/ppm::doc}}

\section{Context}
\label{\detokenize{PM/ppm:context}}

\subsection{Organizational Governance}
\label{\detokenize{PM/ppm:organizational-governance}}
\sphinxAtStartPar
\sphinxstylestrong{Organizational governance} directs a permanent or temporary organization by establishing the governance framework. \\
Governing bodies, executives, and senior management govern their organization to achieve accountability and performance.

\sphinxAtStartPar
An organization’s governance
\begin{itemize}
\item {} 
\sphinxAtStartPar
is based on the specific priorities of the organization, and

\item {} 
\sphinxAtStartPar
spans the range of sometimes conflicting stakeholder interests and may be influenced by the wider governance environment.

\end{itemize}

\sphinxAtStartPar
The elements of organizational governance that address PPPs should be:
\begin{itemize}
\item {} 
\sphinxAtStartPar
an integrated part of the permanent or temporary organization’s overall governance framework;

\item {} 
\sphinxAtStartPar
designed to support the organization’s principles, values, and strategic objectives;

\item {} 
\sphinxAtStartPar
designed to optimize the benefits created by investing resources in selected projects, programmes, and portfolios (PPPs).

\end{itemize}

\sphinxAtStartPar
One possible relationship between organizational governance and the governance of PPPs is shown in Figure 1.


\begin{savenotes}\sphinxattablestart
\sphinxthistablewithglobalstyle
\centering
\begin{tabulary}{\linewidth}[t]{T}
\sphinxtoprule
\sphinxstyletheadfamily 
\sphinxAtStartPar
\sphinxincludegraphics{{FigPPM-1.drawio}.png}
\\
\sphinxmidrule
\sphinxtableatstartofbodyhook
\sphinxAtStartPar
Figure 1 — Example of the context of governance of PPPs
\\
\sphinxbottomrule
\end{tabulary}
\sphinxtableafterendhook\par
\sphinxattableend\end{savenotes}

\sphinxAtStartPar
The shaded box represents the governance framework.
Arrows are a generalized representation of the flow of knowledge, documents, deliverables, and other artifacts.
PPP is the acronym in the diagram for PPPs.


\subsection{Governing Bodies}
\label{\detokenize{PM/ppm:governing-bodies}}
\sphinxAtStartPar
Different governing bodies may exist depending on organizational needs and the governing PPPs.

\sphinxAtStartPar
Each governing body may have accountability and responsibility for:
\begin{itemize}
\item {} 
\sphinxAtStartPar
complying with the objectives, values, and principles established by the organization’s overall governing body;

\item {} 
\sphinxAtStartPar
addressing the requirements of stakeholders;

\item {} 
\sphinxAtStartPar
complying with organizational and legal requirements;

\item {} 
\sphinxAtStartPar
developing and maintaining policies, procedures, and processes;

\item {} 
\sphinxAtStartPar
setting objectives for and providing direction to the organizational entities being governed;

\item {} 
\sphinxAtStartPar
delegating responsibilities to, empowering, and supporting the managers:
\begin{itemize}
\item {} 
\sphinxAtStartPar
delegations should balance authority and responsibility for the required actions, * the governing body remains accountable;

\end{itemize}

\item {} 
\sphinxAtStartPar
monitoring conformance to and achievement of the objectives;

\item {} 
\sphinxAtStartPar
providing final decision\sphinxhyphen{}making on escalated critical issues.

\end{itemize}


\subsection{Differences between Governance and Management}
\label{\detokenize{PM/ppm:differences-between-governance-and-management}}
\sphinxAtStartPar
Governance authorizes, directs, empowers, provides oversight, and limits management’s actions.

\sphinxAtStartPar
Management should work within the constraints set by the organization’s governance to achieve the organization’s objectives.

\sphinxAtStartPar
Governance and management functions may be performed at different levels and in different parts of the organization, but the governing body remains accountable for the organization’s performance.

\sphinxAtStartPar
While governance and management are different, everyone involved in governance and management should be responsible for working proactively towards achieving the organization’s objectives.


\section{Governance of Projects, Programmes, and Portfolios}
\label{\detokenize{PM/ppm:governance-of-projects-programmes-and-portfolios}}

\subsection{General}
\label{\detokenize{PM/ppm:general}}
\sphinxAtStartPar
The governance of PPPs should be an integrated part of the organization’s overall governance. \\
The governance framework should integrate across the PPPs within the organization and, where necessary, incorporate the requirements of other participating organizations. \\
The organization’s overall governance should support and properly manage PPPs.

\sphinxAtStartPar
The governance of PPPs should:
\begin{itemize}
\item {} 
\sphinxAtStartPar
reflect the values and principles of the organization or organizations responsible for the projects,

\item {} 
\sphinxAtStartPar
programmes and portfolios being governed;

\item {} 
\sphinxAtStartPar
facilitate achieving the organization’s objectives while complying with the constraints set by its

\item {} 
\sphinxAtStartPar
governance framework;

\item {} 
\sphinxAtStartPar
consider the cultural and ethical norms of:
\begin{itemize}
\item {} 
\sphinxAtStartPar
any other organizations involved;

\item {} 
\sphinxAtStartPar
communities in which the organization operates.

\end{itemize}

\end{itemize}


\subsection{Values}
\label{\detokenize{PM/ppm:values}}
\sphinxAtStartPar
The values expressed through the governance of PPPs should remain consistent to, and aligned with the organization’s values.

\sphinxAtStartPar
Within this document, the concept of values are those values that are adopted or decided by the organization or participating organizations. \\
These values should
\begin{itemize}
\item {} 
\sphinxAtStartPar
determine or influence the standards of behavior of the members of the organization or organizations, and

\item {} 
\sphinxAtStartPar
be generally accepted within the wider community in which the organization operates.

\end{itemize}

\sphinxAtStartPar
The organization’s values
\begin{itemize}
\item {} 
\sphinxAtStartPar
may be documented, and

\item {} 
\sphinxAtStartPar
should reflect what is ethically acceptable and valuable to the organization’s stakeholders.

\end{itemize}

\sphinxAtStartPar
Where conflicting values exist among the stakeholder communities, there should be agreement on managing these conflicts.


\subsection{Principles}
\label{\detokenize{PM/ppm:principles}}
\sphinxAtStartPar
Principles are reflected in the fundamental policies and practices adopted by the organization’s governing body to support its values and achieve its objectives. \\
The governing body should
\begin{itemize}
\item {} 
\sphinxAtStartPar
identify and document key principles for the governance of PPPs that align with the organization’s values, and

\item {} 
\sphinxAtStartPar
identify the objectives of the governance framework.

\end{itemize}


\subsection{Guidelines for the governance of PPPs}
\label{\detokenize{PM/ppm:guidelines-for-the-governance-of-ppps}}

\subsubsection{General}
\label{\detokenize{PM/ppm:id1}}
\sphinxAtStartPar
The guidelines for the governance of PPPs should enable the creation of the governance framework to be adopted by the organization’s governing body and support its values, principles, and the achievement of its objectives. \\
For the purposes of this document, the governing body should be accountable for implementing the governance framework for PPPs. \\
The governing body should consider the principles and guidelines in designing and implementing the governance framework for PPPs.


\subsubsection{Guidelines}
\label{\detokenize{PM/ppm:guidelines}}
\sphinxAtStartPar
The governing body should develop specific guidelines that provide the context within which its
PPPs should be managed by the organization’s values and requirements. \\
The guidelines should include:
\begin{itemize}
\item {} 
\sphinxAtStartPar
alignment of the governance of project, programme, and portfolio management with the

\item {} 
\sphinxAtStartPar
organization’s policies, values, and objectives;

\item {} 
\sphinxAtStartPar
a process for developing new and modified values and policies where gaps exist at the organizational

\item {} 
\sphinxAtStartPar
level or improvements are required;

\item {} 
\sphinxAtStartPar
development, implementation, and maintenance of the governance framework for projects,

\item {} 
\sphinxAtStartPar
programmes and portfolios, which include:
\begin{itemize}
\item {} 
\sphinxAtStartPar
establishing roles, responsibilities, and accountabilities;

\item {} 
\sphinxAtStartPar
defining guidelines for the appointment of human resources;

\end{itemize}

\item {} 
\sphinxAtStartPar
enabling effective communication between governance and management entities;

\item {} 
\sphinxAtStartPar
providing for the separation of the governance function from the management role;

\item {} 
\sphinxAtStartPar
providing oversight to enable conformance with the governance guidelines;

\item {} 
\sphinxAtStartPar
improving the governance framework for PPPs.

\end{itemize}


\subsubsection{Performance of Projects, Programmes, and Portfolios}
\label{\detokenize{PM/ppm:performance-of-projects-programmes-and-portfolios}}
\sphinxAtStartPar
The governance framework should contribute to and provide oversight of the creation and realization
of value for stakeholders by:
\begin{itemize}
\item {} 
\sphinxAtStartPar
the selection of members of the governing body and delegated governance entities that have the appropriate levels of capability, competence, authority, experience, and access to the resources they require;

\item {} 
\sphinxAtStartPar
responsible management of human and other resources and their use.

\end{itemize}


\subsection{Framework}
\label{\detokenize{PM/ppm:framework}}

\subsubsection{General}
\label{\detokenize{PM/ppm:id2}}
\sphinxAtStartPar
The governing body should establish a governance framework for PPPs.
The governance framework should comply with the organizational governance values, principles, and guidelines.

\sphinxAtStartPar
The framework should include the policies, processes, procedures, guidelines, boundaries, interfaces,
roles, responsibilities, and accountabilities needed to implement and maintain the organization’s governance values and principles, as indicated in Figure 2. \\
The framework should be capable of being documented, communicated, and monitored. \\
The governance framework for PPPs, and their interfaces, should be reviewed regularly.


\begin{savenotes}\sphinxattablestart
\sphinxthistablewithglobalstyle
\centering
\begin{tabulary}{\linewidth}[t]{T}
\sphinxtoprule
\sphinxstyletheadfamily 
\sphinxAtStartPar
\sphinxincludegraphics{{FigPPM-2.drawio}.png}
\\
\sphinxmidrule
\sphinxtableatstartofbodyhook
\sphinxAtStartPar
Figure 2 — Example of context of governance framework of PPPs
\\
\sphinxbottomrule
\end{tabulary}
\sphinxtableafterendhook\par
\sphinxattableend\end{savenotes}

\sphinxAtStartPar
The dotted line and shaded box represent aspects of the governance framework applicable to the referenced
guidelines. \\
Arrows are a generalized representation of the flow of knowledge, documents, deliverables, and other
artifacts. \\
PPP is the acronym in the diagram for PPPs.

\sphinxAtStartPar
Figure 2 offers one possible view of the context of the governance of an organization. \\
The major elements are:
\begin{itemize}
\item {} 
\sphinxAtStartPar
the environment in which the organization or organizations function;

\item {} 
\sphinxAtStartPar
the relationship between the guidelines, stakeholders, and the governing body;

\item {} 
\sphinxAtStartPar
the disciplines of project, programme, and portfolio management and the interface with operations or other organizations;

\item {} 
\sphinxAtStartPar
the guidelines for the governance framework;

\item {} 
\sphinxAtStartPar
the governance guidelines for PPPs.

\end{itemize}

\sphinxAtStartPar
The necessary governance functions and responsibilities should be defined and allocated to each unit or entity at a level of complexity appropriate to the organization’s needs.


\subsubsection{Governance Interfaces}
\label{\detokenize{PM/ppm:governance-interfaces}}
\sphinxAtStartPar
The governing body should determine the interfaces among the entities responsible for the governance of projects, programmes and portfolios, and other governance entities. \\
The interfaces may be characterized by the flow of information, resources, or requirements.

\sphinxAtStartPar
As indicated in Figure 2, these flows generally create two primary governance interfaces which may need definition within the organization’s overall governance context:
\begin{itemize}
\item {} 
\sphinxAtStartPar
the interface between the organization’s governance and the governance of PPPs;

\item {} 
\sphinxAtStartPar
the interface between the governance of PPPs and:

\item {} 
\sphinxAtStartPar
the governance of operations;
\begin{itemize}
\item {} 
\sphinxAtStartPar
other areas of the organization;

\item {} 
\sphinxAtStartPar
the management of other organizations.

\end{itemize}

\end{itemize}


\subsubsection{Implementation and Maintenance of the Governance Framework}
\label{\detokenize{PM/ppm:implementation-and-maintenance-of-the-governance-framework}}
\sphinxAtStartPar
The organization should identify and provide or acquire the necessary support, resources, and knowledge for the implementation, improvement, and sustainment of the governance framework for projects, programmes and portfolios.

\sphinxAtStartPar
Factors to consider during the development, implementation, and maintenance of the governance framework for projects, programmes and portfolios may include:
\begin{itemize}
\item {} 
\sphinxAtStartPar
the organization’s existing governance framework and the legal context of stakeholders;

\item {} 
\sphinxAtStartPar
the way management roles and responsibilities and governance roles and responsibilities are defined and allocated;

\item {} 
\sphinxAtStartPar
the preparedness of the people within the organization to understand and support the organization’s principles and values and contribute to the organization’s governance;

\item {} 
\sphinxAtStartPar
the potential need for an independent and autonomous audit or review or decision gates;

\item {} 
\sphinxAtStartPar
the continuous improvement and sustainment of the governance framework should be an integral part of the organizational governance framework.

\end{itemize}

\sphinxAtStartPar
Once the governance framework has been established, the unique requirements for each discipline should be identified and addressed. \\
See Annex A for further information on the governance framework’s implementation, continuous improvement, and sustainment.


\section{Governance of Projects}
\label{\detokenize{PM/ppm:governance-of-projects}}

\section{Governance of Programmes}
\label{\detokenize{PM/ppm:governance-of-programmes}}

\section{Governance of Portfolios}
\label{\detokenize{PM/ppm:governance-of-portfolios}}

\subsection{General}
\label{\detokenize{PM/ppm:id3}}
\sphinxAtStartPar
Governance of portfolios should be supported by processes, procedures, and standards appropriate for governance requirements.

\sphinxAtStartPar
Governance of portfolios should be aligned with organizational governance.

\sphinxAtStartPar
In addition to the guidelines for the governance of PPPs, the following paragraphs describe the authority and responsibilities of the portfolio governing body and the guidelines
and framework for establishing and maintaining governance for each portfolio. \\
These elements should be considered in conjunction with the guidelines for the governance of projects and
programmes, as applicable.


\subsection{Portfolio Governing Body}
\label{\detokenize{PM/ppm:portfolio-governing-body}}
\sphinxAtStartPar
A portfolio governing body (for example an investment committee, a portfolio board consisting of a
body of executive or senior managers) should be established and granted its authority by the governing
body of the organization.
The responsibilities of the portfolio governing body should include, but are not limited to:
\begin{itemize}
\item {} 
\sphinxAtStartPar
aligning the governance of the portfolio with the organization’s governance;

\item {} 
\sphinxAtStartPar
ensuring the portfolio meets its legal obligations in the jurisdictions affecting its work;

\item {} 
\sphinxAtStartPar
establishing and demonstrating support for the objectives and vision of the portfolio in alignment with organizational strategy;

\item {} 
\sphinxAtStartPar
validating the alignment of the governance of projects and programmes with the governance of the portfolio and the organization’s governance;

\item {} 
\sphinxAtStartPar
engaging with and supporting the management of the portfolio in achieving the portfolio’s objectives;

\item {} 
\sphinxAtStartPar
determining and, as appropriate, delegating levels of decision\sphinxhyphen{}making authority and other mandates;

\item {} 
\sphinxAtStartPar
defining roles, responsibilities, authorities, and accountabilities within the portfolio;

\item {} 
\sphinxAtStartPar
providing effective and efficient leadership based upon an ethical foundation;

\item {} 
\sphinxAtStartPar
authorizing and validating the required resources and capabilities to support the effective and efficient project, programme, and portfolio management, as applicable;

\item {} 
\sphinxAtStartPar
providing appropriate and timely access to finances for the portfolio;

\item {} 
\sphinxAtStartPar
verifying that the portfolio justification and objectives are aligned with the changing strategy and needs of the organization;

\item {} 
\sphinxAtStartPar
providing awareness of individual project, programme, and overall portfolio risks;

\item {} 
\sphinxAtStartPar
validating the alignment of the governance of projects and programmes with the governance of the portfolio and the organization’s governance;

\item {} 
\sphinxAtStartPar
ensuring the appropriate use of risk and opportunity management practices on the portfolio;

\item {} 
\sphinxAtStartPar
establishing and validating policies, processes, procedures, and authorities for the governance of portfolios (which could include project and programme selection, prioritization, authorization criteria, categorization, mechanisms for strategic alignment, and benefits realization and optimization).

\end{itemize}


\subsection{Guidelines for the Governance of Portfolio}
\label{\detokenize{PM/ppm:guidelines-for-the-governance-of-portfolio}}

\subsubsection{General}
\label{\detokenize{PM/ppm:id4}}
\sphinxAtStartPar
A portfolio operates in an environment that includes the application of guidelines for the governance of projects, programmes, and portfolios, as identified in 5.4. \\
The application of the guidelines is established in a governance framework and supported by the guidelines for the governance of portfolios. \\
The application of these guidelines is governed by the portfolio governing body.


\subsubsection{Portfolio Management Policy}
\label{\detokenize{PM/ppm:portfolio-management-policy}}
\sphinxAtStartPar
A policy should be developed that identifies the strategic vision, objectives, roles, responsibilities,
authorities, and accountabilities of the portfolio management function. Delegation authority for accountability and responsibility should be stated in the policy.
The portfolio management policy is reviewed and updated under changing circumstances.


\subsubsection{Risk}
\label{\detokenize{PM/ppm:risk}}
\sphinxAtStartPar
The risk thresholds of the portfolio should be established, including consideration of the organization’s and stakeholders’ policies and risk tolerances, and communicated to key stakeholders. \\
Policies
and procedures should be established and communicated to the governing bodies of projects and programmes, as appropriate. \\
The portfolio risk profile should be reviewed and monitored at established intervals.


\subsubsection{Stakeholders}
\label{\detokenize{PM/ppm:stakeholders}}
\sphinxAtStartPar
Guidance for the relationships and engagement with stakeholders should be provided that considers the portfolio stakeholders’ legitimate interests, expectations, and conflicting interests.


\subsubsection{Portfolio Audit or Review}
\label{\detokenize{PM/ppm:portfolio-audit-or-review}}
\sphinxAtStartPar
An internal or external portfolio audit or review process should be established. \\
The audit function may include the evaluation of organizational strategy realization and compliance with organizational governance.


\subsubsection{Sustainability and Statutory Requirements}
\label{\detokenize{PM/ppm:sustainability-and-statutory-requirements}}
\sphinxAtStartPar
Policies and procedures should be established that direct the actions to be taken concerning sustainability and statutory requirements (such as health, safety, security, legal, regulatory, economic, environmental, and social) for the portfolio. \\
The policies and procedures should be formally communicated to the governing bodies of projects and programmes, as appropriate.


\subsubsection{Reporting}
\label{\detokenize{PM/ppm:reporting}}
\sphinxAtStartPar
Portfolio reporting should be established and aligned with the portfolio objectives and organizational governance. The level of transparency and disclosure of portfolio reporting should be defined. \\
The integrity of portfolio reports should be verified and validated. \\
Governing body decisions should be documented.


\subsection{Framework}
\label{\detokenize{PM/ppm:id5}}
\sphinxAtStartPar
The governance framework for portfolios establishes and defines the boundaries, interfaces, roles, responsibilities, and accountabilities restricting and enabling the management of portfolios and may include the reporting structure, portfolio management practices, risk management processes and risk tolerance thresholds, and decision criteria for review. \\
The governance framework should be documented, reviewed, updated, and archived as required and by changing circumstances. \\
Figure 3 illustrates an example of the context of a governance framework highlighting the governance of a portfolio or portfolios.


\begin{savenotes}\sphinxattablestart
\sphinxthistablewithglobalstyle
\centering
\begin{tabulary}{\linewidth}[t]{T}
\sphinxtoprule
\sphinxstyletheadfamily 
\sphinxAtStartPar
\sphinxincludegraphics{{FigPPM-4.drawio}.png}
\\
\sphinxmidrule
\sphinxtableatstartofbodyhook
\sphinxAtStartPar
Figure 3 — Example of context of governance framework of PPPs
\\
\sphinxbottomrule
\end{tabulary}
\sphinxtableafterendhook\par
\sphinxattableend\end{savenotes}

\sphinxstepscope


\part{Agile Project Management}

\sphinxstepscope


\chapter{Agile Domains, Tools, and Techniques}
\label{\detokenize{APM/agile:agile-domains-tools-and-techniques}}\label{\detokenize{APM/agile::doc}}
\sphinxAtStartPar
Notes from Mike Griffith, \sphinxstyleemphasis{PMI\sphinxhyphen{}ACP Exam Prep,} Second Edition.


\section{Introduction}
\label{\detokenize{APM/agile:introduction}}

\subsection{Domains}
\label{\detokenize{APM/agile:domains}}

\begin{savenotes}\sphinxattablestart
\sphinxthistablewithglobalstyle
\centering
\begin{tabulary}{\linewidth}[t]{TTT}
\sphinxtoprule
\sphinxstyletheadfamily 
\sphinxAtStartPar
Domain
&\sphinxstyletheadfamily 
\sphinxAtStartPar
Weight
&\sphinxstyletheadfamily 
\sphinxAtStartPar
Sub\sphinxhyphen{}Domain
\\
\sphinxmidrule
\sphinxtableatstartofbodyhook
\sphinxAtStartPar
Agile Principles and Mindset
&
\sphinxAtStartPar
16\%
&
\sphinxAtStartPar

\\
\sphinxhline
\sphinxAtStartPar
Value\sphinxhyphen{}Driven Delivery
&
\sphinxAtStartPar
20\%
&
\sphinxAtStartPar
Define Positive ValueAvoid Potential DownsidesPrioritizationIncremental Development
\\
\sphinxhline
\sphinxAtStartPar
Stakeholder Engagement
&
\sphinxAtStartPar
17\%
&
\sphinxAtStartPar
Understand Stakeholder NeedsEnsure Stakeholder InvolvementManage Stakeholder Expectations
\\
\sphinxhline
\sphinxAtStartPar
Team Performance
&
\sphinxAtStartPar
16\%
&
\sphinxAtStartPar
Team FormationTeam EmpowermentTeam Collaboration and Commitment
\\
\sphinxhline
\sphinxAtStartPar
Adaptive Planning
&
\sphinxAtStartPar
12\%
&
\sphinxAtStartPar
Levels of PlanningAdaptationAgile Sizing and Estimation
\\
\sphinxhline
\sphinxAtStartPar
Problem Detection and Resolution
&
\sphinxAtStartPar
10\%
&
\sphinxAtStartPar

\\
\sphinxhline
\sphinxAtStartPar
Continuous Improvement (Product, Process, People)
&
\sphinxAtStartPar
9\%
&
\sphinxAtStartPar

\\
\sphinxbottomrule
\end{tabulary}
\sphinxtableafterendhook\par
\sphinxattableend\end{savenotes}


\subsection{Tools and Techniques}
\label{\detokenize{APM/agile:tools-and-techniques}}

\begin{savenotes}
\sphinxatlongtablestart
\sphinxthistablewithglobalstyle
\makeatletter
  \LTleft \@totalleftmargin plus1fill
  \LTright\dimexpr\columnwidth-\@totalleftmargin-\linewidth\relax plus1fill
\makeatother
\begin{longtable}{ll}
\sphinxtoprule
\sphinxstyletheadfamily 
\sphinxAtStartPar
Toolkit
&\sphinxstyletheadfamily 
\sphinxAtStartPar
Tool/Technique
\\
\sphinxmidrule
\endfirsthead

\multicolumn{2}{c}{\sphinxnorowcolor
    \makebox[0pt]{\sphinxtablecontinued{\tablename\ \thetable{} \textendash{} continued from previous page}}%
}\\
\sphinxtoprule
\sphinxstyletheadfamily 
\sphinxAtStartPar
Toolkit
&\sphinxstyletheadfamily 
\sphinxAtStartPar
Tool/Technique
\\
\sphinxmidrule
\endhead

\sphinxbottomrule
\multicolumn{2}{r}{\sphinxnorowcolor
    \makebox[0pt][r]{\sphinxtablecontinued{continues on next page}}%
}\\
\endfoot

\endlastfoot
\sphinxtableatstartofbodyhook

\sphinxAtStartPar
Agile Analysis and Design
&
\sphinxAtStartPar
Product Roadmap
\\
\sphinxhline
\sphinxAtStartPar

&
\sphinxAtStartPar
User Stories/Backlog
\\
\sphinxhline
\sphinxAtStartPar

&
\sphinxAtStartPar
Story Maps
\\
\sphinxhline
\sphinxAtStartPar

&
\sphinxAtStartPar
Progressive Elaboration
\\
\sphinxhline
\sphinxAtStartPar

&
\sphinxAtStartPar
Wireframes
\\
\sphinxhline
\sphinxAtStartPar

&
\sphinxAtStartPar
Chartering
\\
\sphinxhline
\sphinxAtStartPar

&
\sphinxAtStartPar
Personas
\\
\sphinxhline
\sphinxAtStartPar

&
\sphinxAtStartPar
Agile Modeling
\\
\sphinxhline
\sphinxAtStartPar

&
\sphinxAtStartPar
Workshops
\\
\sphinxhline
\sphinxAtStartPar

&
\sphinxAtStartPar
Learning Cycle
\\
\sphinxhline
\sphinxAtStartPar

&
\sphinxAtStartPar
Collaboration Games
\\
\sphinxhline
\sphinxAtStartPar
Agile Estimation
&
\sphinxAtStartPar
Relative sizing/story points/T\sphinxhyphen{}shirt sizing
\\
\sphinxhline
\sphinxAtStartPar

&
\sphinxAtStartPar
Wide ban Delphi/Planning Poker
\\
\sphinxhline
\sphinxAtStartPar

&
\sphinxAtStartPar
Affinity Estimation
\\
\sphinxhline
\sphinxAtStartPar

&
\sphinxAtStartPar
Ideal Time
\\
\sphinxhline
\sphinxAtStartPar
Communications
&
\sphinxAtStartPar
Information Radiator
\\
\sphinxhline
\sphinxAtStartPar

&
\sphinxAtStartPar
Team Space Agile Tooling
\\
\sphinxhline
\sphinxAtStartPar

&
\sphinxAtStartPar
Osmotic Communications for Co\sphinxhyphen{}located and/or Distributed Teams
\\
\sphinxhline
\sphinxAtStartPar

&
\sphinxAtStartPar
Two\sphinxhyphen{}way Communications (Trustwhorty, Conversation Driven)
\\
\sphinxhline
\sphinxAtStartPar

&
\sphinxAtStartPar
Socia\sphinxhyphen{}media based Communication
\\
\sphinxhline
\sphinxAtStartPar

&
\sphinxAtStartPar
Active Listening
\\
\sphinxhline
\sphinxAtStartPar

&
\sphinxAtStartPar
Brainstorming
\\
\sphinxhline
\sphinxAtStartPar

&
\sphinxAtStartPar
Feedback Methods
\\
\sphinxhline
\sphinxAtStartPar
Interpersonal Skills
&
\sphinxAtStartPar
Emotional Intelligence
\\
\sphinxhline
\sphinxAtStartPar

&
\sphinxAtStartPar
Collaboration
\\
\sphinxhline
\sphinxAtStartPar

&
\sphinxAtStartPar
Adaptive Leadership
\\
\sphinxhline
\sphinxAtStartPar

&
\sphinxAtStartPar
Servant Leadership
\\
\sphinxhline
\sphinxAtStartPar

&
\sphinxAtStartPar
Negotiation
\\
\sphinxhline
\sphinxAtStartPar

&
\sphinxAtStartPar
Conflict Resolution
\\
\sphinxhline
\sphinxAtStartPar
Metrics
&
\sphinxAtStartPar
Velocity/Throughput/Productivity
\\
\sphinxhline
\sphinxAtStartPar

&
\sphinxAtStartPar
Cycle Time
\\
\sphinxhline
\sphinxAtStartPar

&
\sphinxAtStartPar
Lead Time
\\
\sphinxhline
\sphinxAtStartPar

&
\sphinxAtStartPar
EVM for Agile Projects
\\
\sphinxhline
\sphinxAtStartPar

&
\sphinxAtStartPar
Defect Rate
\\
\sphinxhline
\sphinxAtStartPar

&
\sphinxAtStartPar
Approved Iterations
\\
\sphinxhline
\sphinxAtStartPar

&
\sphinxAtStartPar
Work in Progress
\\
\sphinxhline
\sphinxAtStartPar
Planning, Monitoring, and Adapting
&
\sphinxAtStartPar
Reviews
\\
\sphinxhline
\sphinxAtStartPar

&
\sphinxAtStartPar
Kanban Board
\\
\sphinxhline
\sphinxAtStartPar

&
\sphinxAtStartPar
Task Board
\\
\sphinxhline
\sphinxAtStartPar

&
\sphinxAtStartPar
Timeboxing
\\
\sphinxhline
\sphinxAtStartPar

&
\sphinxAtStartPar
Iteration and Release Planning
\\
\sphinxhline
\sphinxAtStartPar

&
\sphinxAtStartPar
Variance and Trend Analysis
\\
\sphinxhline
\sphinxAtStartPar

&
\sphinxAtStartPar
WIP Limits
\\
\sphinxhline
\sphinxAtStartPar

&
\sphinxAtStartPar
Daily Stand Ups
\\
\sphinxhline
\sphinxAtStartPar

&
\sphinxAtStartPar
Burn down/up Charts
\\
\sphinxhline
\sphinxAtStartPar

&
\sphinxAtStartPar
Cumulative Flow Diagram
\\
\sphinxhline
\sphinxAtStartPar

&
\sphinxAtStartPar
Backlog Grooming/Refinement
\\
\sphinxhline
\sphinxAtStartPar

&
\sphinxAtStartPar
Product\sphinxhyphen{}feedback Loop
\\
\sphinxhline
\sphinxAtStartPar
Process Improvement
&
\sphinxAtStartPar
Kaizen
\\
\sphinxhline
\sphinxAtStartPar

&
\sphinxAtStartPar
Five WHYs
\\
\sphinxhline
\sphinxAtStartPar

&
\sphinxAtStartPar
Retrospectives, Intraspectives
\\
\sphinxhline
\sphinxAtStartPar

&
\sphinxAtStartPar
Process Tailoring/Hybrid Models
\\
\sphinxhline
\sphinxAtStartPar

&
\sphinxAtStartPar
Value Stream Mapping
\\
\sphinxhline
\sphinxAtStartPar

&
\sphinxAtStartPar
Control Limits
\\
\sphinxhline
\sphinxAtStartPar

&
\sphinxAtStartPar
Pre\sphinxhyphen{}mortem (Rule Setting, Failure Analysis)
\\
\sphinxhline
\sphinxAtStartPar

&
\sphinxAtStartPar
Fishbone Diagram Analysis
\\
\sphinxhline
\sphinxAtStartPar
Product Quality
&
\sphinxAtStartPar
Frequent Verification and Validation
\\
\sphinxhline
\sphinxAtStartPar

&
\sphinxAtStartPar
Definition of Done
\\
\sphinxhline
\sphinxAtStartPar

&
\sphinxAtStartPar
Continuous Integration
\\
\sphinxhline
\sphinxAtStartPar

&
\sphinxAtStartPar
Testing, Including Exporatory and Usability
\\
\sphinxhline
\sphinxAtStartPar
Risk Management
&
\sphinxAtStartPar
Risk Adjusted Backlog
\\
\sphinxhline
\sphinxAtStartPar

&
\sphinxAtStartPar
Risk Burn Down Graphs
\\
\sphinxhline
\sphinxAtStartPar

&
\sphinxAtStartPar
Risk\sphinxhyphen{}based Spike
\\
\sphinxhline
\sphinxAtStartPar

&
\sphinxAtStartPar
Architectural Spike
\\
\sphinxhline
\sphinxAtStartPar
Value\sphinxhyphen{}Based Prioritization
&
\sphinxAtStartPar
ROI/NPV/IRR
\\
\sphinxhline
\sphinxAtStartPar

&
\sphinxAtStartPar
Compliance
\\
\sphinxhline
\sphinxAtStartPar

&
\sphinxAtStartPar
Customer Value Prioritization
\\
\sphinxhline
\sphinxAtStartPar

&
\sphinxAtStartPar
Requirements Reviews
\\
\sphinxhline
\sphinxAtStartPar

&
\sphinxAtStartPar
Minimal Viable Product (MVP)
\\
\sphinxhline
\sphinxAtStartPar

&
\sphinxAtStartPar
Minimal Marketable Feature (MMF)
\\
\sphinxhline
\sphinxAtStartPar

&
\sphinxAtStartPar
Relative Prioritization/Ranking
\\
\sphinxhline
\sphinxAtStartPar

&
\sphinxAtStartPar
MoSCoW
\\
\sphinxhline
\sphinxAtStartPar

&
\sphinxAtStartPar
Kano Analysis
\\
\sphinxbottomrule
\end{longtable}
\sphinxtableafterendhook
\sphinxatlongtableend
\end{savenotes}


\section{Agile Principles and Mindset}
\label{\detokenize{APM/agile:agile-principles-and-mindset}}

\subsection{The Agile Mindset}
\label{\detokenize{APM/agile:the-agile-mindset}}

\subsubsection{Declaration of Interdependence (DOI)}
\label{\detokenize{APM/agile:declaration-of-interdependence-doi}}\begin{itemize}
\item {} 
\sphinxAtStartPar
Written in 2005 by Agile Project Leadership Network

\item {} 
\sphinxAtStartPar
Six Precepts:

\end{itemize}
\begin{enumerate}
\sphinxsetlistlabels{\arabic}{enumi}{enumii}{}{.}%
\item {} 
\sphinxAtStartPar
We increase \sphinxstylestrong{return on investment} by making continuous flow of value our focus.

\item {} 
\sphinxAtStartPar
We \sphinxstylestrong{deliver reliable results} by engaging customers in frequent interactions and shared ownership.

\item {} 
\sphinxAtStartPar
We \sphinxstylestrong{expect uncertainty} and manage for it through iterations, anticipation, and adaptation.

\item {} 
\sphinxAtStartPar
We \sphinxstylestrong{unleash creativity} and innovation by recognizing that individuals are the ultimate source of value, and creating an environment where they can make a difference.

\item {} 
\sphinxAtStartPar
We \sphinxstylestrong{boost performance} through group accountability for results and shared responsibility for team effectiveness.

\item {} 
\sphinxAtStartPar
We \sphinxstylestrong{improve effectiveness and reliability} through situationally specific strategies, processes, and practices.

\end{enumerate}


\subsubsection{The Agile Triangle}
\label{\detokenize{APM/agile:the-agile-triangle}}

\begin{savenotes}\sphinxattablestart
\sphinxthistablewithglobalstyle
\centering
\begin{tabulary}{\linewidth}[t]{TTT}
\sphinxtoprule
\sphinxstyletheadfamily 
\sphinxAtStartPar
Methodology
&\sphinxstyletheadfamily 
\sphinxAtStartPar
Constraint
&\sphinxstyletheadfamily 
\sphinxAtStartPar
Variable
\\
\sphinxmidrule
\sphinxtableatstartofbodyhook
\sphinxAtStartPar
Predictive
&
\sphinxAtStartPar
Scope
&
\sphinxAtStartPar
Time, Cost
\\
\sphinxhline
\sphinxAtStartPar
Agile
&
\sphinxAtStartPar
Time, Cost
&
\sphinxAtStartPar
Scope
\\
\sphinxbottomrule
\end{tabulary}
\sphinxtableafterendhook\par
\sphinxattableend\end{savenotes}


\subsection{The Agile Manifesto}
\label{\detokenize{APM/agile:the-agile-manifesto}}

\subsubsection{Four Values}
\label{\detokenize{APM/agile:four-values}}\begin{quote}

\sphinxAtStartPar
\sphinxstylestrong{Individuals and interactions} over processes and tools

\sphinxAtStartPar
\sphinxstylestrong{Working software} over comprehensive documentation

\sphinxAtStartPar
\sphinxstylestrong{Customer collaboration} over contract negotiation

\sphinxAtStartPar
\sphinxstylestrong{Responding} to change over following a plan
\end{quote}


\subsubsection{Twelve Principles}
\label{\detokenize{APM/agile:twelve-principles}}\begin{quote}

\sphinxAtStartPar
Our highest priority is to satisfy the customer through early and continuous delivery of valuable software.

\sphinxAtStartPar
Welcome changing requirements, even late in development. Agile processes harness change for the customer’s competitive advantage.

\sphinxAtStartPar
Deliver working software frequently, from a couple of weeks to a couple of months, with a preference to the shorter timescale.

\sphinxAtStartPar
Business people and developers must work together daily throughout the project.

\sphinxAtStartPar
Build projects around motivated individuals. Give them the environment and support they need, and trust them to get the job done.

\sphinxAtStartPar
The most efficient and effective method of conveying information to and within a development team is face\sphinxhyphen{}to\sphinxhyphen{}face conversation.

\sphinxAtStartPar
Working software is the primary measure of progress.

\sphinxAtStartPar
Agile processes promote sustainable development. The sponsors, developers, and users should be able to maintain a constant pace indefinitely.

\sphinxAtStartPar
Continuous attention to technical excellence and good design enhances agility.

\sphinxAtStartPar
Simplicity—the art of maximizing the amount of work not done—is essential.

\sphinxAtStartPar
The best architectures, requirements, and designs emerge from self\sphinxhyphen{}organizing teams.

\sphinxAtStartPar
At regular intervals, the team reflects on how to become more effective, then tunes and adjust sits behavior accordingly.
\end{quote}


\subsection{Agile Methodologies}
\label{\detokenize{APM/agile:agile-methodologies}}\begin{itemize}
\item {} 
\sphinxAtStartPar
Scrum

\item {} 
\sphinxAtStartPar
Extreme Programming (XP)

\item {} 
\sphinxAtStartPar
Lean Product Development

\item {} 
\sphinxAtStartPar
Kanban

\item {} 
\sphinxAtStartPar
Feature\sphinxhyphen{}driven Development (FDD)

\item {} 
\sphinxAtStartPar
Dynamic Systems Development Method (DSDM)

\item {} 
\sphinxAtStartPar
Crystal

\end{itemize}


\subsubsection{Scrum}
\label{\detokenize{APM/agile:scrum}}

\paragraph{Process}
\label{\detokenize{APM/agile:process}}

\begin{savenotes}\sphinxattablestart
\sphinxthistablewithglobalstyle
\centering
\begin{tabulary}{\linewidth}[t]{T}
\sphinxtoprule
\sphinxstyletheadfamily 
\sphinxAtStartPar
\sphinxincludegraphics{{agile-scrum}.svg}
\\
\sphinxmidrule
\sphinxtableatstartofbodyhook
\sphinxAtStartPar
Scrum Process
\\
\sphinxbottomrule
\end{tabulary}
\sphinxtableafterendhook\par
\sphinxattableend\end{savenotes}


\paragraph{Principles}
\label{\detokenize{APM/agile:principles}}\begin{itemize}
\item {} 
\sphinxAtStartPar
Transparency

\item {} 
\sphinxAtStartPar
Inspection

\item {} 
\sphinxAtStartPar
Adaptation

\end{itemize}


\paragraph{Values}
\label{\detokenize{APM/agile:values}}\begin{itemize}
\item {} 
\sphinxAtStartPar
Focus

\item {} 
\sphinxAtStartPar
Courage

\item {} 
\sphinxAtStartPar
Openness

\item {} 
\sphinxAtStartPar
Commitment

\item {} 
\sphinxAtStartPar
Respect

\end{itemize}


\paragraph{Sprints}
\label{\detokenize{APM/agile:sprints}}\begin{itemize}
\item {} 
\sphinxAtStartPar
Sprint = timeboxed iteration of < 1 month

\item {} 
\sphinxAtStartPar
No changes affecting the sprint goal are made throughout the sprint

\item {} 
\sphinxAtStartPar
Scope can be clarified/renegotiated as new information becomes available

\item {} 
\sphinxAtStartPar
Can be cancelled by Product Owner before timebox is over due to
\begin{itemize}
\item {} 
\sphinxAtStartPar
goal becomes obsolete

\item {} 
\sphinxAtStartPar
change in business direction/technology conditions

\item {} 
\sphinxAtStartPar
sequence of Activities
\begin{enumerate}
\sphinxsetlistlabels{\arabic}{enumi}{enumii}{}{.}%
\item {} 
\sphinxAtStartPar
Sprint Planning Meeting

\item {} 
\sphinxAtStartPar
Development period
\begin{itemize}
\item {} 
\sphinxAtStartPar
Daily scrums

\item {} 
\sphinxAtStartPar
Sprint review meeting

\item {} 
\sphinxAtStartPar
Sprint retrospective meeting

\end{itemize}

\end{enumerate}

\end{itemize}

\end{itemize}


\paragraph{Team Roles}
\label{\detokenize{APM/agile:team-roles}}\begin{itemize}
\item {} 
\sphinxAtStartPar
Product Owner

\item {} 
\sphinxAtStartPar
Scrum Master

\item {} 
\sphinxAtStartPar
Development Team

\end{itemize}


\paragraph{Activities (Events/Ceremonies)}
\label{\detokenize{APM/agile:activities-events-ceremonies}}\begin{itemize}
\item {} 
\sphinxAtStartPar
Product Backlog Refinement

\item {} 
\sphinxAtStartPar
Sprint Planning Meeting

\item {} 
\sphinxAtStartPar
Daily Scrums

\item {} 
\sphinxAtStartPar
Sprint Reviews

\item {} 
\sphinxAtStartPar
Sprint Retrospectives

\end{itemize}


\paragraph{Artifacts}
\label{\detokenize{APM/agile:artifacts}}\begin{itemize}
\item {} 
\sphinxAtStartPar
Product Increment

\item {} 
\sphinxAtStartPar
Product Backlog

\item {} 
\sphinxAtStartPar
Sprint Backlog

\end{itemize}


\subsubsection{Extreme Programming (XP)}
\label{\detokenize{APM/agile:extreme-programming-xp}}

\paragraph{Core Values}
\label{\detokenize{APM/agile:core-values}}\begin{itemize}
\item {} 
\sphinxAtStartPar
Simplicity

\item {} 
\sphinxAtStartPar
Communication

\item {} 
\sphinxAtStartPar
Feedback

\item {} 
\sphinxAtStartPar
Courage

\item {} 
\sphinxAtStartPar
Respect

\end{itemize}


\paragraph{Team Roles}
\label{\detokenize{APM/agile:id1}}\begin{itemize}
\item {} 
\sphinxAtStartPar
Coach

\item {} 
\sphinxAtStartPar
Customer

\item {} 
\sphinxAtStartPar
Programmers

\item {} 
\sphinxAtStartPar
Testers

\end{itemize}


\paragraph{Practices}
\label{\detokenize{APM/agile:practices}}\begin{itemize}
\item {} 
\sphinxAtStartPar
Whole Team

\item {} 
\sphinxAtStartPar
Planning Games

\item {} 
\sphinxAtStartPar
Small Releases

\item {} 
\sphinxAtStartPar
Customer Tests

\item {} 
\sphinxAtStartPar
Collective Code Ownership

\item {} 
\sphinxAtStartPar
Code Standards

\item {} 
\sphinxAtStartPar
Sustainable Pace

\item {} 
\sphinxAtStartPar
Metaphor

\item {} 
\sphinxAtStartPar
Continuous Integration

\item {} 
\sphinxAtStartPar
Test\sphinxhyphen{}Driven Development

\item {} 
\sphinxAtStartPar
Refactoring

\item {} 
\sphinxAtStartPar
Simple Design

\item {} 
\sphinxAtStartPar
Pair Programming

\end{itemize}


\subsubsection{Lean Product Development}
\label{\detokenize{APM/agile:lean-product-development}}

\paragraph{Core Concepts}
\label{\detokenize{APM/agile:core-concepts}}\begin{itemize}
\item {} 
\sphinxAtStartPar
Eliminate waste

\item {} 
\sphinxAtStartPar
Empower team

\item {} 
\sphinxAtStartPar
Deliver fast

\item {} 
\sphinxAtStartPar
Optimize the whole

\item {} 
\sphinxAtStartPar
Build quality in

\item {} 
\sphinxAtStartPar
Defer decisions

\item {} 
\sphinxAtStartPar
Amplify learning

\end{itemize}


\paragraph{Seven Wastes}
\label{\detokenize{APM/agile:seven-wastes}}\begin{itemize}
\item {} 
\sphinxAtStartPar
Partially done work

\item {} 
\sphinxAtStartPar
Extra processes

\item {} 
\sphinxAtStartPar
Extra features

\item {} 
\sphinxAtStartPar
Task switching

\item {} 
\sphinxAtStartPar
Waiting

\item {} 
\sphinxAtStartPar
Motion

\item {} 
\sphinxAtStartPar
Defects

\end{itemize}


\subsubsection{Kanban}
\label{\detokenize{APM/agile:kanban}}

\begin{savenotes}\sphinxattablestart
\sphinxthistablewithglobalstyle
\centering
\begin{tabulary}{\linewidth}[t]{TTT}
\sphinxtoprule
\sphinxstyletheadfamily 
\sphinxAtStartPar
To Do
&\sphinxstyletheadfamily 
\sphinxAtStartPar
In Progress
&\sphinxstyletheadfamily 
\sphinxAtStartPar
Done
\\
\sphinxmidrule
\sphinxtableatstartofbodyhook
\sphinxAtStartPar
G
&
\sphinxAtStartPar
C
&
\sphinxAtStartPar
A
\\
\sphinxhline
\sphinxAtStartPar
H
&
\sphinxAtStartPar
D
&
\sphinxAtStartPar
B
\\
\sphinxhline
\sphinxAtStartPar
I
&
\sphinxAtStartPar
E
&
\sphinxAtStartPar

\\
\sphinxhline
\sphinxAtStartPar
J
&
\sphinxAtStartPar
F
&
\sphinxAtStartPar

\\
\sphinxbottomrule
\end{tabulary}
\sphinxtableafterendhook\par
\sphinxattableend\end{savenotes}


\paragraph{Principles}
\label{\detokenize{APM/agile:id2}}\begin{itemize}
\item {} 
\sphinxAtStartPar
Visualize workflow

\item {} 
\sphinxAtStartPar
Limit WIP

\item {} 
\sphinxAtStartPar
Manage flow

\item {} 
\sphinxAtStartPar
Make process policies explicit

\item {} 
\sphinxAtStartPar
Improve collaboratively

\end{itemize}


\paragraph{WIP Limits}
\label{\detokenize{APM/agile:wip-limits}}
\sphinxAtStartPar
\(\downarrow\) WIP \(\rightarrow\) \(\uparrow\) Team’s productivity

\sphinxAtStartPar
Little’s Law: \(Queue.Duration = m(Queue.Size)\)


\begin{savenotes}\sphinxattablestart
\sphinxthistablewithglobalstyle
\centering
\begin{tabulary}{\linewidth}[t]{TTTTT}
\sphinxtoprule
\sphinxstyletheadfamily 
\sphinxAtStartPar
Backlog
&\sphinxstyletheadfamily 
\sphinxAtStartPar
Selected (4)
&\sphinxstyletheadfamily 
\sphinxAtStartPar
Develop (3)
&\sphinxstyletheadfamily 
\sphinxAtStartPar
Acceptance (2)
&\sphinxstyletheadfamily 
\sphinxAtStartPar
Deploy
\\
\sphinxmidrule
\sphinxtableatstartofbodyhook
\sphinxAtStartPar
L
&
\sphinxAtStartPar
I
&
\sphinxAtStartPar
F
&
\sphinxAtStartPar
E
&
\sphinxAtStartPar
A
\\
\sphinxhline
\sphinxAtStartPar
M
&
\sphinxAtStartPar
J
&
\sphinxAtStartPar
G
&
\sphinxAtStartPar

&
\sphinxAtStartPar
B
\\
\sphinxhline
\sphinxAtStartPar
N
&
\sphinxAtStartPar
K
&
\sphinxAtStartPar
H
&
\sphinxAtStartPar

&
\sphinxAtStartPar
C
\\
\sphinxhline
\sphinxAtStartPar
O
&
\sphinxAtStartPar

&
\sphinxAtStartPar

&
\sphinxAtStartPar

&
\sphinxAtStartPar
D
\\
\sphinxhline
\sphinxAtStartPar
P
&
\sphinxAtStartPar

&
\sphinxAtStartPar

&
\sphinxAtStartPar

&
\sphinxAtStartPar

\\
\sphinxhline
\sphinxAtStartPar
Q
&
\sphinxAtStartPar

&
\sphinxAtStartPar

&
\sphinxAtStartPar

&
\sphinxAtStartPar

\\
\sphinxbottomrule
\end{tabulary}
\sphinxtableafterendhook\par
\sphinxattableend\end{savenotes}


\subsubsection{Feature\sphinxhyphen{}driven Development (FDD)}
\label{\detokenize{APM/agile:feature-driven-development-fdd}}

\paragraph{Process}
\label{\detokenize{APM/agile:id3}}

\begin{savenotes}\sphinxattablestart
\sphinxthistablewithglobalstyle
\centering
\begin{tabulary}{\linewidth}[t]{T}
\sphinxtoprule
\sphinxstyletheadfamily 
\sphinxAtStartPar
\sphinxincludegraphics{{agile-FDD}.svg}
\\
\sphinxmidrule
\sphinxtableatstartofbodyhook
\sphinxAtStartPar
FDD Process
\\
\sphinxbottomrule
\end{tabulary}
\sphinxtableafterendhook\par
\sphinxattableend\end{savenotes}


\paragraph{Practices}
\label{\detokenize{APM/agile:id4}}\begin{itemize}
\item {} 
\sphinxAtStartPar
Domain object modeling

\item {} 
\sphinxAtStartPar
Developing by feature

\item {} 
\sphinxAtStartPar
Individual class (code) ownership

\item {} 
\sphinxAtStartPar
Feature teams

\item {} 
\sphinxAtStartPar
Inspections

\item {} 
\sphinxAtStartPar
Configuration management

\item {} 
\sphinxAtStartPar
Regular builds

\item {} 
\sphinxAtStartPar
Visibility of progress/results

\end{itemize}


\subsubsection{Dynamic Systems Development Method (DSDM)}
\label{\detokenize{APM/agile:dynamic-systems-development-method-dsdm}}

\paragraph{Process}
\label{\detokenize{APM/agile:id5}}

\begin{savenotes}\sphinxattablestart
\sphinxthistablewithglobalstyle
\centering
\begin{tabulary}{\linewidth}[t]{T}
\sphinxtoprule
\sphinxstyletheadfamily 
\sphinxAtStartPar
\sphinxincludegraphics{{agile-DSDM}.svg}
\\
\sphinxmidrule
\sphinxtableatstartofbodyhook
\sphinxAtStartPar
DSDM Process
\\
\sphinxbottomrule
\end{tabulary}
\sphinxtableafterendhook\par
\sphinxattableend\end{savenotes}


\paragraph{Principles}
\label{\detokenize{APM/agile:id6}}\begin{itemize}
\item {} 
\sphinxAtStartPar
Focus on the business needs

\item {} 
\sphinxAtStartPar
Deliver on time

\item {} 
\sphinxAtStartPar
Collaborate

\item {} 
\sphinxAtStartPar
Never compromise quality

\item {} 
\sphinxAtStartPar
Build incrementally from firm foundations

\item {} 
\sphinxAtStartPar
Develop iteratively

\item {} 
\sphinxAtStartPar
Communicate continuously and clearly

\item {} 
\sphinxAtStartPar
Demonstrate control

\end{itemize}


\subsubsection{Crystal}
\label{\detokenize{APM/agile:crystal}}
\sphinxAtStartPar
Crystal = family of situationally specific, customzied methodologies coded by color names

\sphinxAtStartPar
\(Criticality = f(Defect.Impact)\)


\begin{savenotes}\sphinxattablestart
\sphinxthistablewithglobalstyle
\centering
\begin{tabulary}{\linewidth}[t]{TTTTTT}
\sphinxtoprule
\sphinxstyletheadfamily 
\sphinxAtStartPar
Criticality
&\sphinxstyletheadfamily 
\sphinxAtStartPar
Clear
&\sphinxstyletheadfamily 
\sphinxAtStartPar
Yellow
&\sphinxstyletheadfamily 
\sphinxAtStartPar
Orange
&\sphinxstyletheadfamily 
\sphinxAtStartPar
Red
&\sphinxstyletheadfamily 
\sphinxAtStartPar
Magenta
\\
\sphinxmidrule
\sphinxtableatstartofbodyhook
\sphinxAtStartPar
Life
&
\sphinxAtStartPar
L6
&
\sphinxAtStartPar
L20
&
\sphinxAtStartPar
L40
&
\sphinxAtStartPar
L100
&
\sphinxAtStartPar
\sphinxstylestrong{L200}
\\
\sphinxhline
\sphinxAtStartPar
Essential funds
&
\sphinxAtStartPar
E6
&
\sphinxAtStartPar
\sphinxstylestrong{E20}
&
\sphinxAtStartPar
\sphinxstylestrong{E40}
&
\sphinxAtStartPar
\sphinxstylestrong{E100}
&
\sphinxAtStartPar
\sphinxstylestrong{E200}
\\
\sphinxhline
\sphinxAtStartPar
Discretionary funds
&
\sphinxAtStartPar
\sphinxstylestrong{D6}
&
\sphinxAtStartPar
\sphinxstylestrong{D20}
&
\sphinxAtStartPar
\sphinxstylestrong{D40}
&
\sphinxAtStartPar
\sphinxstylestrong{D100}
&
\sphinxAtStartPar
\sphinxstylestrong{D200}
\\
\sphinxhline
\sphinxAtStartPar
Comfort
&
\sphinxAtStartPar
\sphinxstylestrong{C6}
&
\sphinxAtStartPar
\sphinxstylestrong{C20}
&
\sphinxAtStartPar
\sphinxstylestrong{C40}
&
\sphinxAtStartPar
\sphinxstylestrong{C100}
&
\sphinxAtStartPar
\sphinxstylestrong{C200}
\\
\sphinxhline
\sphinxAtStartPar

&
\sphinxAtStartPar

&
\sphinxAtStartPar

&
\sphinxAtStartPar

&
\sphinxAtStartPar

&
\sphinxAtStartPar

\\
\sphinxhline
\sphinxAtStartPar
Team size
&
\sphinxAtStartPar
1\sphinxhyphen{}6
&
\sphinxAtStartPar
7\sphinxhyphen{}20
&
\sphinxAtStartPar
21\sphinxhyphen{}40
&
\sphinxAtStartPar
41\sphinxhyphen{}100
&
\sphinxAtStartPar
101\sphinxhyphen{}200
\\
\sphinxbottomrule
\end{tabulary}
\sphinxtableafterendhook\par
\sphinxattableend\end{savenotes}


\subsection{Agile Leadership}
\label{\detokenize{APM/agile:agile-leadership}}
\sphinxAtStartPar
Align project objectives with personal objectives to improve productivity


\subsubsection{Management versus Leadership}
\label{\detokenize{APM/agile:management-versus-leadership}}

\begin{savenotes}\sphinxattablestart
\sphinxthistablewithglobalstyle
\centering
\begin{tabulary}{\linewidth}[t]{TT}
\sphinxtoprule
\sphinxstyletheadfamily 
\sphinxAtStartPar
Management Focus
&\sphinxstyletheadfamily 
\sphinxAtStartPar
Leadership
\\
\sphinxmidrule
\sphinxtableatstartofbodyhook
\sphinxAtStartPar
Tasks/things
&
\sphinxAtStartPar
People
\\
\sphinxhline
\sphinxAtStartPar
Control
&
\sphinxAtStartPar
Empowerment
\\
\sphinxhline
\sphinxAtStartPar
Efficiency
&
\sphinxAtStartPar
Effectiveness
\\
\sphinxhline
\sphinxAtStartPar
Doing things right
&
\sphinxAtStartPar
Doing the right things
\\
\sphinxhline
\sphinxAtStartPar
Speed
&
\sphinxAtStartPar
Direction
\\
\sphinxhline
\sphinxAtStartPar
Practices
&
\sphinxAtStartPar
Principles
\\
\sphinxhline
\sphinxAtStartPar
Command
&
\sphinxAtStartPar
Communication
\\
\sphinxbottomrule
\end{tabulary}
\sphinxtableafterendhook\par
\sphinxattableend\end{savenotes}


\subsection{Servant Leadership}
\label{\detokenize{APM/agile:servant-leadership}}

\subsubsection{Duties}
\label{\detokenize{APM/agile:duties}}\begin{itemize}
\item {} 
\sphinxAtStartPar
Shield the team from interruptions

\item {} 
\sphinxAtStartPar
Remove impediments to progress

\item {} 
\sphinxAtStartPar
Communicate

\item {} 
\sphinxAtStartPar
“Carry food and water”

\end{itemize}


\section{Value\sphinxhyphen{}Driven Delivery}
\label{\detokenize{APM/agile:value-driven-delivery}}

\subsection{Assessing Value}
\label{\detokenize{APM/agile:assessing-value}}

\subsubsection{Financial Assessment Metrics}
\label{\detokenize{APM/agile:financial-assessment-metrics}}

\paragraph{Return on Investment (ROI)}
\label{\detokenize{APM/agile:return-on-investment-roi}}
\sphinxAtStartPar
Formula: \(ROI = Investment.Benefits/Investment.Cost\)

\sphinxAtStartPar
Interpretation: \(ROI>1\)


\paragraph{Present Value (PV)}
\label{\detokenize{APM/agile:present-value-pv}}
\sphinxAtStartPar
Formula: \(PV = FV_t/(1+r)^t\)


\paragraph{Net Present Value (NPV)}
\label{\detokenize{APM/agile:net-present-value-npv}}
\sphinxAtStartPar
Formula: \(NPV=\sum_{t=0}^{T} CF_t/(1+r)^t\)

\sphinxAtStartPar
Interpretation:
\begin{itemize}
\item {} 
\sphinxAtStartPar
IF \(NPV>0\) THEN accept ELSE reject

\item {} 
\sphinxAtStartPar
Select project with \sphinxstylestrong{higher} \(NPV\)

\end{itemize}


\begin{savenotes}\sphinxattablestart
\sphinxthistablewithglobalstyle
\centering
\begin{tabulary}{\linewidth}[t]{T}
\sphinxtoprule
\sphinxstyletheadfamily 
\sphinxAtStartPar
\sphinxincludegraphics{{agile-NPV}.svg}
\\
\sphinxmidrule
\sphinxtableatstartofbodyhook
\sphinxAtStartPar
NPV calculation
\\
\sphinxbottomrule
\end{tabulary}
\sphinxtableafterendhook\par
\sphinxattableend\end{savenotes}


\paragraph{Internal Rate of Return (IRR)}
\label{\detokenize{APM/agile:internal-rate-of-return-irr}}
\sphinxAtStartPar
Formula: \(IRR = r : NPV(r) = 0\)

\sphinxAtStartPar
Interpretation:
\begin{itemize}
\item {} 
\sphinxAtStartPar
IF \(IRR>r\) THEN accept ELSE reject

\item {} 
\sphinxAtStartPar
Select project with \sphinxstylestrong{higher} \(IRR\)

\end{itemize}


\begin{savenotes}\sphinxattablestart
\sphinxthistablewithglobalstyle
\centering
\begin{tabulary}{\linewidth}[t]{T}
\sphinxtoprule
\sphinxstyletheadfamily 
\sphinxAtStartPar
\sphinxincludegraphics{{agile-IRR}.svg}
\\
\sphinxmidrule
\sphinxtableatstartofbodyhook
\sphinxAtStartPar
IRR graphical derivation
\\
\sphinxbottomrule
\end{tabulary}
\sphinxtableafterendhook\par
\sphinxattableend\end{savenotes}


\subsubsection{Earned Value Management}
\label{\detokenize{APM/agile:earned-value-management}}

\begin{savenotes}\sphinxattablestart
\sphinxthistablewithglobalstyle
\centering
\begin{tabulary}{\linewidth}[t]{T}
\sphinxtoprule
\sphinxstyletheadfamily 
\sphinxAtStartPar
\sphinxincludegraphics{{agile-EVM}.svg}
\\
\sphinxmidrule
\sphinxtableatstartofbodyhook
\sphinxAtStartPar
EVM calculation
\\
\sphinxbottomrule
\end{tabulary}
\sphinxtableafterendhook\par
\sphinxattableend\end{savenotes}


\begin{savenotes}\sphinxattablestart
\sphinxthistablewithglobalstyle
\centering
\begin{tabulary}{\linewidth}[t]{TTT}
\sphinxtoprule
\sphinxstyletheadfamily 
\sphinxAtStartPar
Symbol
&\sphinxstyletheadfamily 
\sphinxAtStartPar
Formula
&\sphinxstyletheadfamily 
\sphinxAtStartPar
Name
\\
\sphinxmidrule
\sphinxtableatstartofbodyhook
\sphinxAtStartPar
\(AT\)
&
\sphinxAtStartPar

&
\sphinxAtStartPar
Actual Time
\\
\sphinxhline
\sphinxAtStartPar
\(WS\)
&
\sphinxAtStartPar

&
\sphinxAtStartPar
Work Scheduled
\\
\sphinxhline
\sphinxAtStartPar
\(WP\)
&
\sphinxAtStartPar

&
\sphinxAtStartPar
Work Performed
\\
\sphinxhline
\sphinxAtStartPar
\(\mathrm{BAC}\)
&
\sphinxAtStartPar

&
\sphinxAtStartPar
Budget at Completion
\\
\sphinxhline
\sphinxAtStartPar
\(\mathrm{PD}\)
&
\sphinxAtStartPar

&
\sphinxAtStartPar
Planned Duration
\\
\sphinxhline
\sphinxAtStartPar
\(AC\)
&
\sphinxAtStartPar

&
\sphinxAtStartPar
Actual Cost
\\
\sphinxhline
\sphinxAtStartPar
\(PV\)
&
\sphinxAtStartPar
\(\mathrm{BAC} \cdot WS\)
&
\sphinxAtStartPar
Planned Value
\\
\sphinxhline
\sphinxAtStartPar
\(EV\)
&
\sphinxAtStartPar
\(\mathrm{BAC} \cdot WP\)
&
\sphinxAtStartPar
Earned Value
\\
\sphinxhline
\sphinxAtStartPar
\(CV\)
&
\sphinxAtStartPar
\(EV-AC\)
&
\sphinxAtStartPar
Cost Variance
\\
\sphinxhline
\sphinxAtStartPar
\(SV\)
&
\sphinxAtStartPar
\(EV-PV\)
&
\sphinxAtStartPar
Schedule Variance
\\
\sphinxhline
\sphinxAtStartPar
\(CPI\)
&
\sphinxAtStartPar
\(EV/AC\)
&
\sphinxAtStartPar
Cost Performance Index
\\
\sphinxhline
\sphinxAtStartPar
\(SPI\)
&
\sphinxAtStartPar
\(EV/PV\)
&
\sphinxAtStartPar
Schedule Performance Index
\\
\sphinxhline
\sphinxAtStartPar
\(cEAC\)
&
\sphinxAtStartPar
\(\mathrm{BAC}\)\(\mathrm{BAC}-CV\)\(\mathrm{BAC}/CPI\)
&
\sphinxAtStartPar
Cost Estimate at Completion
\\
\sphinxhline
\sphinxAtStartPar
\(tEAC\)
&
\sphinxAtStartPar
\(\mathrm{PD}\)\(\mathrm{PD}/SPI\)
&
\sphinxAtStartPar
Time Estimate at Completion
\\
\sphinxhline
\sphinxAtStartPar
\(cETC\)
&
\sphinxAtStartPar
\(cEAC-AC\)
&
\sphinxAtStartPar
Cost Estimate to Complete
\\
\sphinxhline
\sphinxAtStartPar
\(tETC\)
&
\sphinxAtStartPar
\(tEAC-AT\)
&
\sphinxAtStartPar
Time Estimate to Complete
\\
\sphinxhline
\sphinxAtStartPar
\(cVAC\)
&
\sphinxAtStartPar
\(cEAC-\mathrm{BAC}\)
&
\sphinxAtStartPar
Cost Variance at Completion
\\
\sphinxhline
\sphinxAtStartPar
\(tVAC\)
&
\sphinxAtStartPar
\(tEAC-\mathrm{PD}\)
&
\sphinxAtStartPar
Time Variance at Completion
\\
\sphinxbottomrule
\end{tabulary}
\sphinxtableafterendhook\par
\sphinxattableend\end{savenotes}


\subsubsection{Agile Project Accounting}
\label{\detokenize{APM/agile:agile-project-accounting}}\begin{itemize}
\item {} 
\sphinxAtStartPar
Break down product/service into MVP

\item {} 
\sphinxAtStartPar
Deliver MVP asap

\item {} 
\sphinxAtStartPar
Exploit opportunities for early benefits by using part of the product/service while completing the remainder

\end{itemize}


\subsubsection{Key Performance Indexes}
\label{\detokenize{APM/agile:key-performance-indexes}}\begin{itemize}
\item {} 
\sphinxAtStartPar
Rate of Progress

\item {} 
\sphinxAtStartPar
Remaining Work

\item {} 
\sphinxAtStartPar
Likely Completion Date

\item {} 
\sphinxAtStartPar
Likely Costs Remaining

\end{itemize}


\subsubsection{Regulatory Compliance}
\label{\detokenize{APM/agile:regulatory-compliance}}\begin{itemize}
\item {} 
\sphinxAtStartPar
Regulations \(\rightarrow \) safety

\item {} 
\sphinxAtStartPar
A project that is subejct to regulatory compliance require special documentation to prove that required practices were followed

\item {} 
\sphinxAtStartPar
Approaches for integrating regulatory compliance:
\begin{itemize}
\item {} 
\sphinxAtStartPar
Doing compliance work \sphinxstylestrong{during} product development to keep them linked and relevant

\item {} 
\sphinxAtStartPar
Doing compliance work \sphinxstylestrong{after} product development to avoid rework

\end{itemize}

\end{itemize}


\subsection{Prioritizing Value}
\label{\detokenize{APM/agile:prioritizing-value}}

\subsubsection{Customer\sphinxhyphen{}Valued Prioritization}
\label{\detokenize{APM/agile:customer-valued-prioritization}}
\sphinxAtStartPar
Work on items that maximize value delivered to customer first


\subsubsection{Prioritization Schemes}
\label{\detokenize{APM/agile:prioritization-schemes}}

\paragraph{MoSCoW}
\label{\detokenize{APM/agile:moscow}}

\begin{savenotes}\sphinxattablestart
\sphinxthistablewithglobalstyle
\centering
\begin{tabulary}{\linewidth}[t]{TT}
\sphinxtoprule
\sphinxstyletheadfamily 
\sphinxAtStartPar
Definition
&\sphinxstyletheadfamily 
\sphinxAtStartPar
Priority
\\
\sphinxmidrule
\sphinxtableatstartofbodyhook
\sphinxAtStartPar
\sphinxstylestrong{M}ust have
&
\sphinxAtStartPar
Top
\\
\sphinxhline
\sphinxAtStartPar
\sphinxstylestrong{S}hould have
&
\sphinxAtStartPar
Medium
\\
\sphinxhline
\sphinxAtStartPar
\sphinxstylestrong{C}ould have
&
\sphinxAtStartPar
Low
\\
\sphinxhline
\sphinxAtStartPar
\sphinxstylestrong{W}on’t have this time
&
\sphinxAtStartPar
Null
\\
\sphinxbottomrule
\end{tabulary}
\sphinxtableafterendhook\par
\sphinxattableend\end{savenotes}


\paragraph{Kano Analysis}
\label{\detokenize{APM/agile:kano-analysis}}

\begin{savenotes}\sphinxattablestart
\sphinxthistablewithglobalstyle
\centering
\begin{tabulary}{\linewidth}[t]{T}
\sphinxtoprule
\sphinxstyletheadfamily 
\sphinxAtStartPar
\sphinxincludegraphics{{agile-Kano}.svg}
\\
\sphinxmidrule
\sphinxtableatstartofbodyhook
\sphinxAtStartPar
Kano Analysis
\\
\sphinxbottomrule
\end{tabulary}
\sphinxtableafterendhook\par
\sphinxattableend\end{savenotes}


\subsubsection{Relative Prioritization/Ranking}
\label{\detokenize{APM/agile:relative-prioritization-ranking}}

\begin{savenotes}\sphinxattablestart
\sphinxthistablewithglobalstyle
\centering
\begin{tabulary}{\linewidth}[t]{T}
\sphinxtoprule
\sphinxstyletheadfamily 
\sphinxAtStartPar
\sphinxincludegraphics{{agile-rel_prio_rank}.svg}
\\
\sphinxmidrule
\sphinxtableatstartofbodyhook
\sphinxAtStartPar
Incorporating changes into a relative priority list
\\
\sphinxbottomrule
\end{tabulary}
\sphinxtableafterendhook\par
\sphinxattableend\end{savenotes}


\subsection{Deliver Incrementally}
\label{\detokenize{APM/agile:deliver-incrementally}}
\sphinxAtStartPar
Delivering the “plain\sphinxhyphen{}vanilla” version of a product/service allows realizing benefits to get an early \(ROI\)


\begin{savenotes}\sphinxattablestart
\sphinxthistablewithglobalstyle
\centering
\begin{tabulary}{\linewidth}[t]{T}
\sphinxtoprule
\sphinxstyletheadfamily 
\sphinxAtStartPar
\sphinxincludegraphics{{agile-cost_of_change}.svg}
\\
\sphinxmidrule
\sphinxtableatstartofbodyhook
\sphinxAtStartPar
Cost of Change (Image copyright Scott W. Ambler, www.agilemodeling.com)
\\
\sphinxbottomrule
\end{tabulary}
\sphinxtableafterendhook\par
\sphinxattableend\end{savenotes}


\subsubsection{Minimum Viable Product (MVP)}
\label{\detokenize{APM/agile:minimum-viable-product-mvp}}
\sphinxAtStartPar
MVP = Package of functionality that is complete enough to be useful to the users or the market, yet still small enough that it does not represent the entire project


\subsubsection{Agile Tooling}
\label{\detokenize{APM/agile:agile-tooling}}
\sphinxAtStartPar
Prefer low tech, high touch tools over sophisticated computerized models


\subsubsection{Task/Kanban Boards}
\label{\detokenize{APM/agile:task-kanban-boards}}

\subsubsection{Work in Progress (WIP)}
\label{\detokenize{APM/agile:work-in-progress-wip}}\begin{itemize}
\item {} 
\sphinxAtStartPar
Work started but not started yet

\item {} 
\sphinxAtStartPar
Excessive WIP:
\begin{itemize}
\item {} 
\sphinxAtStartPar
consumes investment capital and delivers no \(ROI\) until converted into product/service

\item {} 
\sphinxAtStartPar
hides bottlenecks/inefficiencies

\item {} 
\sphinxAtStartPar
increases probability of rework

\end{itemize}

\end{itemize}


\subsubsection{WIP Limits}
\label{\detokenize{APM/agile:id7}}
\sphinxAtStartPar
Set limit to WIP (to Task/Kanban Board) \(\rightarrow\) Optimize \textasciitilde{}\textasciitilde{}resource utilization\textasciitilde{}\textasciitilde{} throughput


\begin{savenotes}\sphinxattablestart
\sphinxthistablewithglobalstyle
\centering
\begin{tabulary}{\linewidth}[t]{T}
\sphinxtoprule
\sphinxstyletheadfamily 
\sphinxAtStartPar
\sphinxincludegraphics{{agile-kanban_no_limit}.svg}
\\
\sphinxmidrule
\sphinxtableatstartofbodyhook
\sphinxAtStartPar
Kanban board without limits
\\
\sphinxbottomrule
\end{tabulary}
\sphinxtableafterendhook\par
\sphinxattableend\end{savenotes}


\begin{savenotes}\sphinxattablestart
\sphinxthistablewithglobalstyle
\centering
\begin{tabulary}{\linewidth}[t]{T}
\sphinxtoprule
\sphinxstyletheadfamily 
\sphinxAtStartPar
\sphinxincludegraphics{{agile-kanban_yes_limit}.svg}
\\
\sphinxmidrule
\sphinxtableatstartofbodyhook
\sphinxAtStartPar
Kanban board with limit too strict
\\
\sphinxbottomrule
\end{tabulary}
\sphinxtableafterendhook\par
\sphinxattableend\end{savenotes}


\begin{savenotes}\sphinxattablestart
\sphinxthistablewithglobalstyle
\centering
\begin{tabulary}{\linewidth}[t]{T}
\sphinxtoprule
\sphinxstyletheadfamily 
\sphinxAtStartPar
\sphinxincludegraphics{{agile-kanban_yes_limit_right}.svg}
\\
\sphinxmidrule
\sphinxtableatstartofbodyhook
\sphinxAtStartPar
Kanban board with right limit
\\
\sphinxbottomrule
\end{tabulary}
\sphinxtableafterendhook\par
\sphinxattableend\end{savenotes}


\subsubsection{Cumulative Flow Diagram (CFDs)}
\label{\detokenize{APM/agile:cumulative-flow-diagram-cfds}}
\sphinxAtStartPar
Used for tracking and forecasting delivery of value


\begin{savenotes}\sphinxattablestart
\sphinxthistablewithglobalstyle
\centering
\begin{tabulary}{\linewidth}[t]{T}
\sphinxtoprule
\sphinxstyletheadfamily 
\sphinxAtStartPar
\sphinxincludegraphics{{agile-CFD1}.svg}
\\
\sphinxmidrule
\sphinxtableatstartofbodyhook
\sphinxAtStartPar
CFD
\\
\sphinxbottomrule
\end{tabulary}
\sphinxtableafterendhook\par
\sphinxattableend\end{savenotes}


\begin{savenotes}\sphinxattablestart
\sphinxthistablewithglobalstyle
\centering
\begin{tabulary}{\linewidth}[t]{T}
\sphinxtoprule
\sphinxstyletheadfamily 
\sphinxAtStartPar
\sphinxincludegraphics{{agile-CFD2}.svg}
\\
\sphinxmidrule
\sphinxtableatstartofbodyhook
\sphinxAtStartPar
Detailed CFD
\\
\sphinxbottomrule
\end{tabulary}
\sphinxtableafterendhook\par
\sphinxattableend\end{savenotes}


\paragraph{Little’s Law}
\label{\detokenize{APM/agile:little-s-law}}
\sphinxAtStartPar
\(Queue = In Progress - Done\)


\subsubsection{Bottleneck and Theory of Constraints (TOC)}
\label{\detokenize{APM/agile:bottleneck-and-theory-of-constraints-toc}}
\sphinxAtStartPar
Figure


\subsection{(To Do) Agile Contracting}
\label{\detokenize{APM/agile:to-do-agile-contracting}}

\subsection{Verifying and Validating Value}
\label{\detokenize{APM/agile:verifying-and-validating-value}}

\subsubsection{(To Do) Frequent Verification and Validation}
\label{\detokenize{APM/agile:to-do-frequent-verification-and-validation}}

\subsubsection{Testing and Verification in Software Development}
\label{\detokenize{APM/agile:testing-and-verification-in-software-development}}

\paragraph{Continuous Integration (CI)}
\label{\detokenize{APM/agile:continuous-integration-ci}}

\subparagraph{Objective}
\label{\detokenize{APM/agile:objective}}\begin{itemize}
\item {} 
\sphinxAtStartPar
Incorporate new and changed code into project code repository

\item {} 
\sphinxAtStartPar
Find and resolve problems asap

\item {} 
\sphinxAtStartPar
Ensure system still performs as intended after the new code is integrated

\end{itemize}


\subparagraph{Process}
\label{\detokenize{APM/agile:id8}}

\begin{savenotes}\sphinxattablestart
\sphinxthistablewithglobalstyle
\centering
\begin{tabulary}{\linewidth}[t]{T}
\sphinxtoprule
\sphinxstyletheadfamily 
\sphinxAtStartPar
\sphinxincludegraphics{{agile-continuous_integration}.svg}
\\
\sphinxmidrule
\sphinxtableatstartofbodyhook
\sphinxAtStartPar
Continuous integration
\\
\sphinxbottomrule
\end{tabulary}
\sphinxtableafterendhook\par
\sphinxattableend\end{savenotes}


\subparagraph{Components}
\label{\detokenize{APM/agile:components}}\begin{itemize}
\item {} 
\sphinxAtStartPar
Source code control system

\item {} 
\sphinxAtStartPar
Build tools

\item {} 
\sphinxAtStartPar
Test tools

\item {} 
\sphinxAtStartPar
Scheduler/trigger

\item {} 
\sphinxAtStartPar
Notifications

\end{itemize}


\subparagraph{Pros and Cons}
\label{\detokenize{APM/agile:pros-and-cons}}
\sphinxAtStartPar
Pros
\begin{itemize}
\item {} 
\sphinxAtStartPar
Early warning of wrong code

\item {} 
\sphinxAtStartPar
Problems fixed as they occur \(\rightarrow\) \(\downarrow\) cost of change

\item {} 
\sphinxAtStartPar
Immediate feedback

\item {} 
\sphinxAtStartPar
Frequent unit testing

\item {} 
\sphinxAtStartPar
Can be reverted to last stable version

\end{itemize}

\sphinxAtStartPar
Cons
\begin{itemize}
\item {} 
\sphinxAtStartPar
Require setup time

\item {} 
\sphinxAtStartPar
Cost of procuring machines

\item {} 
\sphinxAtStartPar
Require time to automate system

\end{itemize}


\subsubsection{Test\sphinxhyphen{}Driven Development (TDD)/Test\sphinxhyphen{}First Development (TFD)}
\label{\detokenize{APM/agile:test-driven-development-tdd-test-first-development-tfd}}
\sphinxAtStartPar
Philosopy = tests should be written before code

\sphinxAtStartPar
First tests will fail as no code has been written

\sphinxAtStartPar
Start coding \(\rightarrow\) run tests until the code passes all tests

\sphinxAtStartPar
Refactoring = clean up design to make it easier to understand and maintain without changing the code’s behavior

\sphinxAtStartPar
Red, Green, Refactor/Clean = Writing a test that initially fails, adding code until it passes, and refactoring the code


\begin{savenotes}\sphinxattablestart
\sphinxthistablewithglobalstyle
\centering
\begin{tabulary}{\linewidth}[t]{T}
\sphinxtoprule
\sphinxstyletheadfamily 
\sphinxAtStartPar
\sphinxincludegraphics{{agile-TDD}.svg}
\\
\sphinxmidrule
\sphinxtableatstartofbodyhook
\sphinxAtStartPar
TDD process
\\
\sphinxbottomrule
\end{tabulary}
\sphinxtableafterendhook\par
\sphinxattableend\end{savenotes}


\subsubsection{Acceptance Test\sphinxhyphen{}Driven Development (ATDD)}
\label{\detokenize{APM/agile:acceptance-test-driven-development-atdd}}
\sphinxAtStartPar
Testing focus on \textasciitilde{}\textasciitilde{}code\textasciitilde{}\textasciitilde{} business

\sphinxAtStartPar
Acceptance tests captured in functional test framework (FIT/FitNesse = Framework Integrated Testing)


\paragraph{Process}
\label{\detokenize{APM/agile:id9}}

\begin{savenotes}\sphinxattablestart
\sphinxthistablewithglobalstyle
\centering
\begin{tabulary}{\linewidth}[t]{T}
\sphinxtoprule
\sphinxstyletheadfamily 
\sphinxAtStartPar
\sphinxincludegraphics{{agile-ATDD}.svg}
\\
\sphinxmidrule
\sphinxtableatstartofbodyhook
\sphinxAtStartPar
ATDD process
\\
\sphinxbottomrule
\end{tabulary}
\sphinxtableafterendhook\par
\sphinxattableend\end{savenotes}
\begin{enumerate}
\sphinxsetlistlabels{\arabic}{enumi}{enumii}{}{.}%
\item {} 
\sphinxAtStartPar
\sphinxstylestrong{Discuss} the requirements = gather acceptance criteria

\item {} 
\sphinxAtStartPar
\sphinxstylestrong{Distill} tests in a framework\sphinxhyphen{}friendly format = structure tests in a table format

\item {} 
\sphinxAtStartPar
\sphinxstylestrong{Develop} the code and hookup the tests = hook tests to the code and run acceptance tests

\item {} 
\sphinxAtStartPar
\sphinxstylestrong{Demo} = exploratory testings

\end{enumerate}


\section{Stakeholder Engagement}
\label{\detokenize{APM/agile:stakeholder-engagement}}

\subsection{Taking Care of Stakeholders}
\label{\detokenize{APM/agile:taking-care-of-stakeholders}}

\subsubsection{Keeping Stakeholders Engaged}
\label{\detokenize{APM/agile:keeping-stakeholders-engaged}}
\sphinxAtStartPar
Benefits
\begin{itemize}
\item {} 
\sphinxAtStartPar
Short iterations keep stakeholders interested in the process

\item {} 
\sphinxAtStartPar
Hear about change requests as soon as possible

\item {} 
\sphinxAtStartPar
Identify potential risks, defects, and issues

\item {} 
\sphinxAtStartPar
Use emotional intelligence and interpersonal skills to understand stakeholders’ concerns and find a positive way to engage them with the project

\item {} 
\sphinxAtStartPar
Establish a process for escalating issues that need a high level of authority to resolve

\end{itemize}


\subsubsection{Incorporating Stakeholder Values}
\label{\detokenize{APM/agile:incorporating-stakeholder-values}}
\sphinxAtStartPar
Agile methods focus on bringing project priorities into alignment with stakeholder priorities by
\begin{itemize}
\item {} 
\sphinxAtStartPar
engaging the PO in the prioritization of the backlog, and

\item {} 
\sphinxAtStartPar
inviting stakeholders to planning meetings and retrospectives

\end{itemize}


\subsubsection{Incorporating Community Values}
\label{\detokenize{APM/agile:incorporating-community-values}}
\sphinxAtStartPar
Values shared by Scrum and XP:
\begin{itemize}
\item {} 
\sphinxAtStartPar
\sphinxstylestrong{Respect} = seek consensus
\begin{itemize}
\item {} 
\sphinxAtStartPar
Don’t judge suggestions

\item {} 
\sphinxAtStartPar
No idea is stupid

\end{itemize}

\item {} 
\sphinxAtStartPar
\sphinxstylestrong{Courage}
\begin{itemize}
\item {} 
\sphinxAtStartPar
Perform early evaluations

\item {} 
\sphinxAtStartPar
Focus on transparency by showing
\begin{itemize}
\item {} 
\sphinxAtStartPar
Velocity data

\item {} 
\sphinxAtStartPar
Defect rates

\end{itemize}

\end{itemize}

\end{itemize}


\subsection{Establishing a Shared Vision}
\label{\detokenize{APM/agile:establishing-a-shared-vision}}

\subsubsection{Agile Chartering}
\label{\detokenize{APM/agile:agile-chartering}}
\sphinxAtStartPar
Project charter content:
\begin{itemize}
\item {} 
\sphinxAtStartPar
Goal

\item {} 
\sphinxAtStartPar
Purpose

\item {} 
\sphinxAtStartPar
Composition

\item {} 
\sphinxAtStartPar
Approach

\item {} 
\sphinxAtStartPar
Authorization from sponsor to proceed

\end{itemize}


\paragraph{Agile versus Non\sphinxhyphen{}Agile Charters}
\label{\detokenize{APM/agile:agile-versus-non-agile-charters}}
\sphinxAtStartPar
Shared goal =
\begin{itemize}
\item {} 
\sphinxAtStartPar
gain agreement about Who, What, Where, When, Why, How

\item {} 
\sphinxAtStartPar
obtain authority to proceed

\end{itemize}

\sphinxAtStartPar
Differences:
\begin{itemize}
\item {} 
\sphinxAtStartPar
Agile.details < Non\sphinxhyphen{}agile.details

\item {} 
\sphinxAtStartPar
Agile.focus @ How

\end{itemize}


\paragraph{Developing an Agile Charter}
\label{\detokenize{APM/agile:developing-an-agile-charter}}
\sphinxAtStartPar
W5H
\begin{itemize}
\item {} 
\sphinxAtStartPar
\sphinxstylestrong{Who}
\begin{itemize}
\item {} 
\sphinxAtStartPar
Participants

\item {} 
\sphinxAtStartPar
Stakeholders

\end{itemize}

\item {} 
\sphinxAtStartPar
\sphinxstylestrong{What}
\begin{itemize}
\item {} 
\sphinxAtStartPar
Vision

\item {} 
\sphinxAtStartPar
Mission

\item {} 
\sphinxAtStartPar
Goals

\item {} 
\sphinxAtStartPar
Objectives

\end{itemize}

\item {} 
\sphinxAtStartPar
\sphinxstylestrong{Where}
\begin{itemize}
\item {} 
\sphinxAtStartPar
Work sites

\item {} 
\sphinxAtStartPar
Deployment requirements

\item {} 
\sphinxAtStartPar
…

\end{itemize}

\item {} 
\sphinxAtStartPar
\sphinxstylestrong{When}
\begin{itemize}
\item {} 
\sphinxAtStartPar
Start

\item {} 
\sphinxAtStartPar
End

\end{itemize}

\item {} 
\sphinxAtStartPar
\sphinxstylestrong{Why} = Business rationale

\item {} 
\sphinxAtStartPar
\sphinxstylestrong{How} = Approach

\end{itemize}

\sphinxAtStartPar
Project elevator statement


\begin{savenotes}\sphinxattablestart
\sphinxthistablewithglobalstyle
\centering
\begin{tabulary}{\linewidth}[t]{TT}
\sphinxtoprule
\sphinxstyletheadfamily 
\sphinxAtStartPar
Statement
&\sphinxstyletheadfamily 
\sphinxAtStartPar
Description
\\
\sphinxmidrule
\sphinxtableatstartofbodyhook
\sphinxAtStartPar
For
&
\sphinxAtStartPar
Target customers
\\
\sphinxhline
\sphinxAtStartPar
Who
&
\sphinxAtStartPar
Need
\\
\sphinxhline
\sphinxAtStartPar
The
&
\sphinxAtStartPar
Product/service name
\\
\sphinxhline
\sphinxAtStartPar
Is a
&
\sphinxAtStartPar
Product category
\\
\sphinxhline
\sphinxAtStartPar
That
&
\sphinxAtStartPar
Key benefits/reason to buy
\\
\sphinxhline
\sphinxAtStartPar
Unlike
&
\sphinxAtStartPar
Primary competitive alternative(s)
\\
\sphinxbottomrule
\end{tabulary}
\sphinxtableafterendhook\par
\sphinxattableend\end{savenotes}


\subsubsection{Definition of “Done”}
\label{\detokenize{APM/agile:definition-of-done}}\begin{itemize}
\item {} 
\sphinxAtStartPar
Necessary at all levels (i.e., Deliverables, Releases, and User Stories)

\item {} 
\sphinxAtStartPar
Consist of multiple \sphinxstylestrong{acceptance criteria}

\end{itemize}


\subsubsection{Agile Modeling}
\label{\detokenize{APM/agile:agile-modeling}}
\sphinxAtStartPar
Rationale: value of agile models is in \textasciitilde{}\textasciitilde{}preservation\textasciitilde{}\textasciitilde{} creation

\sphinxAtStartPar
Aim:
\begin{itemize}
\item {} 
\sphinxAtStartPar
Reflective purposes

\item {} 
\sphinxAtStartPar
Investigate problems and find solutions

\end{itemize}


\begin{savenotes}\sphinxattablestart
\sphinxthistablewithglobalstyle
\centering
\begin{tabulary}{\linewidth}[t]{T}
\sphinxtoprule
\sphinxstyletheadfamily 
\sphinxAtStartPar
\sphinxincludegraphics{{agile-modeling_value}.svg}
\\
\sphinxmidrule
\sphinxtableatstartofbodyhook
\sphinxAtStartPar
Modeling Value against Time
\\
\sphinxbottomrule
\end{tabulary}
\sphinxtableafterendhook\par
\sphinxattableend\end{savenotes}

\sphinxAtStartPar
Types:
\begin{itemize}
\item {} 
\sphinxAtStartPar
Use case diagrams

\item {} 
\sphinxAtStartPar
Data models

\item {} 
\sphinxAtStartPar
Screen designs

\end{itemize}


\subsubsection{Wireframes}
\label{\detokenize{APM/agile:wireframes}}
\sphinxAtStartPar
Wireframes = quick and cheap mock\sphinxhyphen{}up of a product/service


\subsubsection{Personas}
\label{\detokenize{APM/agile:personas}}
\sphinxAtStartPar
Personas = quick guides/reminders of key stakeholders and their interests

\sphinxAtStartPar
Augment requirements:
\begin{itemize}
\item {} 
\sphinxAtStartPar
Help prioritize work

\item {} 
\sphinxAtStartPar
Stay focused on users

\item {} 
\sphinxAtStartPar
Gain insights into who users will be

\end{itemize}

\sphinxAtStartPar
Help empathize with final users of product/service

\sphinxAtStartPar
Keep focus on delivering features that users will find valuable


\subsection{Communicating with Stakeholders}
\label{\detokenize{APM/agile:communicating-with-stakeholders}}

\subsubsection{Face\sphinxhyphen{}to\sphinxhyphen{}Face (F2F) Communication}
\label{\detokenize{APM/agile:face-to-face-f2f-communication}}
\sphinxAtStartPar
Highest efficiency: highest interactivity \& highest bandwith/information density


\subsubsection{Two\sphinxhyphen{}Way Communication}
\label{\detokenize{APM/agile:two-way-communication}}
\sphinxAtStartPar
Information flow between stakeholders = \sphinxstylestrong{bidirectional}


\begin{savenotes}\sphinxattablestart
\sphinxthistablewithglobalstyle
\centering
\begin{tabulary}{\linewidth}[t]{T}
\sphinxtoprule
\sphinxstyletheadfamily 
\sphinxAtStartPar
\sphinxincludegraphics{{agile-2-way_comm}.svg}
\\
\sphinxmidrule
\sphinxtableatstartofbodyhook
\sphinxAtStartPar
Dispatching vs Collaborative Communication Model
\\
\sphinxbottomrule
\end{tabulary}
\sphinxtableafterendhook\par
\sphinxattableend\end{savenotes}


\subsubsection{Knowledge Sharing}
\label{\detokenize{APM/agile:knowledge-sharing}}
\sphinxAtStartPar
Agile projects are encouraged to take an abundance\sphinxhyphen{}based—rather than scarcity\sphinxhyphen{}based—attitude toward sharing knowledge

\sphinxAtStartPar
Benefits:
\begin{itemize}
\item {} 
\sphinxAtStartPar
\(\uparrow\) \#people who know about something, \(\uparrow\) \#people there will be who can help you when you get stuck

\item {} 
\sphinxAtStartPar
Helps balance the workload between team members

\end{itemize}


\subsubsection{Information Radiators}
\label{\detokenize{APM/agile:information-radiators}}
\sphinxAtStartPar
Information radiators = highly visible displays of information, including:
\begin{itemize}
\item {} 
\sphinxAtStartPar
large charts

\item {} 
\sphinxAtStartPar
graphs

\item {} 
\sphinxAtStartPar
summaries

\end{itemize}

\sphinxAtStartPar
Data:
\begin{itemize}
\item {} 
\sphinxAtStartPar
Features delivered vs features remaining

\item {} 
\sphinxAtStartPar
Who is working on what

\item {} 
\sphinxAtStartPar
Features selected for the current iteration

\item {} 
\sphinxAtStartPar
Velocity and defect metrics

\item {} 
\sphinxAtStartPar
Retrospective findings

\item {} 
\sphinxAtStartPar
List of threats and issues

\item {} 
\sphinxAtStartPar
Story maps

\item {} 
\sphinxAtStartPar
Burn charts

\end{itemize}


\subsubsection{Social Media}
\label{\detokenize{APM/agile:social-media}}

\subsection{Working Collaboratively}
\label{\detokenize{APM/agile:working-collaboratively}}
\sphinxAtStartPar
Benefits:
\begin{itemize}
\item {} 
\sphinxAtStartPar
Generates wiser decisions

\item {} 
\sphinxAtStartPar
Problem solving

\item {} 
\sphinxAtStartPar
Fosters action

\item {} 
\sphinxAtStartPar
Build social capital

\item {} 
\sphinxAtStartPar
Fosters ownership of collective problems

\end{itemize}


\subsubsection{Workshops}
\label{\detokenize{APM/agile:workshops}}
\sphinxAtStartPar
Tips:
\begin{itemize}
\item {} 
\sphinxAtStartPar
Diverse groups reflect a wider range of viewpoints than just a few experts

\item {} 
\sphinxAtStartPar
Use techniques to prevent dominant individuals and extroverts from monopolizing the discussion

\item {} 
\sphinxAtStartPar
Start with an activity that gets everyone participating within the first five minutes

\end{itemize}


\subsubsection{Brainstorming}
\label{\detokenize{APM/agile:brainstorming}}

\paragraph{Methods}
\label{\detokenize{APM/agile:methods}}

\begin{savenotes}\sphinxattablestart
\sphinxthistablewithglobalstyle
\centering
\begin{tabulary}{\linewidth}[t]{TT}
\sphinxtoprule
\sphinxstyletheadfamily 
\sphinxAtStartPar
Method
&\sphinxstyletheadfamily 
\sphinxAtStartPar
Description
\\
\sphinxmidrule
\sphinxtableatstartofbodyhook
\sphinxAtStartPar
Quiet Writing
&
\sphinxAtStartPar
5\sphinxhyphen{}7 minutes to generate list of ideas
\\
\sphinxhline
\sphinxAtStartPar
Round\sphinxhyphen{}Robin
&
\sphinxAtStartPar
Take turns to suggest ideas
\\
\sphinxhline
\sphinxAtStartPar
Free\sphinxhyphen{}for\sphinxhyphen{}All
&
\sphinxAtStartPar
Shout ideas
\\
\sphinxbottomrule
\end{tabulary}
\sphinxtableafterendhook\par
\sphinxattableend\end{savenotes}


\subsubsection{Collaboration Games}
\label{\detokenize{APM/agile:collaboration-games}}

\begin{savenotes}\sphinxattablestart
\sphinxthistablewithglobalstyle
\centering
\begin{tabulary}{\linewidth}[t]{TT}
\sphinxtoprule
\sphinxstyletheadfamily 
\sphinxAtStartPar
Game
&\sphinxstyletheadfamily 
\sphinxAtStartPar
Summary
\\
\sphinxmidrule
\sphinxtableatstartofbodyhook
\sphinxAtStartPar
Remember the Future
&
\sphinxAtStartPar
Vision\sphinxhyphen{}setting and requirements\sphinxhyphen{}elicitation exercise
\\
\sphinxhline
\sphinxAtStartPar
Prune the Product Tree
&
\sphinxAtStartPar
Helps stakeholders gather and shape requirements
\\
\sphinxhline
\sphinxAtStartPar
Speedboat (aka Sailboat)
&
\sphinxAtStartPar
Identify threats and opportunities (risks)
\\
\sphinxhline
\sphinxAtStartPar
Buy a Feature
&
\sphinxAtStartPar
Prioritization exercise
\\
\sphinxhline
\sphinxAtStartPar
Bang\sphinxhyphen{}for\sphinxhyphen{}the\sphinxhyphen{}Buck
&
\sphinxAtStartPar
Value versus cost rankings
\\
\sphinxbottomrule
\end{tabulary}
\sphinxtableafterendhook\par
\sphinxattableend\end{savenotes}


\paragraph{Remember the Future}
\label{\detokenize{APM/agile:remember-the-future}}

\paragraph{Prune the Product Tree}
\label{\detokenize{APM/agile:prune-the-product-tree}}

\paragraph{Speedboat}
\label{\detokenize{APM/agile:speedboat}}

\subsection{Using Critical Interpersonal Skills}
\label{\detokenize{APM/agile:using-critical-interpersonal-skills}}

\subsubsection{Emotional Intelligence}
\label{\detokenize{APM/agile:emotional-intelligence}}

\subsubsection{Active Listening}
\label{\detokenize{APM/agile:active-listening}}

\subsubsection{Facilitation}
\label{\detokenize{APM/agile:facilitation}}

\subsubsection{Negotiation}
\label{\detokenize{APM/agile:negotiation}}

\subsubsection{Conflict Resolution}
\label{\detokenize{APM/agile:conflict-resolution}}

\subsubsection{Participatory Decision Making}
\label{\detokenize{APM/agile:participatory-decision-making}}

\paragraph{Participatory Decision Models}
\label{\detokenize{APM/agile:participatory-decision-models}}

\subparagraph{Simple Voting}
\label{\detokenize{APM/agile:simple-voting}}

\subparagraph{Thumbs Up/Down/Sideways}
\label{\detokenize{APM/agile:thumbs-up-down-sideways}}

\subparagraph{Fist\sphinxhyphen{}of\sphinxhyphen{}Five Voting}
\label{\detokenize{APM/agile:fist-of-five-voting}}

\subparagraph{Highstmith’s Decision Spectrum}
\label{\detokenize{APM/agile:highstmith-s-decision-spectrum}}

\section{Team Performance}
\label{\detokenize{APM/agile:team-performance}}

\subsection{Agile Team Roles}
\label{\detokenize{APM/agile:agile-team-roles}}\begin{itemize}
\item {} 
\sphinxAtStartPar
Development Team/Delivery Team

\item {} 
\sphinxAtStartPar
Product Owner/Customer/Proxy Customer/Value Management Team/Business Representative

\item {} 
\sphinxAtStartPar
ScrumMaster/Coach/Team Leader

\item {} 
\sphinxAtStartPar
Project Sponsor

\end{itemize}


\subsubsection{Development Team/Delivery Team}
\label{\detokenize{APM/agile:development-team-delivery-team}}

\subsubsection{Product Owner/Customer/Proxy Customer/Value Management Team/Business Representative}
\label{\detokenize{APM/agile:product-owner-customer-proxy-customer-value-management-team-business-representative}}

\subsubsection{ScrumMaster/Coach/Team Leader}
\label{\detokenize{APM/agile:scrummaster-coach-team-leader}}

\subsubsection{Project Sponsor}
\label{\detokenize{APM/agile:project-sponsor}}

\subsection{Building Agile Teams}
\label{\detokenize{APM/agile:building-agile-teams}}
\sphinxAtStartPar
Development Team
\begin{itemize}
\item {} 
\sphinxAtStartPar
Size < 12

\item {} 
\sphinxAtStartPar
Have complementary skills \& generalizing specialists with cross\sphinxhyphen{}functional skills rather than experts in one field

\item {} 
\sphinxAtStartPar
Commited to a common purpose

\item {} 
\sphinxAtStartPar
Hold themselves mutually accountable \sphinxhyphen{}> shared ownership for project outcomes

\end{itemize}


\subsubsection{Characteristics of High\sphinxhyphen{}Performing Teams}
\label{\detokenize{APM/agile:characteristics-of-high-performing-teams}}\begin{itemize}
\item {} 
\sphinxAtStartPar
Create a shared vision for the team

\item {} 
\sphinxAtStartPar
Set realistic goals

\item {} 
\sphinxAtStartPar
Limit team size to 12 or fewer members

\item {} 
\sphinxAtStartPar
Build a sense of team identity

\item {} 
\sphinxAtStartPar
Provide strong leadership

\end{itemize}


\subsubsection{Models of Team Development}
\label{\detokenize{APM/agile:models-of-team-development}}

\paragraph{Shu\sphinxhyphen{}Ha\sphinxhyphen{}Ri Model of Skill Mastery}
\label{\detokenize{APM/agile:shu-ha-ri-model-of-skill-mastery}}

\begin{savenotes}\sphinxattablestart
\sphinxthistablewithglobalstyle
\centering
\begin{tabulary}{\linewidth}[t]{TT}
\sphinxtoprule
\sphinxstyletheadfamily 
\sphinxAtStartPar
Acronym
&\sphinxstyletheadfamily 
\sphinxAtStartPar
Description
\\
\sphinxmidrule
\sphinxtableatstartofbodyhook
\sphinxAtStartPar
Shu
&
\sphinxAtStartPar
Obeying the rules
\\
\sphinxhline
\sphinxAtStartPar
Ha
&
\sphinxAtStartPar
Consciously moving away from the rules
\\
\sphinxhline
\sphinxAtStartPar
Ri
&
\sphinxAtStartPar
Unconsciously finding an individual path
\\
\sphinxbottomrule
\end{tabulary}
\sphinxtableafterendhook\par
\sphinxattableend\end{savenotes}


\paragraph{Dreyfus Model of Adult Skill Acquisition}
\label{\detokenize{APM/agile:dreyfus-model-of-adult-skill-acquisition}}

\begin{savenotes}\sphinxattablestart
\sphinxthistablewithglobalstyle
\centering
\begin{tabulary}{\linewidth}[t]{TTTTT}
\sphinxtoprule
\sphinxstyletheadfamily 
\sphinxAtStartPar
Stage
&\sphinxstyletheadfamily 
\sphinxAtStartPar
Stage
&\sphinxstyletheadfamily 
\sphinxAtStartPar
Commitment
&\sphinxstyletheadfamily 
\sphinxAtStartPar
Decisions
&\sphinxstyletheadfamily 
\sphinxAtStartPar
Perspective
\\
\sphinxmidrule
\sphinxtableatstartofbodyhook
\sphinxAtStartPar
1
&
\sphinxAtStartPar
Novice
&
\sphinxAtStartPar
Detached
&
\sphinxAtStartPar
Analytic
&
\sphinxAtStartPar
None
\\
\sphinxhline
\sphinxAtStartPar
2
&
\sphinxAtStartPar
Advanced beginner
&
\sphinxAtStartPar
Detached
&
\sphinxAtStartPar
Analytic
&
\sphinxAtStartPar
None
\\
\sphinxhline
\sphinxAtStartPar
3
&
\sphinxAtStartPar
Competent
&
\sphinxAtStartPar
Detached understanding and deciding; involved outcome
&
\sphinxAtStartPar
Analytic
&
\sphinxAtStartPar
Chosen
\\
\sphinxhline
\sphinxAtStartPar
4
&
\sphinxAtStartPar
Proficient
&
\sphinxAtStartPar
Involved understanding; detached deciding
&
\sphinxAtStartPar
Analytic
&
\sphinxAtStartPar
Experienced
\\
\sphinxhline
\sphinxAtStartPar
5
&
\sphinxAtStartPar
Expert
&
\sphinxAtStartPar
Involved
&
\sphinxAtStartPar
Intuitive
&
\sphinxAtStartPar
Experienced
\\
\sphinxbottomrule
\end{tabulary}
\sphinxtableafterendhook\par
\sphinxattableend\end{savenotes}


\paragraph{Tuckman Model of Team Formation and Development}
\label{\detokenize{APM/agile:tuckman-model-of-team-formation-and-development}}

\begin{savenotes}\sphinxattablestart
\sphinxthistablewithglobalstyle
\centering
\begin{tabulary}{\linewidth}[t]{T}
\sphinxtoprule
\sphinxstyletheadfamily 
\sphinxAtStartPar
\sphinxincludegraphics{{agile-tuckman}.svg}
\\
\sphinxmidrule
\sphinxtableatstartofbodyhook
\sphinxAtStartPar
Tuckman model team formation and development stages
\\
\sphinxbottomrule
\end{tabulary}
\sphinxtableafterendhook\par
\sphinxattableend\end{savenotes}


\begin{savenotes}\sphinxattablestart
\sphinxthistablewithglobalstyle
\centering
\begin{tabulary}{\linewidth}[t]{TT}
\sphinxtoprule
\sphinxstyletheadfamily 
\sphinxAtStartPar
Stage
&\sphinxstyletheadfamily 
\sphinxAtStartPar
Description
\\
\sphinxmidrule
\sphinxtableatstartofbodyhook
\sphinxAtStartPar
Forming
&
\sphinxAtStartPar
Working group
\\
\sphinxhline
\sphinxAtStartPar
Storming
&
\sphinxAtStartPar
Pseudo team\(\rightarrow\) Potential team
\\
\sphinxhline
\sphinxAtStartPar
Norming
&
\sphinxAtStartPar
Potential team\(\rightarrow\) Real team
\\
\sphinxhline
\sphinxAtStartPar
Performing
&
\sphinxAtStartPar
Real team\(\rightarrow\) High performing team
\\
\sphinxbottomrule
\end{tabulary}
\sphinxtableafterendhook\par
\sphinxattableend\end{savenotes}


\subsubsection{Adaptive Leadership}
\label{\detokenize{APM/agile:adaptive-leadership}}

\begin{savenotes}\sphinxattablestart
\sphinxthistablewithglobalstyle
\centering
\begin{tabulary}{\linewidth}[t]{TTT}
\sphinxtoprule
\sphinxstyletheadfamily 
\sphinxAtStartPar
Stage
&\sphinxstyletheadfamily 
\sphinxAtStartPar
Team Stage
&\sphinxstyletheadfamily 
\sphinxAtStartPar
Leadership Style
\\
\sphinxmidrule
\sphinxtableatstartofbodyhook
\sphinxAtStartPar
1
&
\sphinxAtStartPar
Forming
&
\sphinxAtStartPar
Directing
\\
\sphinxhline
\sphinxAtStartPar
2
&
\sphinxAtStartPar
Storming
&
\sphinxAtStartPar
Coaching
\\
\sphinxhline
\sphinxAtStartPar
3
&
\sphinxAtStartPar
Norming
&
\sphinxAtStartPar
Supporting
\\
\sphinxhline
\sphinxAtStartPar
4
&
\sphinxAtStartPar
Performing
&
\sphinxAtStartPar
Delegating
\\
\sphinxbottomrule
\end{tabulary}
\sphinxtableafterendhook\par
\sphinxattableend\end{savenotes}


\subsubsection{Team Motivation}
\label{\detokenize{APM/agile:team-motivation}}

\begin{savenotes}\sphinxattablestart
\sphinxthistablewithglobalstyle
\centering
\begin{tabulary}{\linewidth}[t]{T}
\sphinxtoprule
\sphinxstyletheadfamily 
\sphinxAtStartPar
\sphinxincludegraphics{{agile-net_contribution}.svg}
\\
\sphinxmidrule
\sphinxtableatstartofbodyhook
\sphinxAtStartPar
Continuum of Net Contribution
\\
\sphinxbottomrule
\end{tabulary}
\sphinxtableafterendhook\par
\sphinxattableend\end{savenotes}


\subsubsection{Training, Coaching, and Mentoring}
\label{\detokenize{APM/agile:training-coaching-and-mentoring}}

\paragraph{Training}
\label{\detokenize{APM/agile:training}}
\sphinxAtStartPar
Training = teaching skills/knowledge through practice and instructions


\paragraph{Coaching}
\label{\detokenize{APM/agile:coaching}}
\sphinxAtStartPar
Coaching = facilitated process that helps developing and improving performance

\sphinxAtStartPar
Guidelines for 1\sphinxhyphen{}to\sphinxhyphen{}1 coaching:
\begin{itemize}
\item {} 
\sphinxAtStartPar
Meet them a half\sphinxhyphen{}step ahead

\item {} 
\sphinxAtStartPar
Guarantee safety

\item {} 
\sphinxAtStartPar
Partner with managers

\item {} 
\sphinxAtStartPar
Create positive regard

\end{itemize}


\paragraph{Mentoring}
\label{\detokenize{APM/agile:mentoring}}
\sphinxAtStartPar
Mentoring = professional relantionship where:
\begin{itemize}
\item {} 
\sphinxAtStartPar
Mentor \(\rightarrow\) tackles issues on as an\sphinxhyphen{}needed basis

\item {} 
\sphinxAtStartPar
Mentee \(\rightarrow\) free\sphinxhyphen{}flowing agenda

\end{itemize}


\subsection{Creating Collaborative Team Spaces}
\label{\detokenize{APM/agile:creating-collaborative-team-spaces}}

\subsubsection{Co\sphinxhyphen{}located Teams}
\label{\detokenize{APM/agile:co-located-teams}}

\subsubsection{Team Space}
\label{\detokenize{APM/agile:team-space}}

\subsubsection{Osmotic Communication}
\label{\detokenize{APM/agile:osmotic-communication}}

\subsubsection{Global, Cultural, and Team Diversity}
\label{\detokenize{APM/agile:global-cultural-and-team-diversity}}

\subsubsection{Distributed Teams}
\label{\detokenize{APM/agile:distributed-teams}}

\subsection{Tracking Team Performance}
\label{\detokenize{APM/agile:tracking-team-performance}}

\subsubsection{Burn Charts}
\label{\detokenize{APM/agile:burn-charts}}

\paragraph{Burndown Charts}
\label{\detokenize{APM/agile:burndown-charts}}

\begin{savenotes}\sphinxattablestart
\sphinxthistablewithglobalstyle
\centering
\begin{tabulary}{\linewidth}[t]{T}
\sphinxtoprule
\sphinxstyletheadfamily 
\sphinxAtStartPar
\sphinxincludegraphics{{agile-burndown}.svg}
\\
\sphinxmidrule
\sphinxtableatstartofbodyhook
\sphinxAtStartPar
Burndown chart
\\
\sphinxbottomrule
\end{tabulary}
\sphinxtableafterendhook\par
\sphinxattableend\end{savenotes}


\paragraph{Burnup Charts}
\label{\detokenize{APM/agile:burnup-charts}}

\begin{savenotes}\sphinxattablestart
\sphinxthistablewithglobalstyle
\centering
\begin{tabulary}{\linewidth}[t]{T}
\sphinxtoprule
\sphinxstyletheadfamily 
\sphinxAtStartPar
\sphinxincludegraphics{{agile-burnup}.svg}
\\
\sphinxmidrule
\sphinxtableatstartofbodyhook
\sphinxAtStartPar
Burnup chart
\\
\sphinxbottomrule
\end{tabulary}
\sphinxtableafterendhook\par
\sphinxattableend\end{savenotes}


\subsubsection{Velocity}
\label{\detokenize{APM/agile:velocity}}
\sphinxAtStartPar
\(Velocity=Work/Iteration\), where \(Work=StoryPoints, UserStories, Hours,...\)


\begin{savenotes}\sphinxattablestart
\sphinxthistablewithglobalstyle
\centering
\begin{tabulary}{\linewidth}[t]{T}
\sphinxtoprule
\sphinxstyletheadfamily 
\sphinxAtStartPar
\sphinxincludegraphics{{agile-velocity}.svg}
\\
\sphinxmidrule
\sphinxtableatstartofbodyhook
\sphinxAtStartPar
Velocity chart
\\
\sphinxbottomrule
\end{tabulary}
\sphinxtableafterendhook\par
\sphinxattableend\end{savenotes}


\section{Adaptive Planning}
\label{\detokenize{APM/agile:adaptive-planning}}

\subsection{Agile Planning Concepts}
\label{\detokenize{APM/agile:agile-planning-concepts}}

\subsubsection{Adaptive Planning}
\label{\detokenize{APM/agile:id10}}

\subsubsection{Agile versus Non\sphinxhyphen{}Agile Planning}
\label{\detokenize{APM/agile:agile-versus-non-agile-planning}}

\subsubsection{Principles of Agile Planning}
\label{\detokenize{APM/agile:principles-of-agile-planning}}

\subsubsection{Agile Discovery}
\label{\detokenize{APM/agile:agile-discovery}}

\subsubsection{Progressive Elaboration}
\label{\detokenize{APM/agile:progressive-elaboration}}

\subsubsection{Value\sphinxhyphen{}Based Analysis}
\label{\detokenize{APM/agile:value-based-analysis}}

\subsubsection{Value\sphinxhyphen{}Based Decomposition}
\label{\detokenize{APM/agile:value-based-decomposition}}

\subsubsection{Timeboxing}
\label{\detokenize{APM/agile:timeboxing}}

\subsubsection{Estimate Ranges}
\label{\detokenize{APM/agile:estimate-ranges}}

\subsubsection{Ideal Time}
\label{\detokenize{APM/agile:ideal-time}}
\sphinxAtStartPar
Ideal Time = Task duration without distractions

\sphinxAtStartPar
Likely Time = Task duration with distractions


\subsection{Tools for Sizing and Estimating}
\label{\detokenize{APM/agile:tools-for-sizing-and-estimating}}

\subsubsection{Sizing, Estimating, and Planning}
\label{\detokenize{APM/agile:sizing-estimating-and-planning}}

\subsubsection{Decomposition Requirements}
\label{\detokenize{APM/agile:decomposition-requirements}}

\paragraph{Requirements Are Decomposed “Just in Time”}
\label{\detokenize{APM/agile:requirements-are-decomposed-just-in-time}}

\subsubsection{User Stories}
\label{\detokenize{APM/agile:user-stories}}

\paragraph{Creating the User Stories}
\label{\detokenize{APM/agile:creating-the-user-stories}}
\sphinxAtStartPar
Template 1
\begin{quote}

\sphinxAtStartPar
As a \sphinxcode{\sphinxupquote{<Role>,}} I want \sphinxcode{\sphinxupquote{<Functionality>}}, so that \sphinxcode{\sphinxupquote{<Business benefit>}}.
\end{quote}

\sphinxAtStartPar
Template 2
\begin{quote}

\sphinxAtStartPar
Given

\sphinxAtStartPar
When

\sphinxAtStartPar
Then
\end{quote}


\subparagraph{The Three C’s}
\label{\detokenize{APM/agile:the-three-c-s}}

\begin{savenotes}\sphinxattablestart
\sphinxthistablewithglobalstyle
\centering
\begin{tabulary}{\linewidth}[t]{TT}
\sphinxtoprule
\sphinxstyletheadfamily 
\sphinxAtStartPar
C
&\sphinxstyletheadfamily 
\sphinxAtStartPar
Description
\\
\sphinxmidrule
\sphinxtableatstartofbodyhook
\sphinxAtStartPar
Card
&
\sphinxAtStartPar

\\
\sphinxhline
\sphinxAtStartPar
Conversation
&
\sphinxAtStartPar

\\
\sphinxhline
\sphinxAtStartPar
Confirmation
&
\sphinxAtStartPar

\\
\sphinxbottomrule
\end{tabulary}
\sphinxtableafterendhook\par
\sphinxattableend\end{savenotes}


\subparagraph{INVEST: Characteristics of Effective User Stories}
\label{\detokenize{APM/agile:invest-characteristics-of-effective-user-stories}}

\begin{savenotes}\sphinxattablestart
\sphinxthistablewithglobalstyle
\centering
\begin{tabulary}{\linewidth}[t]{TT}
\sphinxtoprule
\sphinxstyletheadfamily 
\sphinxAtStartPar
Letter
&\sphinxstyletheadfamily 
\sphinxAtStartPar
Description
\\
\sphinxmidrule
\sphinxtableatstartofbodyhook
\sphinxAtStartPar
\sphinxstylestrong{I}ndependent
&
\sphinxAtStartPar

\\
\sphinxhline
\sphinxAtStartPar
\sphinxstylestrong{N}egotiable
&
\sphinxAtStartPar

\\
\sphinxhline
\sphinxAtStartPar
\sphinxstylestrong{V}aluable
&
\sphinxAtStartPar

\\
\sphinxhline
\sphinxAtStartPar
\sphinxstylestrong{E}stimatable
&
\sphinxAtStartPar

\\
\sphinxhline
\sphinxAtStartPar
\sphinxstylestrong{S}mall
&
\sphinxAtStartPar

\\
\sphinxhline
\sphinxAtStartPar
\sphinxstylestrong{T}estable
&
\sphinxAtStartPar

\\
\sphinxbottomrule
\end{tabulary}
\sphinxtableafterendhook\par
\sphinxattableend\end{savenotes}


\subsubsection{User Story Backlog (Product Backlog)}
\label{\detokenize{APM/agile:user-story-backlog-product-backlog}}

\subsubsection{Refining (Grooming) the Backlog}
\label{\detokenize{APM/agile:refining-grooming-the-backlog}}

\subsubsection{Relative Sizing and Story Points}
\label{\detokenize{APM/agile:relative-sizing-and-story-points}}

\paragraph{The Fibonacci Sequence}
\label{\detokenize{APM/agile:the-fibonacci-sequence}}

\paragraph{Guidelines for Using Story Points}
\label{\detokenize{APM/agile:guidelines-for-using-story-points}}\begin{itemize}
\item {} 
\sphinxAtStartPar
The team should own the definition of their story points

\item {} 
\sphinxAtStartPar
Story point estimates shouldbe all\sphinxhyphen{}inclusive

\item {} 
\sphinxAtStartPar
The point sizes should be relative

\item {} 
\sphinxAtStartPar
When disaggregating estimates, the totals don’t need to match

\item {} 
\sphinxAtStartPar
Complexity, work effort, andrisk should all be included in the estimates

\end{itemize}


\subsubsection{Affinity Estimating}
\label{\detokenize{APM/agile:affinity-estimating}}

\subsubsection{T\sphinxhyphen{}shirt Sizing}
\label{\detokenize{APM/agile:t-shirt-sizing}}
\sphinxAtStartPar
ES < S < M < L < XL < XXL


\subsubsection{Story Maps}
\label{\detokenize{APM/agile:story-maps}}

\subsubsection{Product Roadmap}
\label{\detokenize{APM/agile:product-roadmap}}

\subsubsection{Wideband Delphi}
\label{\detokenize{APM/agile:wideband-delphi}}
\sphinxAtStartPar
Biases:
\begin{itemize}
\item {} 
\sphinxAtStartPar
Bandwagon

\item {} 
\sphinxAtStartPar
HIPPO

\item {} 
\sphinxAtStartPar
Groupthink

\end{itemize}

\sphinxAtStartPar
Characteristics:
\begin{itemize}
\item {} 
\sphinxAtStartPar
Iterative

\item {} 
\sphinxAtStartPar
Adaptive

\item {} 
\sphinxAtStartPar
Collaborative

\end{itemize}


\subsubsection{Planning Poker}
\label{\detokenize{APM/agile:planning-poker}}

\subsection{Release and Iteration Planning}
\label{\detokenize{APM/agile:release-and-iteration-planning}}

\subsubsection{Spikes}
\label{\detokenize{APM/agile:spikes}}

\paragraph{Architectural Spike}
\label{\detokenize{APM/agile:architectural-spike}}
\sphinxAtStartPar
Short, timeboxed effort dedicated to “proof of concept” — checking whether the approach the team hopes to use will work for the project


\paragraph{Risk\sphinxhyphen{}Based Spike}
\label{\detokenize{APM/agile:risk-based-spike}}
\sphinxAtStartPar
Short, timeboxed effort that makes the team sets aside to investigate, reduce or eliminate an issue/threat


\subsubsection{Release Planning}
\label{\detokenize{APM/agile:release-planning}}

\paragraph{Selecting the User Stories for the Release}
\label{\detokenize{APM/agile:selecting-the-user-stories-for-the-release}}

\paragraph{How Much Can We Get Done?}
\label{\detokenize{APM/agile:how-much-can-we-get-done}}

\paragraph{Estimating Velocity for the First Iteration}
\label{\detokenize{APM/agile:estimating-velocity-for-the-first-iteration}}

\paragraph{Slicing the Stories}
\label{\detokenize{APM/agile:slicing-the-stories}}

\subsubsection{Iteration Planning}
\label{\detokenize{APM/agile:iteration-planning}}

\paragraph{The Iteration Planning Process}
\label{\detokenize{APM/agile:the-iteration-planning-process}}\begin{itemize}
\item {} 
\sphinxAtStartPar
Discuss the user stories in the backlog

\item {} 
\sphinxAtStartPar
Select the user stories for the iteration

\item {} 
\sphinxAtStartPar
Define the acceptance criteria and write the acceptance tests for the stories

\item {} 
\sphinxAtStartPar
Break down the user stories into tasks

\item {} 
\sphinxAtStartPar
Estimate the tasks

\end{itemize}


\paragraph{Iteration Planning Summary}
\label{\detokenize{APM/agile:iteration-planning-summary}}

\paragraph{Selecting the User Stories}
\label{\detokenize{APM/agile:selecting-the-user-stories}}

\paragraph{Defining the Acceptance Criteria nad Writing the Acceptance Tests}
\label{\detokenize{APM/agile:defining-the-acceptance-criteria-nad-writing-the-acceptance-tests}}

\paragraph{Estimating the Tasks}
\label{\detokenize{APM/agile:estimating-the-tasks}}

\paragraph{Use Actual Results to Refine Estimates}
\label{\detokenize{APM/agile:use-actual-results-to-refine-estimates}}

\subsection{Daily Stand\sphinxhyphen{}Ups}
\label{\detokenize{APM/agile:daily-stand-ups}}

\section{Problem Detection and Resolution}
\label{\detokenize{APM/agile:problem-detection-and-resolution}}

\subsection{Detecting Problems}
\label{\detokenize{APM/agile:detecting-problems}}

\subsubsection{Lead Time and Cycle Time}
\label{\detokenize{APM/agile:lead-time-and-cycle-time}}
\sphinxAtStartPar
\(Task.LT=Time(ToDo \rightarrow Done)\)


\paragraph{Cycle Time, WIP, and Throughput}
\label{\detokenize{APM/agile:cycle-time-wip-and-throughput}}
\sphinxAtStartPar
\(CT=WIP/TH\)


\paragraph{Throughput and Productivity}
\label{\detokenize{APM/agile:throughput-and-productivity}}
\sphinxAtStartPar
\(TH=Work/Time\)

\sphinxAtStartPar
\(Productivity=Work/TeamMember\)


\subsubsection{Defects}
\label{\detokenize{APM/agile:defects}}
\sphinxAtStartPar
\(Defect.CT=Time(Occurred \rightarrow Fixed)\)


\paragraph{Defect Rates}
\label{\detokenize{APM/agile:defect-rates}}
\sphinxAtStartPar
\(DefectRate=\#Defect/Time\)


\subsubsection{Variance Analysis}
\label{\detokenize{APM/agile:variance-analysis}}

\paragraph{Causes of Variation}
\label{\detokenize{APM/agile:causes-of-variation}}\begin{itemize}
\item {} 
\sphinxAtStartPar
Common cause = average day\sphinxhyphen{}2\sphinxhyphen{}day differences of doing work

\item {} 
\sphinxAtStartPar
Special cause = greater degree of variance \(\leftarrow\) special/new factors

\end{itemize}


\subsubsection{Trend Analysis}
\label{\detokenize{APM/agile:trend-analysis}}
\sphinxAtStartPar
Metrics
\begin{itemize}
\item {} 
\sphinxAtStartPar
Lagging \(\rightarrow\) view of past

\item {} 
\sphinxAtStartPar
Leading \(\rightarrow\) view of future/what is occurring now/starting to happen \(\rightarrow\) can adapt/replan accordinly

\end{itemize}


\subsubsection{Control Limits}
\label{\detokenize{APM/agile:control-limits}}

\subsection{Managing Threats and Issues}
\label{\detokenize{APM/agile:managing-threats-and-issues}}

\subsubsection{Risk\sphinxhyphen{}Adjusted Backlog}
\label{\detokenize{APM/agile:risk-adjusted-backlog}}

\paragraph{Creating the Risk\sphinxhyphen{}Adjusted Backlog}
\label{\detokenize{APM/agile:creating-the-risk-adjusted-backlog}}
\sphinxAtStartPar
\(EVM = Probability \cdot Impact [\$]\)


\subsubsection{Risk Severity}
\label{\detokenize{APM/agile:risk-severity}}
\sphinxAtStartPar
\(Severity=Probability \cdot Impact [l/m/h]\)


\subsubsection{Risk Burndown Graphs}
\label{\detokenize{APM/agile:risk-burndown-graphs}}

\subsection{Solving Problems}
\label{\detokenize{APM/agile:solving-problems}}

\subsubsection{Problem Solving as Continuous Improvement}
\label{\detokenize{APM/agile:problem-solving-as-continuous-improvement}}

\subsubsection{Engage the Team}
\label{\detokenize{APM/agile:engage-the-team}}

\paragraph{The Benefits of Team Engagement}
\label{\detokenize{APM/agile:the-benefits-of-team-engagement}}\begin{itemize}
\item {} 
\sphinxAtStartPar
By asking the team for a solution, we inherit consensus for the proposal

\item {} 
\sphinxAtStartPar
Engaging the team accesses a broader knowledge base

\item {} 
\sphinxAtStartPar
Team solutions are practical

\item {} 
\sphinxAtStartPar
When consulted, people work hard to generate good ideas

\item {} 
\sphinxAtStartPar
Asking for help shows confidence, not weakness

\item {} 
\sphinxAtStartPar
Seeking others’ ideas models desired behavior

\end{itemize}


\paragraph{Considerations and Cautions for Engaging the Team}
\label{\detokenize{APM/agile:considerations-and-cautions-for-engaging-the-team}}\begin{itemize}
\item {} 
\sphinxAtStartPar
Involve the team where it can be most helpful

\item {} 
\sphinxAtStartPar
Solve real problems

\item {} 
\sphinxAtStartPar
Team cohesion is necessary

\item {} 
\sphinxAtStartPar
Check in after team or project changes

\item {} 
\sphinxAtStartPar
Be sure to follow through

\end{itemize}


\section{Continuous Improvement}
\label{\detokenize{APM/agile:continuous-improvement}}

\subsection{Kaizen}
\label{\detokenize{APM/agile:kaizen}}
\sphinxAtStartPar
Kaizen = process for continuous improvement

\sphinxAtStartPar
Focus on:
\begin{itemize}
\item {} 
\sphinxAtStartPar
encourage the team

\item {} 
\sphinxAtStartPar
frequently initiate and implement small, incremental improvements

\end{itemize}

\sphinxAtStartPar
PDCA Cycle = Plan \sphinxhyphen{} Do \sphinxhyphen{} Check \sphinxhyphen{} Act


\subsection{Continuous Improvement—Process}
\label{\detokenize{APM/agile:continuous-improvement-process}}

\subsubsection{Process Tailoring}
\label{\detokenize{APM/agile:process-tailoring}}
\sphinxAtStartPar
Process tailor = adapting our implementation of agile to better fit our project environment

\sphinxAtStartPar
Teams new to agile should use their methodology “out\sphinxhyphen{}of\sphinxhyphen{}the\sphinxhyphen{}box” for a few projects before attempting to change it

\sphinxAtStartPar
All techniques and practices in an agile methodology are designed to work in balance with each other


\paragraph{Hybrid Models}
\label{\detokenize{APM/agile:hybrid-models}}

\subparagraph{Agile\sphinxhyphen{}Agile Hybrid: Scrum\sphinxhyphen{}XP}
\label{\detokenize{APM/agile:agile-agile-hybrid-scrum-xp}}

\begin{savenotes}\sphinxattablestart
\sphinxthistablewithglobalstyle
\centering
\begin{tabulary}{\linewidth}[t]{TT}
\sphinxtoprule
\sphinxstyletheadfamily 
\sphinxAtStartPar
Methodology
&\sphinxstyletheadfamily 
\sphinxAtStartPar
Focus
\\
\sphinxmidrule
\sphinxtableatstartofbodyhook
\sphinxAtStartPar
XP
&
\sphinxAtStartPar
Technical guidance
\\
\sphinxhline
\sphinxAtStartPar
Scrum
&
\sphinxAtStartPar
Project governance
\\
\sphinxbottomrule
\end{tabulary}
\sphinxtableafterendhook\par
\sphinxattableend\end{savenotes}


\subparagraph{Agile\sphinxhyphen{}Traditional Hybrids}
\label{\detokenize{APM/agile:agile-traditional-hybrids}}
\sphinxAtStartPar
Implement agile components into linear project execution


\subsubsection{Systems Thinking}
\label{\detokenize{APM/agile:systems-thinking}}
\sphinxAtStartPar
Understand the systems\sphinxhyphen{}level environment for the project

\sphinxAtStartPar
Figure


\subsubsection{Process Analysis}
\label{\detokenize{APM/agile:process-analysis}}
\sphinxAtStartPar
Process Analysis = reviewing and diagnosing issues with a team’s agile methods


\paragraph{Methodology Anti\sphinxhyphen{}Patterns}
\label{\detokenize{APM/agile:methodology-anti-patterns}}\begin{itemize}
\item {} 
\sphinxAtStartPar
One size for all projects

\item {} 
\sphinxAtStartPar
Intolerant

\item {} 
\sphinxAtStartPar
Heavy

\item {} 
\sphinxAtStartPar
Embellished

\item {} 
\sphinxAtStartPar
Untried

\item {} 
\sphinxAtStartPar
Used Once

\end{itemize}


\paragraph{Success Criteria}
\label{\detokenize{APM/agile:success-criteria}}\begin{itemize}
\item {} 
\sphinxAtStartPar
Project got shipped

\item {} 
\sphinxAtStartPar
Leadership remained intact

\item {} 
\sphinxAtStartPar
Team would work the same way again

\end{itemize}


\paragraph{Methodology Success Patterns}
\label{\detokenize{APM/agile:methodology-success-patterns}}\begin{itemize}
\item {} 
\sphinxAtStartPar
Interactive, face\sphinxhyphen{}to\sphinxhyphen{}face communication is the cheapest channel for exchanging information

\item {} 
\sphinxAtStartPar
Excess methodology weight is costly

\item {} 
\sphinxAtStartPar
Larger teams need heavier methodologies

\item {} 
\sphinxAtStartPar
Projects with greater criticality require greater ceremony

\item {} 
\sphinxAtStartPar
Feedback and communication reduce the need for intermediate deliverables

\item {} 
\sphinxAtStartPar
Discipline, skills, and understanding counter process, formality, and coumentation

\item {} 
\sphinxAtStartPar
Efficiency is expendable in nonbottleneck activities

\end{itemize}


\subsubsection{Value Stream Mapping}
\label{\detokenize{APM/agile:value-stream-mapping}}

\paragraph{Process}
\label{\detokenize{APM/agile:id11}}\begin{enumerate}
\sphinxsetlistlabels{\arabic}{enumi}{enumii}{}{.}%
\item {} 
\sphinxAtStartPar
Identify product/service to be analyzed

\item {} 
\sphinxAtStartPar
Create a value stream map of the current process, identifying steps, queues, delays, and information flows

\item {} 
\sphinxAtStartPar
Review the map to find delays, waste, and constraints

\item {} 
\sphinxAtStartPar
Create a new value stream map of the desired future state of the process, optimized to remove or reduce delays, waste, and constraints

\item {} 
\sphinxAtStartPar
Develop a roadmap for creating the optimized state

\item {} 
\sphinxAtStartPar
Plan to revisit the process in the future to continually refine and optimize it

\end{enumerate}


\paragraph{Metrics}
\label{\detokenize{APM/agile:metrics}}

\begin{savenotes}\sphinxattablestart
\sphinxthistablewithglobalstyle
\centering
\begin{tabulary}{\linewidth}[t]{TT}
\sphinxtoprule
\sphinxstyletheadfamily 
\sphinxAtStartPar
Term
&\sphinxstyletheadfamily 
\sphinxAtStartPar
Formula
\\
\sphinxmidrule
\sphinxtableatstartofbodyhook
\sphinxAtStartPar
Total cycle time
&
\sphinxAtStartPar
\(TCT=VAT+NVAT\)
\\
\sphinxhline
\sphinxAtStartPar
Value\sphinxhyphen{}added time
&
\sphinxAtStartPar
\(VAT\)
\\
\sphinxhline
\sphinxAtStartPar
Nonvalue\sphinxhyphen{}added time
&
\sphinxAtStartPar
\(NVAT\)
\\
\sphinxhline
\sphinxAtStartPar
Process cycle efficiency
&
\sphinxAtStartPar
\(VAT/TCT=VAT/(VAT+NVAT)\)
\\
\sphinxbottomrule
\end{tabulary}
\sphinxtableafterendhook\par
\sphinxattableend\end{savenotes}


\subsubsection{Project Pre\sphinxhyphen{}Mortems}
\label{\detokenize{APM/agile:project-pre-mortems}}\begin{enumerate}
\sphinxsetlistlabels{\arabic}{enumi}{enumii}{}{.}%
\item {} 
\sphinxAtStartPar
Imagine the Failure

\item {} 
\sphinxAtStartPar
Generate the Reasons for Failure

\item {} 
\sphinxAtStartPar
Consolidate the List

\item {} 
\sphinxAtStartPar
Revisit the Plan

\end{enumerate}


\subsection{Continuous Improvement—Product}
\label{\detokenize{APM/agile:continuous-improvement-product}}

\subsubsection{Reviews}
\label{\detokenize{APM/agile:reviews}}

\paragraph{The Scientific Method}
\label{\detokenize{APM/agile:the-scientific-method}}

\subsubsection{Product Feedback Loops and Learning Cycles}
\label{\detokenize{APM/agile:product-feedback-loops-and-learning-cycles}}

\subsubsection{Feedback Methods}
\label{\detokenize{APM/agile:feedback-methods}}

\subsubsection{Approved Iterations}
\label{\detokenize{APM/agile:approved-iterations}}

\subsection{Continuous Improvement—People}
\label{\detokenize{APM/agile:continuous-improvement-people}}

\subsubsection{Retrospectives}
\label{\detokenize{APM/agile:retrospectives}}

\paragraph{Benefits}
\label{\detokenize{APM/agile:benefits}}\begin{itemize}
\item {} 
\sphinxAtStartPar
Improved productivity

\item {} 
\sphinxAtStartPar
Improved capability

\item {} 
\sphinxAtStartPar
Improved quality

\item {} 
\sphinxAtStartPar
Improved capacity

\end{itemize}


\paragraph{Process}
\label{\detokenize{APM/agile:id12}}

\begin{savenotes}\sphinxattablestart
\sphinxthistablewithglobalstyle
\centering
\begin{tabulary}{\linewidth}[t]{TTT}
\sphinxtoprule
\sphinxstyletheadfamily 
\sphinxAtStartPar
Stage
&\sphinxstyletheadfamily 
\sphinxAtStartPar
Name
&\sphinxstyletheadfamily 
\sphinxAtStartPar
Typical Time
\\
\sphinxmidrule
\sphinxtableatstartofbodyhook
\sphinxAtStartPar
1
&
\sphinxAtStartPar
Set stage
&
\sphinxAtStartPar
6
\\
\sphinxhline
\sphinxAtStartPar
2
&
\sphinxAtStartPar
Gather data
&
\sphinxAtStartPar
40
\\
\sphinxhline
\sphinxAtStartPar
3
&
\sphinxAtStartPar
Generate insights
&
\sphinxAtStartPar
25
\\
\sphinxhline
\sphinxAtStartPar
4
&
\sphinxAtStartPar
Decide what to do
&
\sphinxAtStartPar
20
\\
\sphinxhline
\sphinxAtStartPar
5
&
\sphinxAtStartPar
Close retrospective
&
\sphinxAtStartPar
20
\\
\sphinxbottomrule
\end{tabulary}
\sphinxtableafterendhook\par
\sphinxattableend\end{savenotes}


\paragraph{Set stage}
\label{\detokenize{APM/agile:set-stage}}

\subparagraph{Activities}
\label{\detokenize{APM/agile:activities}}\begin{itemize}
\item {} 
\sphinxAtStartPar
Check\sphinxhyphen{}In

\item {} 
\sphinxAtStartPar
Focus On/Off
\begin{itemize}
\item {} 
\sphinxAtStartPar
Inquiry rather than Advocacy

\item {} 
\sphinxAtStartPar
Dialogue rather than Debate

\item {} 
\sphinxAtStartPar
Conversation rather than Argument

\item {} 
\sphinxAtStartPar
Udnerstanding rather than Defending

\end{itemize}

\item {} 
\sphinxAtStartPar
ESVP
\begin{itemize}
\item {} 
\sphinxAtStartPar
Explorers

\item {} 
\sphinxAtStartPar
Shoppers

\item {} 
\sphinxAtStartPar
Vacationers

\item {} 
\sphinxAtStartPar
Prisoners

\end{itemize}

\item {} 
\sphinxAtStartPar
Working Agreements

\end{itemize}


\paragraph{Gather data}
\label{\detokenize{APM/agile:gather-data}}

\subparagraph{Techniques}
\label{\detokenize{APM/agile:techniques}}\begin{itemize}
\item {} 
\sphinxAtStartPar
Timeline

\item {} 
\sphinxAtStartPar
Triple Nickels

\item {} 
\sphinxAtStartPar
Color Code Dots

\item {} 
\sphinxAtStartPar
Mad, Sad, Glad

\item {} 
\sphinxAtStartPar
Locate Strengths

\item {} 
\sphinxAtStartPar
Satisfcation Histogram

\item {} 
\sphinxAtStartPar
Team Radar

\item {} 
\sphinxAtStartPar
Like to Like

\end{itemize}


\paragraph{Generate insights}
\label{\detokenize{APM/agile:generate-insights}}

\subparagraph{Five Whys}
\label{\detokenize{APM/agile:five-whys}}

\subparagraph{Fishbone Analysis}
\label{\detokenize{APM/agile:fishbone-analysis}}

\paragraph{Decide what to do}
\label{\detokenize{APM/agile:decide-what-to-do}}

\paragraph{Close retrospective}
\label{\detokenize{APM/agile:close-retrospective}}

\subsubsection{Team Self\sphinxhyphen{}Assessments}
\label{\detokenize{APM/agile:team-self-assessments}}

\paragraph{Shore’s Team Self\sphinxhyphen{}Assessment Scoring Model}
\label{\detokenize{APM/agile:shore-s-team-self-assessment-scoring-model}}

\paragraph{Tabaka’s Team Self\sphinxhyphen{}Assessment Model}
\label{\detokenize{APM/agile:tabaka-s-team-self-assessment-model}}
\sphinxstepscope


\chapter{Scrum Body of Knowledge}
\label{\detokenize{APM/sbok:scrum-body-of-knowledge}}\label{\detokenize{APM/sbok::doc}}
\sphinxAtStartPar
Notes from SCRUMstudy™ 2022, \sphinxstyleemphasis{A Guide to the SCRUM BODY OF KNOWLEDGE (SBOK® GUIDE),} Fourth Edition.


\section{Introduction}
\label{\detokenize{APM/sbok:introduction}}

\subsection{Overview of Scrum}
\label{\detokenize{APM/sbok:overview-of-scrum}}
\sphinxAtStartPar
A Scrum project involves a collaborative effort to create a new product, service, or other results defined in the \sphinxstylestrong{Project Vision Statement}.

\sphinxAtStartPar
Projects are impacted by time, cost, scope, quality, resources, organizational capabilities, and other limitations that make it difficult to plan , execute, manage, and ultimately succeed.

\sphinxAtStartPar
Scrum:
\begin{itemize}
\item {} 
\sphinxAtStartPar
is an adaptive, iterative, fast, flexible, and effective framework designed to deliver significant value quickly and throughout a project, and

\item {} 
\sphinxAtStartPar
may also be used to manage the continuous maintenance of products and services, track issues, and manage changes.

\end{itemize}

\sphinxAtStartPar
A key strength of Scrum is its use of cross\sphinxhyphen{}functional, self\sphinxhyphen{}organized, and empowered teams who divide their work into short, concentrated work cycles called \sphinxstylestrong{Sprints}.


\begin{savenotes}\sphinxattablestart
\sphinxthistablewithglobalstyle
\centering
\begin{tabulary}{\linewidth}[t]{T}
\sphinxtoprule
\sphinxstyletheadfamily 
\sphinxAtStartPar
\sphinxincludegraphics{{sbok-1}.svg}
\\
\sphinxmidrule
\sphinxtableatstartofbodyhook
\sphinxAtStartPar
Figure 1 — Scrum Flow for 1x Sprint
\\
\sphinxbottomrule
\end{tabulary}
\sphinxtableafterendhook\par
\sphinxattableend\end{savenotes}

\sphinxAtStartPar
The Scrum cycle begins with a \sphinxstylestrong{Stakeholder Meeting}, during which the \sphinxstyleemphasis{Project Vision} is created. The Product Owner (PO) then develops a \sphinxstylestrong{Prioritized Product Backlog} which contains a prioritized list of business and project requirements in the form of User Stories (USs). Each **Sprint **begins with a \sphinxstylestrong{Sprint Planning Meeting}, during which high priority USs are considered for inclusion in the \sphinxstylestrong{Sprint}. A **Sprint **generally lasts one to four weeks and involves the Developers Team (DT) working to create potentially shippable deliverables or product increments. During the \sphinxstylestrong{Sprint}, short, highly focused \sphinxstylestrong{Daily Standup Meetings} are conducted where team members discuss daily progress. The PO can
\begin{itemize}
\item {} 
\sphinxAtStartPar
assess completed deliverables during the \sphinxstylestrong{Sprint}, and

\item {} 
\sphinxAtStartPar
accept the deliverables that meet the predefined Acceptance Criteria (AC).

\end{itemize}

\sphinxAtStartPar
Toward the end of the \sphinxstylestrong{Sprint}, a \sphinxstylestrong{Sprint Review Meeting} is held during which the PO and relevant business stakeholders are provided a demonstration of the deliverables.
The \sphinxstylestrong{Sprint} cycle ends with a \sphinxstylestrong{Retrospect Sprint Meeting} where the team members discuss ways they can improve their work and performance as they move forward into the subsequent \sphinxstylestrong{Sprint}.


\subsection{Scrum vs. Waterfall PM}
\label{\detokenize{APM/sbok:scrum-vs-waterfall-pm}}

\begin{savenotes}\sphinxattablestart
\sphinxthistablewithglobalstyle
\centering
\begin{tabulary}{\linewidth}[t]{TTT}
\sphinxtoprule
\sphinxstyletheadfamily 
\sphinxAtStartPar
Criterion
&\sphinxstyletheadfamily 
\sphinxAtStartPar
Scrum
&\sphinxstyletheadfamily 
\sphinxAtStartPar
Waterfall
\\
\sphinxmidrule
\sphinxtableatstartofbodyhook
\sphinxAtStartPar
Emphasis on
&
\sphinxAtStartPar
People
&
\sphinxAtStartPar
Processes
\\
\sphinxhline
\sphinxAtStartPar
Documentation
&
\sphinxAtStartPar
Minimal \sphinxhyphen{} only as required
&
\sphinxAtStartPar
Comprehensive
\\
\sphinxhline
\sphinxAtStartPar
Process style
&
\sphinxAtStartPar
Iterative
&
\sphinxAtStartPar
Linear
\\
\sphinxhline
\sphinxAtStartPar
Upfront planning
&
\sphinxAtStartPar
Low
&
\sphinxAtStartPar
High
\\
\sphinxhline
\sphinxAtStartPar
Prioritization of requirements
&
\sphinxAtStartPar
Base on business value and regularly updated
&
\sphinxAtStartPar
Fixed in the project plan
\\
\sphinxhline
\sphinxAtStartPar
Quality Assurance
&
\sphinxAtStartPar
Customer centric
&
\sphinxAtStartPar
Process centric
\\
\sphinxhline
\sphinxAtStartPar
Organization
&
\sphinxAtStartPar
Self\sphinxhyphen{}organized
&
\sphinxAtStartPar
Managed
\\
\sphinxhline
\sphinxAtStartPar
Management style
&
\sphinxAtStartPar
Decentralized
&
\sphinxAtStartPar
Centralized
\\
\sphinxhline
\sphinxAtStartPar
Change
&
\sphinxAtStartPar
Updates to Prioritized Product Backlog
&
\sphinxAtStartPar
Formal Change Management System
\\
\sphinxhline
\sphinxAtStartPar
Leadership
&
\sphinxAtStartPar
Supporting
&
\sphinxAtStartPar
Command and control
\\
\sphinxhline
\sphinxAtStartPar
Performance measurement
&
\sphinxAtStartPar
Business value
&
\sphinxAtStartPar
Plan conformity
\\
\sphinxhline
\sphinxAtStartPar
Return on Investment
&
\sphinxAtStartPar
Early/throughout project life cycle
&
\sphinxAtStartPar
End of project life cycle
\\
\sphinxhline
\sphinxAtStartPar
Customer involvement
&
\sphinxAtStartPar
High throughout the project
&
\sphinxAtStartPar
Varies depending on the project life cycle
\\
\sphinxbottomrule
\end{tabulary}
\sphinxtableafterendhook\par
\sphinxattableend\end{savenotes}


\subsection{Why Use Scrum?}
\label{\detokenize{APM/sbok:why-use-scrum}}
\sphinxAtStartPar
Some of the key benefits of using Scrum in any project are:


\begin{savenotes}\sphinxattablestart
\sphinxthistablewithglobalstyle
\centering
\begin{tabulary}{\linewidth}[t]{TT}
\sphinxtoprule
\sphinxstyletheadfamily 
\sphinxAtStartPar
Benefit
&\sphinxstyletheadfamily 
\sphinxAtStartPar
Description
\\
\sphinxmidrule
\sphinxtableatstartofbodyhook
\sphinxAtStartPar
Adaptability
&
\sphinxAtStartPar
Empirical process control and iterative delivery make projects adaptable and open to incorporating change
\\
\sphinxhline
\sphinxAtStartPar
Transparency
&
\sphinxAtStartPar
All information radiators like a \sphinxstylestrong{Scrumboard} and \sphinxstylestrong{Sprint Burndown Chart} are shared, leading to an open work environment
\\
\sphinxhline
\sphinxAtStartPar
Continuous Feedback
&
\sphinxAtStartPar
Continuous feedback is provided through the \sphinxstyleemphasis{Conduct Daily Standup} and \sphinxstyleemphasis{Demonstrate and Validate Sprint} processes
\\
\sphinxhline
\sphinxAtStartPar
Continuous Improvement
&
\sphinxAtStartPar
The deliverables are improved progressively Sprint by Sprint , through the \sphinxstyleemphasis{Refine Prioritized Product Backlog} process
\\
\sphinxhline
\sphinxAtStartPar
Continuous Value Delivery
&
\sphinxAtStartPar
Iterative processes enable the continuous delivery of value through the \sphinxstyleemphasis{Ship Deliverables} process as frequently as the customer requires
\\
\sphinxhline
\sphinxAtStartPar
Sustainable Pace
&
\sphinxAtStartPar
Scrum processes are designed such that the people involved can work at a sustainable pace that they can, in theory, continue indefinitely
\\
\sphinxhline
\sphinxAtStartPar
Early High Value Delivery
&
\sphinxAtStartPar
The \sphinxstyleemphasis{Create Prioritized Product Backlog} process ensures that the highest value requirements of the customer are satisfied first.
\\
\sphinxhline
\sphinxAtStartPar
Efficient Development Process
&
\sphinxAtStartPar
Time\sphinxhyphen{}boxing and minimizing non\sphinxhyphen{}essential work leads to higher efficiency levels.
\\
\sphinxhline
\sphinxAtStartPar
Motivation
&
\sphinxAtStartPar
The \sphinxstyleemphasis{Conduct Daily Standup} and \sphinxstyleemphasis{Retrospect Sprint} processes lead to greater levels of motivation among employees
\\
\sphinxhline
\sphinxAtStartPar
Faster Problem Resolution
&
\sphinxAtStartPar
Collaboration and colocation of cross\sphinxhyphen{}functional teams lead to faster problem\sphinxhyphen{}solving
\\
\sphinxhline
\sphinxAtStartPar
Effective Deliverables
&
\sphinxAtStartPar
The \sphinxstyleemphasis{Create Prioritized Product Backlog} process and regular reviews after creating deliverables ensure effective deliverables to the customer
\\
\sphinxhline
\sphinxAtStartPar
Customer Centric
&
\sphinxAtStartPar
Emphasis on business value and having a collaborative approach to engage business stakeholders ensures a customer\sphinxhyphen{}oriented framework
\\
\sphinxhline
\sphinxAtStartPar
High Trust Environment
&
\sphinxAtStartPar
\sphinxstyleemphasis{Conduct Daily Standup} and \sphinxstyleemphasis{Retrospect Sprint} processes promote transparency and collaboration, leading to a high\sphinxhyphen{}trust work environment ensuring low friction among employees
\\
\sphinxhline
\sphinxAtStartPar
Collective Ownership
&
\sphinxAtStartPar
The \sphinxstyleemphasis{Commit User Stories} process allows team members to take ownership of the project and their work, leading to better quality
\\
\sphinxhline
\sphinxAtStartPar
High Velocity
&
\sphinxAtStartPar
A collaborative framework enables highly skilled cross\sphinxhyphen{}functional teams to achieve their full potential and high velocity
\\
\sphinxhline
\sphinxAtStartPar
Innovative Environment
&
\sphinxAtStartPar
The \sphinxstyleemphasis{Retrospect Sprint} and \sphinxstyleemphasis{Retrospect Release} processes create an environment of introspection, learning, and adaptability leading to an innovative and creative work environment
\\
\sphinxbottomrule
\end{tabulary}
\sphinxtableafterendhook\par
\sphinxattableend\end{savenotes}


\subsubsection{Scalability}
\label{\detokenize{APM/sbok:scalability}}
\sphinxAtStartPar
To be effective, DT should ideally have 6 to 10 members. This practice may be the reason for the misconception that the Scrum framework can only be applied to small projects. However, the framework can easily be scaled for effective use in large projects, programs, and portfolios. In situations where the DT size exceeds ten people, multiple Development Teams can be formed to work on the project.

\sphinxAtStartPar
The logical approach of the guidelines and principles in this framework can be used to manage projects of any size, spanning geographies and organizations. Large projects may have multiple DTs working in parallel, making it necessary to synchronize and facilitate the flow of information and enhance communication. Large or complex projects are often implemented as a program or portfolio.


\subsection{Framework}
\label{\detokenize{APM/sbok:framework}}
\sphinxAtStartPar
The Scrum framework consists of three areas:
\begin{itemize}
\item {} 
\sphinxAtStartPar
principles,

\item {} 
\sphinxAtStartPar
aspects, and

\item {} 
\sphinxAtStartPar
processes.

\end{itemize}


\begin{savenotes}\sphinxattablestart
\sphinxthistablewithglobalstyle
\centering
\begin{tabulary}{\linewidth}[t]{T}
\sphinxtoprule
\sphinxstyletheadfamily 
\sphinxAtStartPar
\sphinxincludegraphics{{sbok-2}.svg}
\\
\sphinxmidrule
\sphinxtableatstartofbodyhook
\sphinxAtStartPar
Figure 2 — Scrum Framework
\\
\sphinxbottomrule
\end{tabulary}
\sphinxtableafterendhook\par
\sphinxattableend\end{savenotes}


\subsubsection{Principles}
\label{\detokenize{APM/sbok:principles}}
\sphinxAtStartPar
Scrum principles are the core guidelines for applying the Scrum framework and should mandatorily be used in all Scrum projects.

\sphinxAtStartPar
The six Scrum principles presented are:


\begin{savenotes}\sphinxattablestart
\sphinxthistablewithglobalstyle
\centering
\begin{tabulary}{\linewidth}[t]{TT}
\sphinxtoprule
\sphinxstyletheadfamily 
\sphinxAtStartPar
Principle
&\sphinxstyletheadfamily 
\sphinxAtStartPar
Description
\\
\sphinxmidrule
\sphinxtableatstartofbodyhook
\sphinxAtStartPar
Empirical Process Control
&
\sphinxAtStartPar
Aids learning through experimentation, especially when the problem is not well defined or when there are no clear solutions
\\
\sphinxhline
\sphinxAtStartPar
Self\sphinxhyphen{}organization
&
\sphinxAtStartPar
Today’s workers deliver significantly greater value when self\sphinxhyphen{}organized, resulting in better team buy\sphinxhyphen{}in, shared ownership, and an innovative and creative environment that is more conducive to growth
\\
\sphinxhline
\sphinxAtStartPar
Collaboration
&
\sphinxAtStartPar
Awareness, articulation, and appropriation. Project delivery is a shared value\sphinxhyphen{}creation process with teams working and interacting together, as well as with the customer and other business stakeholders, to deliver the greatest value
\\
\sphinxhline
\sphinxAtStartPar
Value\sphinxhyphen{}based Prioritization
&
\sphinxAtStartPar
Focus on delivering maximum business value, from early in the project and continuing throughout
\\
\sphinxhline
\sphinxAtStartPar
Time\sphinxhyphen{}boxing
&
\sphinxAtStartPar
Time is considered a limiting constraint in Scrum and is used to help effectively manage project planning and execution.
\\
\sphinxhline
\sphinxAtStartPar
Iterative Development
&
\sphinxAtStartPar
How to better manage changes and build products that satisfy customer needs
\\
\sphinxbottomrule
\end{tabulary}
\sphinxtableafterendhook\par
\sphinxattableend\end{savenotes}


\subsubsection{Aspects}
\label{\detokenize{APM/sbok:aspects}}\begin{itemize}
\item {} 
\sphinxAtStartPar
Organization
\begin{itemize}
\item {} 
\sphinxAtStartPar
Core Roles
\begin{itemize}
\item {} 
\sphinxAtStartPar
\sphinxstylestrong{Product Owner}: the person responsible for achieving maximum business value for the project, articulating customer requirements and maintaining business justification for the project.

\item {} 
\sphinxAtStartPar
\sphinxstylestrong{Scrum Master}: facilitator who ensures that the Development Team is provided with an environment conducive to completing the project successfully, guides, facilitates, and teaches Scrum practices to everyone involved in the project; clears impediments for the team, ensures that Scrum processes are being followed.

\item {} 
\sphinxAtStartPar
\sphinxstylestrong{Development Team}: group or team responsible for understanding the requirements specified by the Product Owner and creating the project deliverables.

\end{itemize}

\item {} 
\sphinxAtStartPar
Non\sphinxhyphen{}core Roles
\begin{itemize}
\item {} 
\sphinxAtStartPar
\sphinxstylestrong{Business Stakeholder(s)}

\item {} 
\sphinxAtStartPar
\sphinxstylestrong{Scrum Guidance Body}

\item {} 
\sphinxAtStartPar
\sphinxstylestrong{Vendors}

\end{itemize}

\end{itemize}

\item {} 
\sphinxAtStartPar
Business Justification
\begin{itemize}
\item {} 
\sphinxAtStartPar
Proper business assessment before starting any project helps key decision makers understand the business need for a change or a new product or service, the justification for moving forward with a project, and its viability

\item {} 
\sphinxAtStartPar
It is impossible to guarantee project success at completion, irrespective of the size or complexity of a project \(\rightarrow\) Considering this uncertainty of achieving success, Scrum attempts to start delivering results as early in the project as possible.

\item {} 
\sphinxAtStartPar
This early delivery of results, and thereby value, provides an opportunity for reinvestment and proves the worth of the project to interested business stakeholders.

\end{itemize}

\item {} 
\sphinxAtStartPar
Quality
\begin{itemize}
\item {} 
\sphinxAtStartPar
Quality is the ability of the completed product or deliverables to meet the Acceptance Criteria and achieve the business value expected by the customer.

\item {} 
\sphinxAtStartPar
To ensure a project meets quality requirements, Scrum adopts an approach of continuous improvement whereby
the team learns from experience and stakeholder engagement to constantly keep the Prioritized Product Backlog
updated with any changes in requirements

\item {} 
\sphinxAtStartPar
The Prioritized Product Backlog is simply never complete until the closure or termination of the project

\item {} 
\sphinxAtStartPar
Any changes to the requirements reflect changes in the internal and external business environment and allow the team to continually work and adapt to achieve those requirements

\end{itemize}

\item {} 
\sphinxAtStartPar
Change
\begin{itemize}
\item {} 
\sphinxAtStartPar
Every project, regardless of its method or framework used, is exposed to change.

\item {} 
\sphinxAtStartPar
It is imperative that project team members understand that the Scrum development processes are designed to embrace change.

\item {} 
\sphinxAtStartPar
Organizations should try to maximize the benefits that arise from change and minimize any negative impacts
through diligent change management processes by the principles of Scrum.

\item {} 
\sphinxAtStartPar
A primary principle of Scrum is its acknowledgment that business stakeholders (e.g., customers, users, and sponsors) change their minds about what they want and need throughout a project (sometimes referred to as requirements churn), and it is difficult, if not impossible, for business stakeholders to define all requirements during project initiation.

\item {} 
\sphinxAtStartPar
Scrum projects welcome change using short, iterative Sprints that incorporate customer feedback on each Sprint’s deliverables.

\item {} 
\sphinxAtStartPar
This enables the customer to regularly interact with the Development Team members, view deliverables as they are ready, and change requirements if needed earlier in the Sprint.

\item {} 
\sphinxAtStartPar
Also, the portfolio or program management teams can respond to Change Requests about Scrum projects
applicable at their level.

\end{itemize}

\item {} 
\sphinxAtStartPar
Risk
\begin{itemize}
\item {} 
\sphinxAtStartPar
Risk is defined as an uncertain event or set of events that can affect the objectives of a project and may
contribute to its success or failure.

\item {} 
\sphinxAtStartPar
Risks that are likely to positively impact the project are referred to as opportunities, whereas threats are risks that could affect the project negatively.

\item {} 
\sphinxAtStartPar
Managing risk must be done proactively, and it is an iterative process that should begin at project initiation and continue throughout the project’s lifecycle.

\item {} 
\sphinxAtStartPar
The process of managing risks should follow some standardized steps to ensure that risks are identified and evaluated and a proper course of action is determined and acted upon accordingly.

\item {} 
\sphinxAtStartPar
Risks should be identified, assessed, and responded to based on two factors — the probability of each risk’s occurrence and the possible impact in the event of such occurrence.

\item {} 
\sphinxAtStartPar
Risks with a high probability and impact value (determined by multiplying both factors) should be addressed before those with a relatively lower value. In general, once a risk is identified, it is important to understand the risk with regards to the probable causes and the potential effects if the risk occurs.

\end{itemize}

\end{itemize}


\subsubsection{Processes}
\label{\detokenize{APM/sbok:processes}}

\begin{savenotes}\sphinxattablestart
\sphinxthistablewithglobalstyle
\centering
\begin{tabulary}{\linewidth}[t]{TT}
\sphinxtoprule
\sphinxstyletheadfamily 
\sphinxAtStartPar
Phase
&\sphinxstyletheadfamily 
\sphinxAtStartPar
Fundamental Scrum Processes
\\
\sphinxmidrule
\sphinxtableatstartofbodyhook
\sphinxAtStartPar
Initiate
&
\sphinxAtStartPar
\sphinxstyleemphasis{Create Project Vision}
\\
\sphinxhline
\sphinxAtStartPar

&
\sphinxAtStartPar
\sphinxstyleemphasis{Identify Scrum Master and Business Stakeholder(s)}
\\
\sphinxhline
\sphinxAtStartPar

&
\sphinxAtStartPar
\sphinxstyleemphasis{Form Development Team}
\\
\sphinxhline
\sphinxAtStartPar

&
\sphinxAtStartPar
\sphinxstyleemphasis{Develop Epic(s)}
\\
\sphinxhline
\sphinxAtStartPar

&
\sphinxAtStartPar
\sphinxstyleemphasis{Create Prioritized Product Backlog}
\\
\sphinxhline
\sphinxAtStartPar

&
\sphinxAtStartPar
\sphinxstyleemphasis{Conduct Release Planning}
\\
\sphinxhline
\sphinxAtStartPar
Plan and Estimate
&
\sphinxAtStartPar
\sphinxstyleemphasis{Create User Stories}
\\
\sphinxhline
\sphinxAtStartPar

&
\sphinxAtStartPar
\sphinxstyleemphasis{Estimate User Stories}
\\
\sphinxhline
\sphinxAtStartPar

&
\sphinxAtStartPar
\sphinxstyleemphasis{Commit User Stories}
\\
\sphinxhline
\sphinxAtStartPar

&
\sphinxAtStartPar
\sphinxstyleemphasis{Identify Tasks}
\\
\sphinxhline
\sphinxAtStartPar

&
\sphinxAtStartPar
\sphinxstyleemphasis{Estimate Tasks}
\\
\sphinxhline
\sphinxAtStartPar

&
\sphinxAtStartPar
\sphinxstyleemphasis{Update Sprint Backlog}
\\
\sphinxhline
\sphinxAtStartPar
Implement
&
\sphinxAtStartPar
\sphinxstyleemphasis{Create Deliverables}
\\
\sphinxhline
\sphinxAtStartPar

&
\sphinxAtStartPar
\sphinxstyleemphasis{Conduct Daily Standup}
\\
\sphinxhline
\sphinxAtStartPar

&
\sphinxAtStartPar
\sphinxstyleemphasis{Refine Prioritized Product Backlog}
\\
\sphinxhline
\sphinxAtStartPar
Review and Retrospect
&
\sphinxAtStartPar
\sphinxstyleemphasis{Demonstrate and Validate Sprint}
\\
\sphinxhline
\sphinxAtStartPar

&
\sphinxAtStartPar
\sphinxstyleemphasis{Retrospect Sprint}
\\
\sphinxhline
\sphinxAtStartPar
Release
&
\sphinxAtStartPar
\sphinxstyleemphasis{Ship Deliverables}
\\
\sphinxhline
\sphinxAtStartPar

&
\sphinxAtStartPar
\sphinxstyleemphasis{Retrospect Release}
\\
\sphinxbottomrule
\end{tabulary}
\sphinxtableafterendhook\par
\sphinxattableend\end{savenotes}


\paragraph{Initiate}
\label{\detokenize{APM/sbok:initiate}}
\sphinxAtStartPar
The processes relevant to the \sphinxstylestrong{Initiate} phase are as follows:
\begin{enumerate}
\sphinxsetlistlabels{\arabic}{enumi}{enumii}{}{.}%
\item {} 
\sphinxAtStartPar
\sphinxstylestrong{Create Project Vision}

\end{enumerate}
\begin{itemize}
\item {} 
\sphinxAtStartPar
The project business case is reviewed to create a \sphinxstylestrong{Project Vision Statement} that will serve as the inspiration and provide a focus for the entire project

\item {} 
\sphinxAtStartPar
The PO is identified

\end{itemize}
\begin{enumerate}
\sphinxsetlistlabels{\arabic}{enumi}{enumii}{}{.}%
\setcounter{enumi}{1}
\item {} 
\sphinxAtStartPar
\sphinxstylestrong{Identify Scrum Master and Business Stakeholder(s)}: The SM and business stakeholders are identified using specific selection criteria.

\item {} 
\sphinxAtStartPar
\sphinxstylestrong{Form Development Team}: DT members are identified

\item {} 
\sphinxAtStartPar
\sphinxstylestrong{Develop Epic(s)}

\end{enumerate}
\begin{itemize}
\item {} 
\sphinxAtStartPar
The \sphinxstylestrong{Project Vision Statement} serves as the basis for developing Epics

\item {} 
\sphinxAtStartPar
\sphinxstylestrong{User Group Meetings} may be held to discuss appropriate Epics

\end{itemize}
\begin{enumerate}
\sphinxsetlistlabels{\arabic}{enumi}{enumii}{}{.}%
\setcounter{enumi}{4}
\item {} 
\sphinxAtStartPar
\sphinxstylestrong{Create Prioritized Product Backlog}

\end{enumerate}
\begin{itemize}
\item {} 
\sphinxAtStartPar
Epic(s) are refined, elaborated, and then prioritized to create a Prioritized Product Backlog for the project.

\item {} 
\sphinxAtStartPar
The \sphinxstylestrong{Done Criteria} is also established

\end{itemize}
\begin{enumerate}
\sphinxsetlistlabels{\arabic}{enumi}{enumii}{}{.}%
\setcounter{enumi}{5}
\item {} 
\sphinxAtStartPar
\sphinxstylestrong{Conduct Release Planning}

\end{enumerate}
\begin{itemize}
\item {} 
\sphinxAtStartPar
With support from the Development Team, the PO develops a \sphinxstyleemphasis{\sphinxstylestrong{Release Planning Schedule}}, which is essentially a phased deployment schedule that can be shared with the project’s business stakeholders.

\item {} 
\sphinxAtStartPar
\sphinxstylestrong{Length of Sprint} is also determined

\end{itemize}


\paragraph{Plan and Estimate}
\label{\detokenize{APM/sbok:plan-and-estimate}}
\sphinxAtStartPar
The processes relevant to the Plan and Estimate phase are as follows:
\begin{enumerate}
\sphinxsetlistlabels{\arabic}{enumi}{enumii}{}{.}%
\item {} 
\sphinxAtStartPar
\sphinxstylestrong{Create User Stories}

\end{enumerate}
\begin{itemize}
\item {} 
\sphinxAtStartPar
User Stories and their related Acceptance Criteria are created by the Product Owner and incorporated into the Prioritized Product Backlog

\item {} 
\sphinxAtStartPar
User Stories are designed to ensure that the customer’s requirements are clearly depicted and can be fully understood by all business stakeholders

\end{itemize}
\begin{enumerate}
\sphinxsetlistlabels{\arabic}{enumi}{enumii}{}{.}%
\setcounter{enumi}{1}
\item {} 
\sphinxAtStartPar
\sphinxstylestrong{Estimate User Stories}: the Development Team, supported by the Scrum Master, estimates the User Stories and identifies the effort required to develop the functionality described in each User Story

\item {} 
\sphinxAtStartPar
\sphinxstylestrong{Commit User Stories}

\end{enumerate}
\begin{itemize}
\item {} 
\sphinxAtStartPar
The Development Team commits to delivering Product Owner\sphinxhyphen{}approved User Stories for a Sprint.

\item {} 
\sphinxAtStartPar
The results of this process are the committed

\item {} 
\sphinxAtStartPar
User Stories , and

\item {} 
\sphinxAtStartPar
Sprint Backlog

\end{itemize}
\begin{enumerate}
\sphinxsetlistlabels{\arabic}{enumi}{enumii}{}{.}%
\setcounter{enumi}{3}
\item {} 
\sphinxAtStartPar
\sphinxstylestrong{Identify Tasks}—In this process, the committed User Stories are broken down into specific tasks and compiled into a Task List

\item {} 
\sphinxAtStartPar
\sphinxstylestrong{Estimate Tasks}: The Scrum Core Team estimates the effort required to accomplish each task in the Task List

\item {} 
\sphinxAtStartPar
\sphinxstylestrong{Update Sprint Backlog}; The Scrum Core Team updates the Sprint Backlog with further details about the tasks as part of the Sprint Planning Meeting

\end{enumerate}


\paragraph{Implement}
\label{\detokenize{APM/sbok:implement}}
\sphinxAtStartPar
The processes relevant to the Implement phase are as follows:
\begin{enumerate}
\sphinxsetlistlabels{\arabic}{enumi}{enumii}{}{.}%
\item {} 
\sphinxAtStartPar
\sphinxstylestrong{Create Deliverables}

\end{enumerate}
\begin{itemize}
\item {} 
\sphinxAtStartPar
The Development Team works on the tasks in the Sprint Backlog to create Sprint Deliverables

\item {} 
\sphinxAtStartPar
A Scrumboard is often used to track the work and activities

\item {} 
\sphinxAtStartPar
Issues or problems faced by the Development Team should be updated in an Impediment Log

\end{itemize}
\begin{enumerate}
\sphinxsetlistlabels{\arabic}{enumi}{enumii}{}{.}%
\setcounter{enumi}{1}
\item {} 
\sphinxAtStartPar
\sphinxstylestrong{Conduct Daily Standup}: every day, a highly focused, Time\sphinxhyphen{}boxed meeting, referred to as the Daily Standup Meeting , is conducted to update each other on their individual progress and any impediments they may be facing

\item {} 
\sphinxAtStartPar
\sphinxstylestrong{Refine Prioritized Product Backlog}

\end{enumerate}
\begin{itemize}
\item {} 
\sphinxAtStartPar
The Prioritized Product Backlog is continuously updated and maintained

\item {} 
\sphinxAtStartPar
A Prioritized Product Backlog Review Meeting is held, in which any changes or updates to the backlog are discussed and incorporated into the Prioritized Product Backlog as appropriate.

\end{itemize}


\paragraph{Review and Retrospect}
\label{\detokenize{APM/sbok:review-and-retrospect}}
\sphinxAtStartPar
The processes relevant to the Review and Retrospect phase are as follows:
\begin{enumerate}
\sphinxsetlistlabels{\arabic}{enumi}{enumii}{}{.}%
\item {} 
\sphinxAtStartPar
\sphinxstylestrong{Demonstrate and Validate Sprint}

\end{enumerate}
\begin{itemize}
\item {} 
\sphinxAtStartPar
The Development Team demonstrates the Sprint deliverables to the Product Owner and relevant business stakeholders in a Sprint Review Meeting

\item {} 
\sphinxAtStartPar
The purpose of this meeting is to secure approval and acceptance of the Sprint User Stories by the Product Owner

\end{itemize}
\begin{enumerate}
\sphinxsetlistlabels{\arabic}{enumi}{enumii}{}{.}%
\setcounter{enumi}{1}
\item {} 
\sphinxAtStartPar
\sphinxstylestrong{Retrospect Sprint}

\end{enumerate}
\begin{itemize}
\item {} 
\sphinxAtStartPar
The Scrum Master and Development Team meet to discuss the lessons learned throughout the Sprint

\item {} 
\sphinxAtStartPar
This information is documented and should be applied to future Sprints.

\item {} 
\sphinxAtStartPar
Often, as a result of this discussion, there may be agreed\sphinxhyphen{}on actionable improvements or updated Scrum Guidance Body recommendations.

\end{itemize}


\paragraph{Release}
\label{\detokenize{APM/sbok:release}}
\sphinxAtStartPar
The processes relevant to the Release phase are as follows:
\begin{enumerate}
\sphinxsetlistlabels{\arabic}{enumi}{enumii}{}{.}%
\item {} 
\sphinxAtStartPar
\sphinxstylestrong{Ship Deliverables}

\end{enumerate}
\begin{itemize}
\item {} 
\sphinxAtStartPar
All deliverables from the accepted User Stories of previously completed Sprints are delivered or transitioned to the relevant business stakeholders

\item {} 
\sphinxAtStartPar
A formal Working Deliverables Agreement documents the successful completion of the release

\end{itemize}
\begin{enumerate}
\sphinxsetlistlabels{\arabic}{enumi}{enumii}{}{.}%
\setcounter{enumi}{1}
\item {} 
\sphinxAtStartPar
\sphinxstylestrong{Retrospect Release}

\end{enumerate}
\begin{itemize}
\item {} 
\sphinxAtStartPar
Business stakeholders and Scrum Core Team members assemble to reflect on the release and identify, document and internalize the lessons learned.

\item {} 
\sphinxAtStartPar
Often, these lessons lead to the documentation of agreed actionable improvements to be implemented in future projects.

\end{itemize}

\sphinxstepscope


\part{Systems Engineering}

\sphinxstepscope


\chapter{SEBoK}
\label{\detokenize{SE/sebok:sebok}}\label{\detokenize{SE/sebok::doc}}
\sphinxAtStartPar
Notes from Guide to the Systems Engineering Body of Knowledge (SEBok), version 2.9


\section{Part 3: Systems Engineering and Management (SE\&M)}
\label{\detokenize{SE/sebok:part-3-systems-engineering-and-management-se-m}}

\subsection{Systems Engineering and Management}
\label{\detokenize{SE/sebok:systems-engineering-and-management}}
\sphinxAtStartPar
Systems Engineering and Management (SE\&M) articles provide system lifecycle best practices for defining and executing interdisciplinary processes to ensure that customer needs are satisfied with a technical performance, schedule, and cost compliant solution.


\subsection{SE\&M Knowledge Areas}
\label{\detokenize{SE/sebok:se-m-knowledge-areas}}
\sphinxAtStartPar
The SE\&M articles are organized into the following Knowledge Areas (KAs).
\begin{itemize}
\item {} 
\sphinxAtStartPar
Systems Engineering STEM Overview

\item {} 
\sphinxAtStartPar
Model\sphinxhyphen{}Based Systems Engineering (MBSE)

\item {} 
\sphinxAtStartPar
Systems Life Cycle Approaches

\item {} 
\sphinxAtStartPar
System Life Cycle Models

\item {} 
\sphinxAtStartPar
Systems Engineering Management

\item {} 
\sphinxAtStartPar
Business and Mission Analysis

\item {} 
\sphinxAtStartPar
Stakeholder Needs Definition

\item {} 
\sphinxAtStartPar
System Architecture Definition

\item {} 
\sphinxAtStartPar
Detailed Design Definition

\item {} 
\sphinxAtStartPar
System Analysis

\item {} 
\sphinxAtStartPar
System Realization

\item {} 
\sphinxAtStartPar
System Implementation

\item {} 
\sphinxAtStartPar
System Integration

\item {} 
\sphinxAtStartPar
System Verification

\item {} 
\sphinxAtStartPar
System Transition

\item {} 
\sphinxAtStartPar
System Validation

\item {} 
\sphinxAtStartPar
System Operation

\item {} 
\sphinxAtStartPar
System Maintenance

\item {} 
\sphinxAtStartPar
System Specialty Engineering

\item {} 
\sphinxAtStartPar
Logistics

\item {} 
\sphinxAtStartPar
Service Life Management

\item {} 
\sphinxAtStartPar
Systems Engineering Standards

\end{itemize}


\subsection{Systems Engineering \& Management Overview}
\label{\detokenize{SE/sebok:systems-engineering-management-overview}}
\sphinxAtStartPar
The \sphinxstylestrong{role} of Systems Engineering (SE) is to define system, constraints, allocations, behavior and structure characteristics to satisfy customer needs.

\sphinxAtStartPar
The \sphinxstylestrong{system} is defined in terms of
\begin{itemize}
\item {} 
\sphinxAtStartPar
hierarchical structural elements, and

\item {} 
\sphinxAtStartPar
their behavior interactions.

\end{itemize}

\sphinxAtStartPar
The \sphinxstylestrong{interactions} include the exchange of data, energy, force, or mass which modifies the state of the cooperating elements resulting in emergent, discrete, or continuous \sphinxstylestrong{behaviors}.

\sphinxAtStartPar
The \sphinxstylestrong{behaviors} are at sequential levels of aggregation (bottoms\sphinxhyphen{}up) or decomposition (top\sphinxhyphen{}down) to satisfy requirements, constraints, and allocations.

\sphinxAtStartPar
SE collaborates within an integrated product \sphinxstylestrong{team} with electrical, mechanical, software, and specialty engineering to define the subsystem and component detailed design implementations to develop a holistic technical solution.


\subsection{Model\sphinxhyphen{}Based Systems Engineering (MBSE)}
\label{\detokenize{SE/sebok:model-based-systems-engineering-mbse}}
\sphinxAtStartPar
Model\sphinxhyphen{}based Systems Engineering (MBSE)
\begin{itemize}
\item {} 
\sphinxAtStartPar
is a paradigm that uses formalized representations of systems, known as models, to support and facilitate the performance of SE tasks throughout a system’s life cycle.

\item {} 
\sphinxAtStartPar
is frequently contrasted with legacy document\sphinxhyphen{}based approaches where systems engineering captures system design information via multiple independent documents in various non\sphinxhyphen{}standardized formats.

\item {} 
\sphinxAtStartPar
consolidates of system information in system design models, which provide primary SE artifacts.

\end{itemize}

\sphinxAtStartPar
These system models, which are generally expressed in a standardized modelling language such as Systems Modeling Language (SysML®) express key system information in a concise, consistent, correct, and coherent format.

\sphinxAtStartPar
When implemented properly, MBSE models permit the standardized consolidation and integration of system knowledge across engineering disciplines and subsystems and streamline key systems engineering tasks while also minimizing developmental risk.


\subsubsection{System Models}
\label{\detokenize{SE/sebok:system-models}}

\paragraph{Definition of a model}
\label{\detokenize{SE/sebok:definition-of-a-model}}
\sphinxAtStartPar
Models
\begin{itemize}
\item {} 
\sphinxAtStartPar
are representations that are used to capture, analyze, and/or communicate information about a system or concept.

\item {} 
\sphinxAtStartPar
can vary in scope, purpose, and type, and can be utilized both individually as stand\sphinxhyphen{}alone entities as well as in concert with each other as part of an integrated set.

\end{itemize}


\paragraph{Model Properties}
\label{\detokenize{SE/sebok:model-properties}}
\sphinxAtStartPar
A model can be described and classified with respect to the following properties:
\begin{itemize}
\item {} 
\sphinxAtStartPar
Scope

\item {} 
\sphinxAtStartPar
Domain

\item {} 
\sphinxAtStartPar
Formality

\item {} 
\sphinxAtStartPar
Abstraction

\item {} 
\sphinxAtStartPar
Physical/conceptual

\item {} 
\sphinxAtStartPar
Descriptive/analytical

\item {} 
\sphinxAtStartPar
Fidelity

\item {} 
\sphinxAtStartPar
Completeness

\item {} 
\sphinxAtStartPar
Integration

\item {} 
\sphinxAtStartPar
Quality

\end{itemize}


\paragraph{Criteria for Effective MBSE Models}
\label{\detokenize{SE/sebok:criteria-for-effective-mbse-models}}
\sphinxAtStartPar
While a successful MBSE workflow can involve the use of several different interconnected or standalone models of various scopes and types based on user needs, the main system model in an MBSE projects generally should have
the following characteristics:
\begin{enumerate}
\sphinxsetlistlabels{\arabic}{enumi}{enumii}{}{.}%
\item {} 
\sphinxAtStartPar
A scope which matches the scope of the project (i.e., it should encompass the entire SoI);

\item {} 
\sphinxAtStartPar
Representative of a holistic perspective from all relevant domains.

\item {} 
\sphinxAtStartPar
Strict compliance with a previously established standardized modeling language, whether that be an existing language such as SysML® or a custom formalism.

\item {} 
\sphinxAtStartPar
Fully abstracted, to only include relevant information appropriate for the SoI and its desired use\sphinxhyphen{}case(s).

\item {} 
\sphinxAtStartPar
Conceptual in nature, to permit the capture of intangible information (e.g., system requirements)

\item {} 
\sphinxAtStartPar
Containing a description of the system functional and structural architecture at minimum and supplemented by
integrated analytical/quantitative property descriptions as needed.

\item {} 
\sphinxAtStartPar
Demonstrating sufficient fidelity to capture relevant system elements and behavior.

\item {} 
\sphinxAtStartPar
Fully complete given its scope.

\item {} 
\sphinxAtStartPar
Integrated with any necessary auxiliary models.

\item {} 
\sphinxAtStartPar
Sufficiently high\sphinxhyphen{}quality as to meet the needs of those designing, developing, or otherwise working on the system.
In terms of content, effective system models are expected to capture key system information regarding requirements, system functionality/behavior, structure/form, properties, and interconnections between system components.

\end{enumerate}


\paragraph{Digital Twins}
\label{\detokenize{SE/sebok:digital-twins}}
\sphinxAtStartPar
When MBSE models of physical systems are built with sufficient completeness and fidelity, it is possible for them to function as “digital twins” of the systems they represent.

\sphinxAtStartPar
Digital twins provide a means of accurately representing a system’s form and function throughout the system’s lifecycle, all within a digital environment.

\sphinxAtStartPar
Creating such digital twins allow
\begin{itemize}
\item {} 
\sphinxAtStartPar
testing, analysis, and optimization of systems in a virtual environment at
\begin{itemize}
\item {} 
\sphinxAtStartPar
no risk to the actual system of interest, and

\item {} 
\sphinxAtStartPar
a greatly reduced cost/burden.

\end{itemize}

\item {} 
\sphinxAtStartPar
representing the behavior of systems under conditions which would be impractical or impossible to induce under experimental conditions, thereby making it possible to obtain information not obtainable via study of the original physical system.

\end{itemize}


\section{Knowledge Area: Systems Life Cycle Approaches}
\label{\detokenize{SE/sebok:knowledge-area-systems-life-cycle-approaches}}

\subsection{Systems Life Cycle Approaches}
\label{\detokenize{SE/sebok:systems-life-cycle-approaches}}
\sphinxAtStartPar
Key principles:
\begin{itemize}
\item {} 
\sphinxAtStartPar
life cycle,

\item {} 
\sphinxAtStartPar
life cycle model, and

\item {} 
\sphinxAtStartPar
life cycle processes.

\end{itemize}

\sphinxAtStartPar
A generic SE paradigm is described; this forms a starting point for discussions of more detailed life cycle knowledge.


\subsubsection{Topics}
\label{\detokenize{SE/sebok:topics}}
\sphinxAtStartPar
This KA contains the following topics:
\begin{itemize}
\item {} 
\sphinxAtStartPar
Generic Life Cycle Model

\item {} 
\sphinxAtStartPar
Applying Life Cycle Processes

\item {} 
\sphinxAtStartPar
Life Cycle Processes and Enterprise Need

\end{itemize}


\subsubsection{Life Cycle Terminology}
\label{\detokenize{SE/sebok:life-cycle-terminology}}
\sphinxAtStartPar
The term “life cycle” is used to describe
\begin{itemize}
\item {} 
\sphinxAtStartPar
the complete life of an instance of a system\sphinxhyphen{}of\sphinxhyphen{}interest (SoI), and

\item {} 
\sphinxAtStartPar
the managed combination of multiple such instances to provide capabilities which deliver stakeholder satisfaction.

\end{itemize}

\sphinxAtStartPar
A life cycle model:
\begin{itemize}
\item {} 
\sphinxAtStartPar
identifies the major stages that a specific SoI goes through, from its inception to its retirement.

\item {} 
\sphinxAtStartPar
is generally implemented in development projects and are strongly aligned with management planning and decision making.

\end{itemize}


\subsubsection{Generic Systems Engineering Paradigm}
\label{\detokenize{SE/sebok:generic-systems-engineering-paradigm}}
\sphinxAtStartPar
Overall goals of any SE effort:
\begin{itemize}
\item {} 
\sphinxAtStartPar
understanding of stakeholder value,

\item {} 
\sphinxAtStartPar
selection of a specific need to be addressed,

\item {} 
\sphinxAtStartPar
transformation of that need into a system (the product or service that provides for the need), and

\item {} 
\sphinxAtStartPar
use of that product or service to provide the stakeholder value.

\end{itemize}

\sphinxAtStartPar
SoI’s identified in the formation of a System Breakdown Structure (SBS).
SoI 1 is broken down into its basic elements, which in this case are systems as well (SoI 2 and SoI 3).
These two systems are composed of system elements that are not refined any further.


\subsection{Generic Life Cycle Model}
\label{\detokenize{SE/sebok:generic-life-cycle-model}}
\sphinxAtStartPar
Each SoI has an associated LC model.

\sphinxAtStartPar
The generic LC model below applies to a single SoI.

\sphinxAtStartPar
SE must generally be synchronized across a number of tailored instances of such LC models to fully satisfy stakeholder needs.


\begin{savenotes}\sphinxattablestart
\sphinxthistablewithglobalstyle
\centering
\begin{tabulary}{\linewidth}[t]{T}
\sphinxtoprule
\sphinxstyletheadfamily 
\sphinxAtStartPar
\sphinxincludegraphics{{sebok-SBS}.svg}
\\
\sphinxmidrule
\sphinxtableatstartofbodyhook
\sphinxAtStartPar
System Breakdown Structure
\\
\sphinxbottomrule
\end{tabulary}
\sphinxtableafterendhook\par
\sphinxattableend\end{savenotes}


\begin{savenotes}\sphinxattablestart
\sphinxthistablewithglobalstyle
\centering
\begin{tabulary}{\linewidth}[t]{T}
\sphinxtoprule
\sphinxstyletheadfamily 
\sphinxAtStartPar
\sphinxincludegraphics{{sebok-SOI_LC}.svg}
\\
\sphinxmidrule
\sphinxtableatstartofbodyhook
\sphinxAtStartPar
SoI LC/Processes
\\
\sphinxbottomrule
\end{tabulary}
\sphinxtableafterendhook\par
\sphinxattableend\end{savenotes}


\subsubsection{A Generic System Life Cycle Model}
\label{\detokenize{SE/sebok:a-generic-system-life-cycle-model}}
\sphinxAtStartPar
There is no single “one\sphinxhyphen{}size\sphinxhyphen{}fits\sphinxhyphen{}all” system LC model that can provide specific guidance for all project situations.

\sphinxAtStartPar
The model is defined as a set of stages, within which technical and management
activities are performed.

\sphinxAtStartPar
The stages are terminated by decision gates, where the key stakeholders decide whether
\begin{itemize}
\item {} 
\sphinxAtStartPar
to proceed into the next stage,

\item {} 
\sphinxAtStartPar
to remain in the current stage, or

\item {} 
\sphinxAtStartPar
to terminate or re\sphinxhyphen{}scope related projects.

\end{itemize}

\sphinxAtStartPar
Stages:
\begin{enumerate}
\sphinxsetlistlabels{\arabic}{enumi}{enumii}{}{.}%
\item {} 
\sphinxAtStartPar
Definition
\begin{itemize}
\item {} 
\sphinxAtStartPar
Concept Definition

\item {} 
\sphinxAtStartPar
System Definition

\end{itemize}

\item {} 
\sphinxAtStartPar
System Realization

\item {} 
\sphinxAtStartPar
System Production, Support, and Utilization (PSU)
\begin{itemize}
\item {} 
\sphinxAtStartPar
System Production

\item {} 
\sphinxAtStartPar
System Support

\item {} 
\sphinxAtStartPar
System Utilization

\end{itemize}

\item {} 
\sphinxAtStartPar
System Retirement

\end{enumerate}


\subsection{Applying Life Cycle Processes}
\label{\detokenize{SE/sebok:applying-life-cycle-processes}}
\sphinxAtStartPar
The Generic Life Cycle Model describes a set of life cycle stages and their relationships.

\sphinxAtStartPar
In defining this we described some of the technical and management activities critical to the success of each stage.

\sphinxAtStartPar
While this association of activity to stage is important, we must also recognize the through life relationships between these activities to ensure we take a systems approach.

\sphinxAtStartPar
SE technical and management activities are defined in a set of life cycle processes.

\sphinxAtStartPar
These group together closely related activities and allow us to describe the relationships between them.

\sphinxAtStartPar
In this topic, we discuss a number of views on the nature of the inter\sphinxhyphen{}relationships between process activities within a life cycle model.

\sphinxAtStartPar
In general, the technical and management activities are applied in accordance with the principles of concurrency, iteration and recursion described in the generic systems engineering paradigm.

\sphinxAtStartPar
These principles overlap to some extent and can be seen as related views of the same fundamental need to ensure we can take a holistic systems approach, while allowing for some structuring and sequence of our activities.

\sphinxAtStartPar
The views presented below should be seen as examples of the ways in which different SE authors present these overlapping ideas.


\subsubsection{Life Cycle Process Terminology}
\label{\detokenize{SE/sebok:life-cycle-process-terminology}}

\paragraph{Process}
\label{\detokenize{SE/sebok:process}}\begin{itemize}
\item {} 
\sphinxAtStartPar
Is a series of actions or steps taken in order to achieve a particular end, and

\item {} 
\sphinxAtStartPar
Can be performed by humans or machines transforming inputs into outputs.

\item {} 
\sphinxAtStartPar
Are interpreted in several ways, including
\begin{itemize}
\item {} 
\sphinxAtStartPar
technical,

\item {} 
\sphinxAtStartPar
LC,

\item {} 
\sphinxAtStartPar
business, or

\item {} 
\sphinxAtStartPar
manufacturing flow processes.

\end{itemize}

\end{itemize}


\paragraph{Requirement}
\label{\detokenize{SE/sebok:requirement}}\begin{itemize}
\item {} 
\sphinxAtStartPar
Are something that are needed/wanted but may not be compulsory in all circumstances,

\item {} 
\sphinxAtStartPar
May refer to product/process characteristics/constraints.

\item {} 
\sphinxAtStartPar
Different understandings of requirements are dependent upon
\begin{itemize}
\item {} 
\sphinxAtStartPar
process state,

\item {} 
\sphinxAtStartPar
level of abstraction,

\item {} 
\sphinxAtStartPar
and type (e.g. functional, performance, constraint).

\end{itemize}

\item {} 
\sphinxAtStartPar
May have multiple interpretations over time.

\item {} 
\sphinxAtStartPar
Exist at multiple levels of enterprise/systems with multiple levels of abstraction, ranging from
\begin{itemize}
\item {} 
\sphinxAtStartPar
highest level of the enterprise capability/customer need to

\item {} 
\sphinxAtStartPar
lowest level of the system design.

\end{itemize}

\item {} 
\sphinxAtStartPar
Need to be defined at the appropriate level of detail for the level of the entity to which they apply.

\end{itemize}


\paragraph{Architecture}
\label{\detokenize{SE/sebok:architecture}}\begin{itemize}
\item {} 
\sphinxAtStartPar
Organizational structure of a system, whereby the system can be defined in different contexts.

\item {} 
\sphinxAtStartPar
Is the art or practice of designing the structures.

\item {} 
\sphinxAtStartPar
Can apply for a system product, enterprise, or service.

\item {} 
\sphinxAtStartPar
Closely related to framework, as they are ways of representing architectures.

\end{itemize}


\subsubsection{Life Cycle Process Concurrency}
\label{\detokenize{SE/sebok:life-cycle-process-concurrency}}
\sphinxAtStartPar
In the Generic LC Model, the execution of process activities is not compartmentalized to particular LC stages.


\begin{savenotes}\sphinxattablestart
\sphinxthistablewithglobalstyle
\centering
\begin{tabulary}{\linewidth}[t]{T}
\sphinxtoprule
\sphinxstyletheadfamily 
\sphinxAtStartPar
\sphinxincludegraphics{{sebok-RUP}.svg}
\\
\sphinxmidrule
\sphinxtableatstartofbodyhook
\sphinxAtStartPar
RUP Hump
\\
\sphinxbottomrule
\end{tabulary}
\sphinxtableafterendhook\par
\sphinxattableend\end{savenotes}

\sphinxAtStartPar
The lines on this diagram represent the amount of activity for each process over the generic life cycle.
The peaks (or humps) of activity represent the periods when a process activity becomes the main focus of a stage.
The activity before and after these peaks may represent through life issues raised by a process focus, e.g. how likely maintenance
constraints will be represented in the system requirements.
These considerations help maintain a more holistic perspective in each stage, or they can represent forward planning to ensure the resources needed to complete future activities have been included in estimates and plans, e.g. all resources needed for verification are in place or available.
Ensuring this hump diagram principle is implemented in a way which is achievable, affordable and appropriate to the situation is a critical driver for all life cycle models.


\subsubsection{Life Cycle Process Iteration}
\label{\detokenize{SE/sebok:life-cycle-process-iteration}}
\sphinxAtStartPar
The concept of iteration applies to LC stages within a LC model, and also applies to processes.


\begin{savenotes}\sphinxattablestart
\sphinxthistablewithglobalstyle
\centering
\begin{tabulary}{\linewidth}[t]{T}
\sphinxtoprule
\sphinxstyletheadfamily 
\sphinxAtStartPar
\sphinxincludegraphics{{sebok-LC_process_iteration}.svg}
\\
\sphinxmidrule
\sphinxtableatstartofbodyhook
\sphinxAtStartPar
Concept and System Definition processes iterations
\\
\sphinxbottomrule
\end{tabulary}
\sphinxtableafterendhook\par
\sphinxattableend\end{savenotes}

\sphinxAtStartPar
Figure 3 below gives an example of the iteration between the other life cycle processes.
The iterations in this example relate to the overlaps in process outcomes shown in Figure 1.
They either allow consideration of cross process issues to influence the system definition (e.g. considering likely integration or verification approaches might make us think about failure modes or add data collection or monitoring elements into the system) or they allow risk management and through life planning activities to identify the need for future activities.


\begin{savenotes}\sphinxattablestart
\sphinxthistablewithglobalstyle
\centering
\begin{tabulary}{\linewidth}[t]{T}
\sphinxtoprule
\sphinxstyletheadfamily 
\sphinxAtStartPar
\sphinxincludegraphics{{sebok-LC_process_iteration_2}.svg}
\\
\sphinxmidrule
\sphinxtableatstartofbodyhook
\sphinxAtStartPar
Concept and System Definition processes iterations — System realization
\\
\sphinxbottomrule
\end{tabulary}
\sphinxtableafterendhook\par
\sphinxattableend\end{savenotes}


\subsubsection{Life Cycle Process Recursion}
\label{\detokenize{SE/sebok:life-cycle-process-recursion}}
\sphinxAtStartPar
The comprehensive definition of a SoI is generally achieved using decomposition layers and system elements.

\sphinxAtStartPar
Figure 4 presents a fundamental schema of a SBS.
The comprehensive definition of a SoI is generally achieved using decomposition layers and system elements.
In each decomposition layer and for each system, the System Definition processes are applied recursively because the notion of “system” is in itself recursive; the notions of SoI, system, and system element are based on the same
concepts (see Part 2).


\begin{savenotes}\sphinxattablestart
\sphinxthistablewithglobalstyle
\centering
\begin{tabulary}{\linewidth}[t]{T}
\sphinxtoprule
\sphinxstyletheadfamily 
\sphinxAtStartPar
\sphinxincludegraphics{{sebok-LC_process_iteration_recursion}.svg}
\\
\sphinxmidrule
\sphinxtableatstartofbodyhook
\sphinxAtStartPar
Concept and System Definition processes iterations — Recursion
\\
\sphinxbottomrule
\end{tabulary}
\sphinxtableafterendhook\par
\sphinxattableend\end{savenotes}


\subsubsection{Systems Approach to Solution Synthesis}
\label{\detokenize{SE/sebok:systems-approach-to-solution-synthesis}}

\paragraph{Top\sphinxhyphen{}Down Approach: From Problem to Solution}
\label{\detokenize{SE/sebok:top-down-approach-from-problem-to-solution}}
\sphinxAtStartPar
In a \sphinxstylestrong{top\sphinxhyphen{}down} approach, concept definition activities
\begin{itemize}
\item {} 
\sphinxAtStartPar
are focused primarily on understanding
\begin{itemize}
\item {} 
\sphinxAtStartPar
the problem,

\item {} 
\sphinxAtStartPar
the operational needs/requirements within the problem space, and

\item {} 
\sphinxAtStartPar
the conditions that constrain the solution and bound the solution space.

\end{itemize}

\item {} 
\sphinxAtStartPar
determine
\begin{itemize}
\item {} 
\sphinxAtStartPar
the mission context,

\item {} 
\sphinxAtStartPar
the mission analysis, and

\item {} 
\sphinxAtStartPar
te needs to be fulfilled in that context by a new or modified system (i.e. the SoI), and

\end{itemize}

\item {} 
\sphinxAtStartPar
address stakeholder needs and requirements.

\item {} 
\sphinxAtStartPar
consider functional, behavioral, temporal, and physical aspects of one or more solutions based on the results of concept definition.

\end{itemize}

\sphinxAtStartPar
System analysis:
\begin{itemize}
\item {} 
\sphinxAtStartPar
considers the advantages and disadvantages of the proposed system solutions both in terms of
\begin{itemize}
\item {} 
\sphinxAtStartPar
how they satisfy the needs established in concept definition, as well as

\item {} 
\sphinxAtStartPar
the relative cost, time scales and other development issues.

\end{itemize}

\item {} 
\sphinxAtStartPar
requires further refinement of the concept definition to ensure all legacy relationships and stakeholders relevant to a particular solution architecture have been considered in the stakeholder requirements.

\end{itemize}

\sphinxAtStartPar
The outcomes of this iteration between \sphinxstyleemphasis{Concept Definition} and \sphinxstyleemphasis{System Definition} define
\begin{itemize}
\item {} 
\sphinxAtStartPar
a required system solution and

\item {} 
\sphinxAtStartPar
its associated problem context, which are used for
\begin{itemize}
\item {} 
\sphinxAtStartPar
\sphinxstyleemphasis{System Realization},

\item {} 
\sphinxAtStartPar
\sphinxstyleemphasis{System Deployment and Use}, and

\item {} 
\sphinxAtStartPar
\sphinxstyleemphasis{Product and Service Life Management} of one or more solution implementations.

\end{itemize}

\end{itemize}

\sphinxAtStartPar
In this approach, problem understanding and solution selection activities are
\begin{itemize}
\item {} 
\sphinxAtStartPar
completed in the front\sphinxhyphen{}end portion of system development and design and then

\item {} 
\sphinxAtStartPar
maintained and refined as necessary throughout the LC of any resulting solution systems.

\end{itemize}

\sphinxAtStartPar
Depending upon the LC model, \sphinxstylestrong{top\sphinxhyphen{}down} activities can be
\begin{itemize}
\item {} 
\sphinxAtStartPar
sequential,

\item {} 
\sphinxAtStartPar
iterative,

\item {} 
\sphinxAtStartPar
recursive, or

\item {} 
\sphinxAtStartPar
evolutionary.

\end{itemize}


\paragraph{Bottom\sphinxhyphen{}Up Approach: Evolution of the Solution}
\label{\detokenize{SE/sebok:bottom-up-approach-evolution-of-the-solution}}
\sphinxAtStartPar
In some situations, the concept definition activities
\begin{itemize}
\item {} 
\sphinxAtStartPar
determine the need to evolve existing capabilities or

\item {} 
\sphinxAtStartPar
add new capabilities to an existing system.

\end{itemize}

\sphinxAtStartPar
During the concept definition, the alternatives to address the needs are evaluated.

\sphinxAtStartPar
Engineers are then led to reconsider the system definition in order to modify or adapt some structural, functional, behavioral, or temporal properties during the product or service life cycle for a changing context of use or for the purpose of improving existing solutions.

\sphinxAtStartPar
Reverse engineering is often necessary to
\begin{itemize}
\item {} 
\sphinxAtStartPar
enable system engineers to (re)characterize the properties of the
system\sphinxhyphen{}of\sphinxhyphen{}interest (SoI) or its elements.

\item {} 
\sphinxAtStartPar
ensure that system engineers understand the SoI before beginning modification.

\end{itemize}

\sphinxAtStartPar
A \sphinxstylestrong{bottom\sphinxhyphen{}up} approach is necessary for
\begin{itemize}
\item {} 
\sphinxAtStartPar
analysis purposes, or

\item {} 
\sphinxAtStartPar
(re)using existing elements in the design architecture.

\end{itemize}

\sphinxAtStartPar
Changes in the context of use or a need for improvement can prompt this.
In contrast, a \sphinxstylestrong{top\sphinxhyphen{}down} approach is generally used to define an initial design solution corresponding to a problem or a set of needs.


\paragraph{Solution Synthesis}
\label{\detokenize{SE/sebok:solution-synthesis}}
\sphinxAtStartPar
In most real problems, a combination of \sphinxstylestrong{bottom\sphinxhyphen{}up} and \sphinxstylestrong{top\sphinxhyphen{}down} approaches provides the right mixture of innovative solution thinking driven by need, and constrained and pragmatic thinking driven by what already exists.

\sphinxAtStartPar
This is often referred to as a “middle\sphinxhyphen{}out” approach.

\sphinxAtStartPar
As well as being the most pragmatic approach, synthesis has the potential to
\begin{itemize}
\item {} 
\sphinxAtStartPar
keep the life cycle focused on whole system issues, and

\item {} 
\sphinxAtStartPar
allow the exploration of the focused levels of detail needed to describe realizable solutions.

\end{itemize}


\section{Knowledge Area: System Life Cycle Models}
\label{\detokenize{SE/sebok:knowledge-area-system-life-cycle-models}}

\subsection{Categories of Life Cycle Model}
\label{\detokenize{SE/sebok:categories-of-life-cycle-model}}
\sphinxAtStartPar
Categories of potential LC process models:
\begin{itemize}
\item {} 
\sphinxAtStartPar
Pre\sphinxhyphen{}specified
\begin{itemize}
\item {} 
\sphinxAtStartPar
single\sphinxhyphen{}step

\item {} 
\sphinxAtStartPar
multi\sphinxhyphen{}step

\end{itemize}

\item {} 
\sphinxAtStartPar
Evolutionary
\begin{itemize}
\item {} 
\sphinxAtStartPar
sequential

\item {} 
\sphinxAtStartPar
opportunistic

\item {} 
\sphinxAtStartPar
concurrent

\end{itemize}

\item {} 
\sphinxAtStartPar
Interpersonal and emergent

\end{itemize}

\sphinxAtStartPar
The emergence of integrated, interactive hardware\sphinxhyphen{}software systems made pre\sphinxhyphen{}specified processes potentially harmful, as the most effective human\sphinxhyphen{}system interfaces tended to emerge with its use.
This led to the introduction of more lean approaches to concurrent hardware\sphinxhyphen{}software\sphinxhyphen{}human factors approaches such as:
\begin{itemize}
\item {} 
\sphinxAtStartPar
concurrent vee models, and

\item {} 
\sphinxAtStartPar
Incremental Commitment Spiral Model.

\end{itemize}


\section{System Life Cycle Process Drivers and Choices}
\label{\detokenize{SE/sebok:system-life-cycle-process-drivers-and-choices}}
\sphinxAtStartPar
LC processes:
\begin{itemize}
\item {} 
\sphinxAtStartPar
impacted by many organizational factors,

\item {} 
\sphinxAtStartPar
impact all other aspects of system design and development.

\end{itemize}


\subsection{Fixed\sphinxhyphen{}Requirements and Evolutionary Development Processes}
\label{\detokenize{SE/sebok:fixed-requirements-and-evolutionary-development-processes}}
\sphinxAtStartPar
Aside from the traditional, pre\sphinxhyphen{}specified, sequential, single\sphinxhyphen{}step development process (identified as Fixed Requirements), there are several models of evolutionary development processes; however, there is no one\sphinxhyphen{}size\sphinxhyphen{}fits\sphinxhyphen{}all approach that is best for all situations.

\sphinxAtStartPar
For rapid\sphinxhyphen{}fielding situations, an easiest\sphinxhyphen{}first, prototyping approach may be most appropriate.
For enduring systems, an easiest\sphinxhyphen{}first approach may produce an unscalable system, in which the architecture is incapable of achieving high levels of performance, safety, or security.

\sphinxAtStartPar
In general, system evolution now requires
\begin{itemize}
\item {} 
\sphinxAtStartPar
much higher sustained levels of SE effort,

\item {} 
\sphinxAtStartPar
earlier and continuous integration and testing,

\item {} 
\sphinxAtStartPar
proactive approaches to address sources of system change,

\item {} 
\sphinxAtStartPar
greater levels of concurrent engineering, and

\item {} 
\sphinxAtStartPar
achievement reviews based on evidence of feasibility versus plans and system descriptions.

\end{itemize}

\sphinxAtStartPar
Evolutionary development processes or methods have been in use since the 1960s (and perhaps earlier).

\sphinxAtStartPar
They allow a project to provide an initial capability followed by successive deliveries to reach the desired SoI.

\sphinxAtStartPar
This practice is particularly valuable in cases in which
\begin{itemize}
\item {} 
\sphinxAtStartPar
rapid exploration and implementation of part of the system is desired,

\item {} 
\sphinxAtStartPar
requirements are unclear from the beginning, or are rapidly changing,

\item {} 
\sphinxAtStartPar
funding is constrained,

\item {} 
\sphinxAtStartPar
the customer wishes to hold the SoI open to the possibility of inserting new technology when it becomes mature, and

\item {} 
\sphinxAtStartPar
experimentation is required to develop successive versions.

\end{itemize}

\sphinxAtStartPar
In evolutionary development a capability of the product is developed in an increment of time.
Each cycle of the increment subsumes the system elements of the previous increment and adds new capabilities to the evolving product to create an expanded version of the product in development.
This evolutionary development process, that uses increments, can provide a number of advantages, including:
\begin{itemize}
\item {} 
\sphinxAtStartPar
continuous integration, verification, and validation of the evolving product,

\item {} 
\sphinxAtStartPar
frequent demonstrations of progress,

\item {} 
\sphinxAtStartPar
early detection of defects,

\item {} 
\sphinxAtStartPar
early warning of process problems, and

\item {} 
\sphinxAtStartPar
systematic incorporation of the inevitable rework that may occur.

\end{itemize}


\subsection{Primary Models of Incremental and Evolutionary Development}
\label{\detokenize{SE/sebok:primary-models-of-incremental-and-evolutionary-development}}

\begin{savenotes}\sphinxattablestart
\sphinxthistablewithglobalstyle
\centering
\begin{tabulary}{\linewidth}[t]{T}
\sphinxtoprule
\sphinxstyletheadfamily 
\sphinxAtStartPar
\sphinxincludegraphics{{sebok-models_incremental_delivery}.svg}
\\
\sphinxmidrule
\sphinxtableatstartofbodyhook
\sphinxAtStartPar
Primary models of incremental and evolutionary development
\\
\sphinxbottomrule
\end{tabulary}
\sphinxtableafterendhook\par
\sphinxattableend\end{savenotes}


\begin{savenotes}\sphinxattablestart
\sphinxthistablewithglobalstyle
\centering
\begin{tabulary}{\linewidth}[t]{TTTTT}
\sphinxtoprule
\sphinxstyletheadfamily 
\sphinxAtStartPar
Type
&\sphinxstyletheadfamily 
\sphinxAtStartPar
Subtype
&\sphinxstyletheadfamily 
\sphinxAtStartPar
Pros
&\sphinxstyletheadfamily 
\sphinxAtStartPar
Cons
&\sphinxstyletheadfamily 
\sphinxAtStartPar
Examples
\\
\sphinxmidrule
\sphinxtableatstartofbodyhook
\sphinxAtStartPar
Pre\sphinxhyphen{}specified
&
\sphinxAtStartPar
Single\sphinxhyphen{}step
&
\sphinxAtStartPar
EfficientEasy to verify
&
\sphinxAtStartPar
Difficulties with rapid changeEmerging requirements
&
\sphinxAtStartPar
Simple manufactured products
\\
\sphinxhline
\sphinxAtStartPar

&
\sphinxAtStartPar
Multi\sphinxhyphen{}step
&
\sphinxAtStartPar
Early initial capabilityScalability when stable human\sphinxhyphen{}intensive systems
&
\sphinxAtStartPar
Emergent requirements or rapid changeArchitecture breakers
&
\sphinxAtStartPar
Vehicle platform plus value\sphinxhyphen{}adding pre\sphinxhyphen{}planned product improvements (PPPIs)
\\
\sphinxhline
\sphinxAtStartPar
Evolutionary
&
\sphinxAtStartPar
Sequential
&
\sphinxAtStartPar
Adaptability to changeSmaller human\sphinxhyphen{}intensive systems
&
\sphinxAtStartPar
Easiest\sphinxhyphen{}firstLateCostly fixesSE time gapsSlow for large systems
&
\sphinxAtStartPar
Small: AgileLarger: Rapid fielding
\\
\sphinxhline
\sphinxAtStartPar

&
\sphinxAtStartPar
Opportunistic
&
\sphinxAtStartPar
Mature technology upgrades
&
\sphinxAtStartPar
Emergent requirements or rapid changeSySE time gaps
&
\sphinxAtStartPar
Stable developmentMaturing technology
\\
\sphinxhline
\sphinxAtStartPar

&
\sphinxAtStartPar
Concurrent
&
\sphinxAtStartPar
Emergent requirements or rapid changeStable development incrementsSysE continuity
&
\sphinxAtStartPar
Overkill on small or highly stable systems
&
\sphinxAtStartPar
Rapid, emergent developmentSystems of systems
\\
\sphinxbottomrule
\end{tabulary}
\sphinxtableafterendhook\par
\sphinxattableend\end{savenotes}


\subsection{Incremental and Evolutionary Development Decision Table}
\label{\detokenize{SE/sebok:incremental-and-evolutionary-development-decision-table}}

\begin{savenotes}\sphinxattablestart
\sphinxthistablewithglobalstyle
\centering
\begin{tabulary}{\linewidth}[t]{TTTTTT}
\sphinxtoprule
\sphinxstyletheadfamily 
\sphinxAtStartPar
Type
&\sphinxstyletheadfamily 
\sphinxAtStartPar
Subtype
&\sphinxstyletheadfamily 
\sphinxAtStartPar
Stable, pre\sphinxhyphen{}specific requirements?
&\sphinxstyletheadfamily 
\sphinxAtStartPar
Ok to wait for full systemto be developed?
&\sphinxstyletheadfamily 
\sphinxAtStartPar
Need to wait fornext\sphinxhyphen{}increment priorities?
&\sphinxstyletheadfamily 
\sphinxAtStartPar
Need to wait fornext\sphinxhyphen{}increment enablers?
\\
\sphinxmidrule
\sphinxtableatstartofbodyhook
\sphinxAtStartPar
Pre\sphinxhyphen{}specified
&
\sphinxAtStartPar
Single\sphinxhyphen{}step
&
\sphinxAtStartPar
True
&
\sphinxAtStartPar
True
&
\sphinxAtStartPar

&
\sphinxAtStartPar

\\
\sphinxhline
\sphinxAtStartPar

&
\sphinxAtStartPar
Multi\sphinxhyphen{}step
&
\sphinxAtStartPar
True
&
\sphinxAtStartPar
False
&
\sphinxAtStartPar

&
\sphinxAtStartPar

\\
\sphinxhline
\sphinxAtStartPar
Evolutionary
&
\sphinxAtStartPar
Sequential
&
\sphinxAtStartPar
False
&
\sphinxAtStartPar
False
&
\sphinxAtStartPar
True
&
\sphinxAtStartPar

\\
\sphinxhline
\sphinxAtStartPar

&
\sphinxAtStartPar
Opportunistic
&
\sphinxAtStartPar
False
&
\sphinxAtStartPar
False
&
\sphinxAtStartPar
False
&
\sphinxAtStartPar
True
\\
\sphinxhline
\sphinxAtStartPar

&
\sphinxAtStartPar
Concurrent
&
\sphinxAtStartPar
False
&
\sphinxAtStartPar
False
&
\sphinxAtStartPar
False
&
\sphinxAtStartPar
False
\\
\sphinxbottomrule
\end{tabulary}
\sphinxtableafterendhook\par
\sphinxattableend\end{savenotes}


\section{Evolutionary Sequential SLC Model: Vee}
\label{\detokenize{SE/sebok:evolutionary-sequential-slc-model-vee}}

\begin{savenotes}\sphinxattablestart
\sphinxthistablewithglobalstyle
\centering
\begin{tabulary}{\linewidth}[t]{T}
\sphinxtoprule
\sphinxstyletheadfamily 
\sphinxAtStartPar
\sphinxincludegraphics{{sebok-vee_left}.svg}
\\
\sphinxmidrule
\sphinxtableatstartofbodyhook
\sphinxAtStartPar
Left Side of the Sequential Vee Model
\\
\sphinxbottomrule
\end{tabulary}
\sphinxtableafterendhook\par
\sphinxattableend\end{savenotes}


\begin{savenotes}\sphinxattablestart
\sphinxthistablewithglobalstyle
\centering
\begin{tabulary}{\linewidth}[t]{T}
\sphinxtoprule
\sphinxstyletheadfamily 
\sphinxAtStartPar
\sphinxincludegraphics{{sebok-stages}.svg}
\\
\sphinxmidrule
\sphinxtableatstartofbodyhook
\sphinxAtStartPar
Stages, Purposes, and Major Decision Gates
\\
\sphinxbottomrule
\end{tabulary}
\sphinxtableafterendhook\par
\sphinxattableend\end{savenotes}


\begin{savenotes}\sphinxattablestart
\sphinxthistablewithglobalstyle
\centering
\begin{tabulary}{\linewidth}[t]{T}
\sphinxtoprule
\sphinxstyletheadfamily 
\sphinxAtStartPar
\sphinxincludegraphics{{sebok-vee_activity}.svg}
\\
\sphinxmidrule
\sphinxtableatstartofbodyhook
\sphinxAtStartPar
Vee Activity Diagram
\\
\sphinxbottomrule
\end{tabulary}
\sphinxtableafterendhook\par
\sphinxattableend\end{savenotes}


\begin{savenotes}\sphinxattablestart
\sphinxthistablewithglobalstyle
\centering
\begin{tabulary}{\linewidth}[t]{T}
\sphinxtoprule
\sphinxstyletheadfamily 
\sphinxAtStartPar
\sphinxincludegraphics{{sebok-vee_right}.svg}
\\
\sphinxmidrule
\sphinxtableatstartofbodyhook
\sphinxAtStartPar
Right Side of the Sequential Vee Model
\\
\sphinxbottomrule
\end{tabulary}
\sphinxtableafterendhook\par
\sphinxattableend\end{savenotes}


\section{Evolutionary Incremental SLC Models}
\label{\detokenize{SE/sebok:evolutionary-incremental-slc-models}}

\subsection{Evolutionary Approach}
\label{\detokenize{SE/sebok:evolutionary-approach}}

\begin{savenotes}\sphinxattablestart
\sphinxthistablewithglobalstyle
\centering
\begin{tabulary}{\linewidth}[t]{T}
\sphinxtoprule
\sphinxstyletheadfamily 
\sphinxAtStartPar
\sphinxincludegraphics{{sebok-evolutionary_development}.svg}
\\
\sphinxmidrule
\sphinxtableatstartofbodyhook
\sphinxAtStartPar
Evolutionary Approach
\\
\sphinxbottomrule
\end{tabulary}
\sphinxtableafterendhook\par
\sphinxattableend\end{savenotes}


\subsection{Incremental Approach}
\label{\detokenize{SE/sebok:incremental-approach}}



\subsection{Evolutionary Concurrent LC Model: Incremental Commitment Spiral}
\label{\detokenize{SE/sebok:evolutionary-concurrent-lc-model-incremental-commitment-spiral}}
\sphinxAtStartPar
Each spiral addresses requirements and solutions concurrently, rather than sequentially, as well as
\begin{itemize}
\item {} 
\sphinxAtStartPar
products and processes,

\item {} 
\sphinxAtStartPar
hardware,

\item {} 
\sphinxAtStartPar
software,

\item {} 
\sphinxAtStartPar
human factors aspects, and

\item {} 
\sphinxAtStartPar
business case analyses of alternative product configurations/product line investments.

\end{itemize}

\sphinxAtStartPar
Stakeholders
\begin{itemize}
\item {} 
\sphinxAtStartPar
consider the risks and risk mitigation plans, and

\item {} 
\sphinxAtStartPar
decide on a course of action.

\end{itemize}

\sphinxAtStartPar
If the risks are acceptable and covered by risk mitigation plans, the project proceeds into the next spiral.

\sphinxAtStartPar
The development spirals after the first development commitment review follow the three\sphinxhyphen{}team incremental
development approach for achieving both agility and assurance.


\begin{savenotes}\sphinxattablestart
\sphinxthistablewithglobalstyle
\centering
\begin{tabulary}{\linewidth}[t]{T}
\sphinxtoprule
\sphinxstyletheadfamily 
\sphinxAtStartPar
\sphinxincludegraphics{{sebok-incremental_spiral_phased}.svg}
\\
\sphinxmidrule
\sphinxtableatstartofbodyhook
\sphinxAtStartPar
Phased View of the Generic Incremental Commitment Spiral Model Process
\\
\sphinxbottomrule
\end{tabulary}
\sphinxtableafterendhook\par
\sphinxattableend\end{savenotes}


\subsection{Agile and Lean Processes}
\label{\detokenize{SE/sebok:agile-and-lean-processes}}
\sphinxAtStartPar
Agile development methods can be used to support iterative LC models, allowing flexibility over a linear process that better aligns with the planned LC for a system.

\sphinxAtStartPar
Lean processes are often associated with agile methods, although they are more scalable and applicable to
high\sphinxhyphen{}assurance systems.


\subsubsection{Scrum}
\label{\detokenize{SE/sebok:scrum}}

\subsubsection{Architected Agile Methods}
\label{\detokenize{SE/sebok:architected-agile-methods}}
\sphinxAtStartPar
Over the last decade, several organizations have been able to scale up agile methods by using two layers of
ten\sphinxhyphen{}person Scrum teams.

\sphinxAtStartPar
This involves, among other things, having each Scrum team’s daily meeting followed up by a daily meeting of the Scrum team leaders discussing up\sphinxhyphen{}front investments in evolving system architecture (Boehm
et al. 2010).


\begin{savenotes}\sphinxattablestart
\sphinxthistablewithglobalstyle
\centering
\begin{tabulary}{\linewidth}[t]{T}
\sphinxtoprule
\sphinxstyletheadfamily 
\sphinxAtStartPar
\sphinxincludegraphics{{sebok-architected_agile}.svg}
\\
\sphinxmidrule
\sphinxtableatstartofbodyhook
\sphinxAtStartPar
Architected Agile Process
\\
\sphinxbottomrule
\end{tabulary}
\sphinxtableafterendhook\par
\sphinxattableend\end{savenotes}


\section{System Life Cycle Process Models: Agile Systems Engineering}
\label{\detokenize{SE/sebok:system-life-cycle-process-models-agile-systems-engineering}}
\sphinxAtStartPar
A system LC starts at the concept definition phase, moves through stages until completion of this system, as defined in the concept definition stage.

\sphinxAtStartPar
A model representation of the LC may be
\begin{itemize}
\item {} 
\sphinxAtStartPar
physical,

\item {} 
\sphinxAtStartPar
data, or

\item {} 
\sphinxAtStartPar
graphic.

\end{itemize}

\sphinxAtStartPar
The process describes the steps to accomplish each stage of the LC including input to and output from this stage.

\sphinxAtStartPar
Today’s complex and increasingly highly connected systems face rapid obsolescence under the stress of technological change, environmental change, and rapidly evolving mission needs.
For these systems to remain robust against disruption they must be architected to agilely adapt.
To meet these needs, the system must be assessed to apply the process that best serves the system, subsystem or component of the SoI.

\sphinxAtStartPar
It is important to determine the best LC to use for the SoI early in the concept definition phase.

\sphinxAtStartPar
On a program that is going to operate agilely, especially if it will be a hybrid model with agile, and other LC models it is
important to define and harmonize them at key integration points based on
\begin{itemize}
\item {} 
\sphinxAtStartPar
hardware or

\item {} 
\sphinxAtStartPar
other long\sphinxhyphen{}lead item maturity.

\end{itemize}

\sphinxAtStartPar
In the Agile SE process, the systems engineer works in an iterative, incremental manner, continually modeling, analyzing, developing, and trading options to bring the definition of the system solution into focus.

\sphinxAtStartPar
An example of this work will be analyzing and maintaining
\begin{itemize}
\item {} 
\sphinxAtStartPar
the requirements,

\item {} 
\sphinxAtStartPar
the architectural model of the higher\sphinxhyphen{}level requirements, and

\item {} 
\sphinxAtStartPar
linkage from those high\sphinxhyphen{}level requirements to the analyzed lower\sphinxhyphen{}level requirements.

\item {} 
\sphinxAtStartPar
the interfaces are defined and followed as the development progresses

\end{itemize}


\subsection{Frameworks}
\label{\detokenize{SE/sebok:frameworks}}
\sphinxAtStartPar
The Agile SE process steps that are performed in each of the stages often include:
\begin{enumerate}
\sphinxsetlistlabels{\arabic}{enumi}{enumii}{}{.}%
\item {} 
\sphinxAtStartPar
Define the highest priority and/or highest risk item while keeping design options open.

\item {} 
\sphinxAtStartPar
Design the solutions to meet those requirements, develop their products, perform tests, and demonstrate that product.

\item {} 
\sphinxAtStartPar
For large products in development where multiple teams integrate their work items together to show a demonstratable product, several iterations may be needed to get to that point.

\item {} 
\sphinxAtStartPar
Prior to starting an increment, all teams working to produce demonstrable products, should meet to plan their work, identify dependencies between the teams and establish commitments to meet the plan.

\item {} 
\sphinxAtStartPar
Release product to stakeholders and plan the next increment of work.

\end{enumerate}

\sphinxAtStartPar
This Agile SE Framework aligns with the Scaled Agile Framework (SAFe) depiction of
\begin{itemize}
\item {} 
\sphinxAtStartPar
teams working program and

\item {} 
\sphinxAtStartPar
team backlogs using iterative development.

\end{itemize}

\sphinxAtStartPar
SAFe
\begin{itemize}
\item {} 
\sphinxAtStartPar
is a framework that implements the principles of iterative development,

\item {} 
\sphinxAtStartPar
represents how a large system may have multiple LC processes being followed in parallel over time

\item {} 
\sphinxAtStartPar
key decision points need to be aligned between the multiple LC processes.

\end{itemize}

\sphinxAtStartPar
There are many agile approaches that a program could use as is or combined to adapt to what works best for a given domain.

\sphinxAtStartPar
For a complex system with changing requirements the assessment may result in the decision to use an incremental,
iterative approach for development.

\sphinxAtStartPar
Regardless of which model or framework is selected a program starts with a vision, a budget and usually a period of performance. Then the program’s stakeholders identify the highest value capability to develop first
The list of capabilities is prioritized so that the long\sphinxhyphen{}term development is visible.
However, this prioritized order may change as work progresses.
What is known about the intended product may have well defined requirements and architecture representations and what is conceptual will have those requirements and designs developed incrementally as time progresses.

\sphinxAtStartPar
This incremental method of development is enabled by the use of
\begin{itemize}
\item {} 
\sphinxAtStartPar
an open system architecture,

\item {} 
\sphinxAtStartPar
MBSE tools,

\item {} 
\sphinxAtStartPar
Set\sphinxhyphen{}based design,

\item {} 
\sphinxAtStartPar
design thinking,

\item {} 
\sphinxAtStartPar
continuous integration,

\item {} 
\sphinxAtStartPar
continuous development,

\item {} 
\sphinxAtStartPar
architecture patterns,

\item {} 
\sphinxAtStartPar
microservice architecture, and

\item {} 
\sphinxAtStartPar
Lean sengineering.

\end{itemize}


\section{Knowledge Area: Systems Engineering Management}
\label{\detokenize{SE/sebok:knowledge-area-systems-engineering-management}}

\subsection{Systems Engineering Management}
\label{\detokenize{SE/sebok:systems-engineering-management}}

\subsubsection{Discussion}
\label{\detokenize{SE/sebok:discussion}}\begin{itemize}
\item {} 
\sphinxAtStartPar
Single
\begin{itemize}
\item {} 
\sphinxAtStartPar
SE
\begin{itemize}
\item {} 
\sphinxAtStartPar
Needs and Opportunities Analysis

\item {} 
\sphinxAtStartPar
Operational Concept Development

\item {} 
\sphinxAtStartPar
System Scoping and Requiremens Definition

\item {} 
\sphinxAtStartPar
Architecture Definition

\item {} 
\sphinxAtStartPar
Trade\sphinxhyphen{}off Analysis, Modeling, and Simulation

\end{itemize}

\item {} 
\sphinxAtStartPar
P/SM
\begin{itemize}
\item {} 
\sphinxAtStartPar
Staffing, Organizing, Directing

\item {} 
\sphinxAtStartPar
Cost, Schedule, Performance, Risk Monitoring and Control

\item {} 
\sphinxAtStartPar
Operations Planning and Presentation

\item {} 
\sphinxAtStartPar
Operations Management

\end{itemize}

\item {} 
\sphinxAtStartPar
SI
\begin{itemize}
\item {} 
\sphinxAtStartPar
Production Line Preparation

\item {} 
\sphinxAtStartPar
Production

\item {} 
\sphinxAtStartPar
Production Control

\item {} 
\sphinxAtStartPar
Testing

\end{itemize}

\end{itemize}

\item {} 
\sphinxAtStartPar
Double
\begin{itemize}
\item {} 
\sphinxAtStartPar
SE + P/SM
\begin{itemize}
\item {} 
\sphinxAtStartPar
Business Case Analysis

\item {} 
\sphinxAtStartPar
Systems Engineering Management

\end{itemize}

\item {} 
\sphinxAtStartPar
SE + SI
\begin{itemize}
\item {} 
\sphinxAtStartPar
Production Planning and Analysis

\item {} 
\sphinxAtStartPar
System Integration

\end{itemize}

\item {} 
\sphinxAtStartPar
P/SM + SI
\begin{itemize}
\item {} 
\sphinxAtStartPar
Supply Chain Management

\item {} 
\sphinxAtStartPar
Systems Implementation Management

\end{itemize}

\end{itemize}

\item {} 
\sphinxAtStartPar
Triple
\begin{itemize}
\item {} 
\sphinxAtStartPar
SE + SI + P/SM
\begin{itemize}
\item {} 
\sphinxAtStartPar
Life Cycle Planning and Estimating

\item {} 
\sphinxAtStartPar
Change Analysis and Management

\item {} 
\sphinxAtStartPar
Q\&A, V\&V, Continuous Process Improvement

\end{itemize}

\end{itemize}

\end{itemize}


\section{Knowledge Area: Systems Engineering and Industrial Engineering}
\label{\detokenize{SE/sebok:knowledge-area-systems-engineering-and-industrial-engineering}}

\subsection{Systems Engineering}
\label{\detokenize{SE/sebok:systems-engineering}}\begin{quote}

\sphinxAtStartPar
A transdisciplinary and integrative approach to enable the successful realization, use, and retirement of engineered systems, using systems principles and concepts, and scientific, technological, and management methods.
\end{quote}


\subsection{Industrial Engineering}
\label{\detokenize{SE/sebok:industrial-engineering}}

\subsection{Venn Diagram Comparison}
\label{\detokenize{SE/sebok:venn-diagram-comparison}}
\sphinxAtStartPar
SE
\begin{itemize}
\item {} 
\sphinxAtStartPar
Business/Mission Analysis

\item {} 
\sphinxAtStartPar
Stakeholder Needs \& Requirements

\item {} 
\sphinxAtStartPar
System Requirements

\item {} 
\sphinxAtStartPar
System Architecture (Logical and Physical)

\item {} 
\sphinxAtStartPar
Systems Design and Engineering

\item {} 
\sphinxAtStartPar
Systems Analysis

\item {} 
\sphinxAtStartPar
Implementation

\item {} 
\sphinxAtStartPar
Systems Integration

\item {} 
\sphinxAtStartPar
Systems Verification

\item {} 
\sphinxAtStartPar
Systems Validation

\item {} 
\sphinxAtStartPar
System Operation

\end{itemize}

\sphinxAtStartPar
IE
\begin{itemize}
\item {} 
\sphinxAtStartPar
Work Design  Measurement

\item {} 
\sphinxAtStartPar
Engineering Economics

\item {} 
\sphinxAtStartPar
Facilities Engineering \& Management

\item {} 
\sphinxAtStartPar
Operations Engineering \& Management

\item {} 
\sphinxAtStartPar
Supply Chain Management

\item {} 
\sphinxAtStartPar
Safety

\item {} 
\sphinxAtStartPar
Design \& Manufacturing Engineering

\end{itemize}

\sphinxAtStartPar
SE + IE
\begin{itemize}
\item {} 
\sphinxAtStartPar
OR \& Analysis

\item {} 
\sphinxAtStartPar
Quality \& Reliability Engineering

\item {} 
\sphinxAtStartPar
Ergonomics \& Human Factors

\item {} 
\sphinxAtStartPar
Engineering Management

\item {} 
\sphinxAtStartPar
Information Engineering

\item {} 
\sphinxAtStartPar
Product Design \& Development

\item {} 
\sphinxAtStartPar
Systems Deployment

\item {} 
\sphinxAtStartPar
Updates, Upgrades, Modernization

\item {} 
\sphinxAtStartPar
Service Life Extension

\item {} 
\sphinxAtStartPar
System Maintenance

\item {} 
\sphinxAtStartPar
Logistics

\item {} 
\sphinxAtStartPar
Disposal \& Retirement

\end{itemize}


\subsection{Roles in a System Life Cycle}
\label{\detokenize{SE/sebok:roles-in-a-system-life-cycle}}

\begin{savenotes}\sphinxattablestart
\sphinxthistablewithglobalstyle
\centering
\begin{tabulary}{\linewidth}[t]{TTTT}
\sphinxtoprule
\sphinxstyletheadfamily 
\sphinxAtStartPar
Stage
&\sphinxstyletheadfamily 
\sphinxAtStartPar
Stage
&\sphinxstyletheadfamily 
\sphinxAtStartPar
Role
&\sphinxstyletheadfamily 
\sphinxAtStartPar
Process
\\
\sphinxmidrule
\sphinxtableatstartofbodyhook
\sphinxAtStartPar
1
&
\sphinxAtStartPar
Establish System Need
&
\sphinxAtStartPar
SE
&
\sphinxAtStartPar

\\
\sphinxhline
\sphinxAtStartPar
2
&
\sphinxAtStartPar
Design and Develop
&
\sphinxAtStartPar
DESE
&
\sphinxAtStartPar
DesignTest and Evaluation
\\
\sphinxhline
\sphinxAtStartPar
3
&
\sphinxAtStartPar
Produce System
&
\sphinxAtStartPar
IE
&
\sphinxAtStartPar
Supply Chain ManagementDevelop ProcessesImprove Processes
\\
\sphinxhline
\sphinxAtStartPar
4
&
\sphinxAtStartPar
Deploy System
&
\sphinxAtStartPar
IE
&
\sphinxAtStartPar
TransportationTraining
\\
\sphinxhline
\sphinxAtStartPar
5
&
\sphinxAtStartPar
Operate System
&
\sphinxAtStartPar
SEIESE/DE
&
\sphinxAtStartPar
Reliability GrowthMaintenanceSystem Upgrades
\\
\sphinxhline
\sphinxAtStartPar
6
&
\sphinxAtStartPar
Retire System
&
\sphinxAtStartPar
IE/SE
&
\sphinxAtStartPar

\\
\sphinxbottomrule
\end{tabulary}
\sphinxtableafterendhook\par
\sphinxattableend\end{savenotes}


\section{Knowledge Area: Systems Engineering and Project Management}
\label{\detokenize{SE/sebok:knowledge-area-systems-engineering-and-project-management}}

\subsection{Relationships between Systems Engineering and Project Management}
\label{\detokenize{SE/sebok:relationships-between-systems-engineering-and-project-management}}

\subsubsection{Overlap}
\label{\detokenize{SE/sebok:overlap}}
\sphinxAtStartPar
There is a great deal of significant overlap between the scope of SE and other resources and the scope of PM.

\sphinxAtStartPar
These sources describe the importance of
\begin{itemize}
\item {} 
\sphinxAtStartPar
understanding the scope of the work at hand,

\item {} 
\sphinxAtStartPar
how to plan for critical activities,

\item {} 
\sphinxAtStartPar
how to manage efforts while reducing risk, and

\item {} 
\sphinxAtStartPar
how to successfully deliver value to a customer.

\end{itemize}

\sphinxAtStartPar
The SE working on a project will plan, monitor, confront risk, and deliver the technical aspects of the project, while the PM is concerned with the same kinds of activities for the overall project.


\subsubsection{Defining Roles and Responsibilities}
\label{\detokenize{SE/sebok:defining-roles-and-responsibilities}}
\sphinxAtStartPar
Regardless of how the roles are divided up on a given project, the best way to reduce confusion is to explicitly describe the roles and responsibilities of the PM and the SE, as well as other key team members.

\sphinxAtStartPar
The Project Management Plan (PMP) and the Systems Engineering Management Plan (SEMP) are key documents used to define the processes and methodologies the project will employ to build and deliver a product or
service.

\sphinxAtStartPar
The PMP
\begin{itemize}
\item {} 
\sphinxAtStartPar
is the master planning document for the project,

\item {} 
\sphinxAtStartPar
describes all activities, including technical activities, to be integrated and controlled during the life of the program.

\end{itemize}

\sphinxAtStartPar
The SEMP
\begin{itemize}
\item {} 
\sphinxAtStartPar
is the master planning document for the
systems engineering technical elements

\item {} 
\sphinxAtStartPar
defines SE processes and methodologies used on the project and the relationship of SE activities to other project activities

\item {} 
\sphinxAtStartPar
must be consistent with and evolve in concert with
the PMP

\item {} 
\sphinxAtStartPar
integrate technical management plans and expectations  with customer plans and activities.

\end{itemize}


\section{The Influence of Project Structure and Governance on Systems Engineering and Project Management Relationships}
\label{\detokenize{SE/sebok:the-influence-of-project-structure-and-governance-on-systems-engineering-and-project-management-relationships}}

\subsection{An Overview of Project Structures}
\label{\detokenize{SE/sebok:an-overview-of-project-structures}}
\sphinxAtStartPar
PM and SE governance are dependent on the organization’s structure.

\sphinxAtStartPar
For some projects, SE is subordinated to PM and in other cases, PM provides support to SE.

\sphinxAtStartPar
Projects
\begin{itemize}
\item {} 
\sphinxAtStartPar
exist within the structural model of an organization.

\item {} 
\sphinxAtStartPar
are one\sphinxhyphen{}time, transient events that are
\begin{itemize}
\item {} 
\sphinxAtStartPar
initiated to accomplish a specific purpose and

\item {} 
\sphinxAtStartPar
terminated when the project objectives are achieved.

\end{itemize}

\end{itemize}

\sphinxAtStartPar
Project size:
\begin{itemize}
\item {} 
\sphinxAtStartPar
On small projects, the same person accomplishes the work activities of both PM and SE.
Because the natures of the work activities are significantly different, it is sometimes more effective to have two persons performing PM and SE, each on a part\sphinxhyphen{}time basis.

\item {} 
\sphinxAtStartPar
On larger projects there are typically too many tasks to be accomplished for one person to accomplish all of the necessary
work.

\item {} 
\sphinxAtStartPar
On very large projects, PM and SE offices with a designated project manager and a designated lead systems engineer

\end{itemize}

\sphinxAtStartPar
Projects are typically organized in one of three ways:
\begin{enumerate}
\sphinxsetlistlabels{\arabic}{enumi}{enumii}{}{.}%
\item {} 
\sphinxAtStartPar
by functional structure,

\item {} 
\sphinxAtStartPar
by project structure, and

\item {} 
\sphinxAtStartPar
by a matrix structure.

\end{enumerate}

\sphinxAtStartPar
In a function\sphinxhyphen{}structured organization, workers are grouped by the functions they perform. The systems engineering
functions can be:
\begin{enumerate}
\sphinxsetlistlabels{\arabic}{enumi}{enumii}{}{.}%
\item {} 
\sphinxAtStartPar
distributed among some of the functional organizations,

\item {} 
\sphinxAtStartPar
centralized within one organization, or

\item {} 
\sphinxAtStartPar
a hybrid, with some of the functions being distributed to the projects, some centralized and some distributed to functional organization.

\end{enumerate}


\begin{savenotes}\sphinxattablestart
\sphinxthistablewithglobalstyle
\centering
\begin{tabulary}{\linewidth}[t]{T}
\sphinxtoprule
\sphinxstyletheadfamily 
\sphinxAtStartPar
\sphinxincludegraphics{{sebok-organizational_continuum}.svg}
\\
\sphinxmidrule
\sphinxtableatstartofbodyhook
\sphinxAtStartPar
Organizational Continuum
\\
\sphinxbottomrule
\end{tabulary}
\sphinxtableafterendhook\par
\sphinxattableend\end{savenotes}


\subsection{Schedule\sphinxhyphen{}Driven versus Requirements\sphinxhyphen{}Driven Influences on Structure and Governance}
\label{\detokenize{SE/sebok:schedule-driven-versus-requirements-driven-influences-on-structure-and-governance}}
\sphinxAtStartPar
This article addresses the influences on governance relationships between the project manager and the systems
engineer.

\sphinxAtStartPar
One factor that establishes this relationship is whether a project is schedule\sphinxhyphen{}driven or requirements\sphinxhyphen{}driven.

\sphinxAtStartPar
In general,
\begin{itemize}
\item {} 
\sphinxAtStartPar
a project manager is responsible for delivering an acceptable product/service on the specified delivery date and within the constraints of the specified schedule, budget, resources, and technology.

\item {} 
\sphinxAtStartPar
the systems engineer is responsible for
\begin{itemize}
\item {} 
\sphinxAtStartPar
collecting and defining the operational requirements,

\item {} 
\sphinxAtStartPar
specifying the systems requirements,

\item {} 
\sphinxAtStartPar
developing the system design,

\item {} 
\sphinxAtStartPar
coordinating component development teams,

\item {} 
\sphinxAtStartPar
integrating the system components as they become available,

\item {} 
\sphinxAtStartPar
verifying that the system to be delivered is correct, complete and consistent to its technical specification, and

\item {} 
\sphinxAtStartPar
validating the operation of the system in its intended environment.

\end{itemize}

\end{itemize}

\sphinxAtStartPar
From a governance perspective,
\begin{itemize}
\item {} 
\sphinxAtStartPar
the project manager is often thought of as being a movie producer who is responsible for balancing the schedule, budget, and resource constraints to meet customer satisfaction

\item {} 
\sphinxAtStartPar
the systems engineer is responsible for product content; ergo, the systems engineer is analogous to a movie director.

\end{itemize}

\sphinxAtStartPar
Organizational structures, discussed previously, provide the project manager and systems engineer with different
levels of governance authority.

\sphinxAtStartPar
In addition, schedule and requirements constraints can influence governance
relationships.
\begin{itemize}
\item {} 
\sphinxAtStartPar
A schedule\sphinxhyphen{}driven project
\begin{itemize}
\item {} 
\sphinxAtStartPar
is one for which meeting the project schedule is more important than satisfying all of the project requirements; in these cases lower priority requirements may not be implemented in order to meet the schedule.

\item {} 
\sphinxAtStartPar
examples:
\begin{itemize}
\item {} 
\sphinxAtStartPar
a project that has an external customer with a contractual delivery date and an escalating late delivery penalty, and

\item {} 
\sphinxAtStartPar
a project for which delivery of the system must meet a major milestone (e.g. a project for an announced product release of a cell phone that is driven by market considerations).

\end{itemize}

\end{itemize}

\item {} 
\sphinxAtStartPar
For schedule\sphinxhyphen{}driven projects,
\begin{itemize}
\item {} 
\sphinxAtStartPar
the project manager is responsible for planning and coordinating the work activities and resources for the project so that the team can accomplish the work in a coordinated manner to meet the schedule.

\item {} 
\sphinxAtStartPar
the systems engineer works with the project manager to determine the technical approach that will meet the schedule.

\item {} 
\sphinxAtStartPar
An Integrated Master Schedule (IMS) is often used to coordinate the project.

\item {} 
\sphinxAtStartPar
Examples:
\begin{itemize}
\item {} 
\sphinxAtStartPar
exploratory development of a new system that is needed to mitigate a potential threat (e.g. military research
project) and

\item {} 
\sphinxAtStartPar
projects that must conform to government regulations in order for the delivered system to be safely operated (e.g.,
aviation and medical device regulations).

\end{itemize}

\end{itemize}

\end{itemize}

\sphinxAtStartPar
An Integrated Master Plan is often used to coordinate event\sphinxhyphen{}driven projects.

\sphinxAtStartPar
To satisfy the product requirements, the systems engineer is responsible for making technical decisions and making
the appropriate technical trades.
When the trade space includes cost, schedule, or resources, the systems engineer
interacts with the project manager who is responsible for providing the resources and facilities needed to implement
a system that satisfies the technical requirements.
Management structure:
\begin{itemize}
\item {} 
\sphinxAtStartPar
Schedule\sphinxhyphen{}driven projects are more likely to have a management structure in which the project manager plays the
central role

\item {} 
\sphinxAtStartPar
Requirement\sphinxhyphen{}driven projects are more likely to have a management structure in which the systems engineer plays the central role

\end{itemize}

\sphinxstepscope


\part{Systems Project Management}

\sphinxstepscope


\chapter{SPM Concepts}
\label{\detokenize{SPM/spm-concepts:spm-concepts}}\label{\detokenize{SPM/spm-concepts::doc}}
\sphinxAtStartPar
Notes from ESD.36


\section{Project}
\label{\detokenize{SPM/spm-concepts:project}}
\sphinxAtStartPar
Project = set of tasks that:
\begin{itemize}
\item {} 
\sphinxAtStartPar
are related to each other

\item {} 
\sphinxAtStartPar
have a specific objective to be completed within certain specifications

\item {} 
\sphinxAtStartPar
have defined start and end dates

\item {} 
\sphinxAtStartPar
have funding limits

\item {} 
\sphinxAtStartPar
consume resources

\end{itemize}


\section{Iron Triangle}
\label{\detokenize{SPM/spm-concepts:iron-triangle}}
\sphinxAtStartPar
Project:
\begin{itemize}
\item {} 
\sphinxAtStartPar
Constraints: Scope

\item {} 
\sphinxAtStartPar
Variables: Cost, Schedule

\end{itemize}


\section{System}
\label{\detokenize{SPM/spm-concepts:system}}
\sphinxAtStartPar
System = set of physical/virtual objects whose interrelationships enable desired function(s).
\begin{itemize}
\item {} 
\sphinxAtStartPar
More than the sum of its parts

\item {} 
\sphinxAtStartPar
Undesired (emergent) functions often exist

\item {} 
\sphinxAtStartPar
System complexity scales with the number of objects as well as the type and number of interconnections between them

\item {} 
\sphinxAtStartPar
Instantaneously available functions, versus “lifecycle” properties (scalability, flexibility, robustness, etc.)

\end{itemize}

\sphinxAtStartPar
Product = system sold for profit


\section{Project Management (PM)}
\label{\detokenize{SPM/spm-concepts:project-management-pm}}
\sphinxAtStartPar
Project Management (PM) = body of methods and tools that facilitate the achievement of project objectives
\begin{itemize}
\item {} 
\sphinxAtStartPar
Within time

\item {} 
\sphinxAtStartPar
Within cost

\item {} 
\sphinxAtStartPar
Within scope

\item {} 
\sphinxAtStartPar
At the desired performance/specification level

\item {} 
\sphinxAtStartPar
While effectively and efficiently utilizing resources

\item {} 
\sphinxAtStartPar
While carefully managing risks and opportunities

\end{itemize}


\section{Research \& Development (R\&D)}
\label{\detokenize{SPM/spm-concepts:research-development-r-d}}

\begin{savenotes}\sphinxattablestart
\sphinxthistablewithglobalstyle
\centering
\begin{tabulary}{\linewidth}[t]{TTTT}
\sphinxtoprule
\sphinxstyletheadfamily 
\sphinxAtStartPar
Development
&\sphinxstyletheadfamily 
\sphinxAtStartPar
Structured
&\sphinxstyletheadfamily 
\sphinxAtStartPar
Planning
&\sphinxstyletheadfamily 
\sphinxAtStartPar
Predictive
\\
\sphinxmidrule
\sphinxtableatstartofbodyhook
\sphinxAtStartPar
Research/Technology
&
\sphinxAtStartPar
False
&
\sphinxAtStartPar
Hard
&
\sphinxAtStartPar
False
\\
\sphinxhline
\sphinxAtStartPar
Product/System Development
&
\sphinxAtStartPar
True
&
\sphinxAtStartPar
Easy
&
\sphinxAtStartPar
True
\\
\sphinxbottomrule
\end{tabulary}
\sphinxtableafterendhook\par
\sphinxattableend\end{savenotes}


\section{Task as an Object\sphinxhyphen{}Process\sphinxhyphen{}Diagram}
\label{\detokenize{SPM/spm-concepts:task-as-an-object-process-diagram}}

\begin{savenotes}\sphinxattablestart
\sphinxthistablewithglobalstyle
\centering
\begin{tabulary}{\linewidth}[t]{T}
\sphinxtoprule
\sphinxstyletheadfamily 
\sphinxAtStartPar
\sphinxincludegraphics{{task_as_object_process_diagram}.svg}
\\
\sphinxmidrule
\sphinxtableatstartofbodyhook
\sphinxAtStartPar
Task
\\
\sphinxbottomrule
\end{tabulary}
\sphinxtableafterendhook\par
\sphinxattableend\end{savenotes}


\begin{savenotes}\sphinxattablestart
\sphinxthistablewithglobalstyle
\centering
\begin{tabulary}{\linewidth}[t]{T}
\sphinxtoprule
\sphinxstyletheadfamily 
\sphinxAtStartPar
\sphinxincludegraphics{{project_as_object_process_diagram}.svg}
\\
\sphinxmidrule
\sphinxtableatstartofbodyhook
\sphinxAtStartPar
Project
\\
\sphinxbottomrule
\end{tabulary}
\sphinxtableafterendhook\par
\sphinxattableend\end{savenotes}


\begin{savenotes}\sphinxattablestart
\sphinxthistablewithglobalstyle
\centering
\begin{tabulary}{\linewidth}[t]{T}
\sphinxtoprule
\sphinxstyletheadfamily 
\sphinxAtStartPar
\sphinxincludegraphics{{SPM_ESD36_framework}.svg}
\\
\sphinxmidrule
\sphinxtableatstartofbodyhook
\sphinxAtStartPar
SPM ESD.36 Framework
\\
\sphinxbottomrule
\end{tabulary}
\sphinxtableafterendhook\par
\sphinxattableend\end{savenotes}

\sphinxstepscope


\chapter{Design Structure Matrix (DSM)}
\label{\detokenize{SPM/DSM:design-structure-matrix-dsm}}\label{\detokenize{SPM/DSM::doc}}
\sphinxAtStartPar
Notes from ESD.36


\section{Sequencing Tasks}
\label{\detokenize{SPM/DSM:sequencing-tasks}}

\begin{savenotes}\sphinxattablestart
\sphinxthistablewithglobalstyle
\centering
\begin{tabulary}{\linewidth}[t]{T}
\sphinxtoprule
\sphinxstyletheadfamily 
\sphinxAtStartPar
\sphinxincludegraphics{{sequences}.svg}
\\
\sphinxmidrule
\sphinxtableatstartofbodyhook
\sphinxAtStartPar
Task possible sequences
\\
\sphinxbottomrule
\end{tabulary}
\sphinxtableafterendhook\par
\sphinxattableend\end{savenotes}


\section{DSMs}
\label{\detokenize{SPM/DSM:dsms}}
\sphinxAtStartPar
Types:
\begin{itemize}
\item {} 
\sphinxAtStartPar
Object\sphinxhyphen{}based

\item {} 
\sphinxAtStartPar
Team\sphinxhyphen{}based

\item {} 
\sphinxAtStartPar
Parameter\sphinxhyphen{}based

\item {} 
\sphinxAtStartPar
Task\sphinxhyphen{}based

\end{itemize}


\subsection{Task\sphinxhyphen{}based DSMs}
\label{\detokenize{SPM/DSM:task-based-dsms}}

\begin{savenotes}\sphinxattablestart
\sphinxthistablewithglobalstyle
\centering
\begin{tabulary}{\linewidth}[t]{T}
\sphinxtoprule
\sphinxstyletheadfamily 
\sphinxAtStartPar
\sphinxincludegraphics{{dsm_task}.svg}
\\
\sphinxmidrule
\sphinxtableatstartofbodyhook
\sphinxAtStartPar
DSM Task Notation
\\
\sphinxbottomrule
\end{tabulary}
\sphinxtableafterendhook\par
\sphinxattableend\end{savenotes}


\begin{savenotes}\sphinxattablestart
\sphinxthistablewithglobalstyle
\centering
\begin{tabulary}{\linewidth}[t]{T}
\sphinxtoprule
\sphinxstyletheadfamily 
\sphinxAtStartPar
\sphinxincludegraphics{{graph_to_dsm}.svg}
\\
\sphinxmidrule
\sphinxtableatstartofbodyhook
\sphinxAtStartPar
DSM Task Notation
\\
\sphinxbottomrule
\end{tabulary}
\sphinxtableafterendhook\par
\sphinxattableend\end{savenotes}


\subsubsection{Sequencing/Partitioning}
\label{\detokenize{SPM/DSM:sequencing-partitioning}}
\sphinxAtStartPar
Sequencing = reordering of the DSM rows and columns such that the new DSM arrangement does not contain any feedback marks, thus transforming the DSM into an upper triangular form.


\begin{savenotes}\sphinxattablestart
\sphinxthistablewithglobalstyle
\centering
\begin{tabulary}{\linewidth}[t]{T}
\sphinxtoprule
\sphinxstyletheadfamily 
\sphinxAtStartPar
\sphinxincludegraphics{{dsm_to_partitioned_dsm}.svg}
\\
\sphinxmidrule
\sphinxtableatstartofbodyhook
\sphinxAtStartPar
DSM Sequencing/Partitioning
\\
\sphinxbottomrule
\end{tabulary}
\sphinxtableafterendhook\par
\sphinxattableend\end{savenotes}


\subsubsection{Tearing}
\label{\detokenize{SPM/DSM:tearing}}
\sphinxAtStartPar
Tearing = choosing the set of feedback marks that, if removed from the matrix (and then the matrix is re\sphinxhyphen{}partitioned), will render the matrix upper\sphinxhyphen{}triangular

\sphinxAtStartPar
Criteria when making tearing decisions:
\begin{itemize}
\item {} 
\sphinxAtStartPar
Minimal number of tears: the motivation behind this criterion is that tears represent an approximation or an initial guess to be used; we would rather reduce the number of these guesses used.

\item {} 
\sphinxAtStartPar
Confine tears to the smallest blocks along the diagonal: the motivation behind this criterion is that if there are to be iterations within iterations (i.e. blocks within blocks), these inner iterations are done more often. Therefore, it is desirable to confine the inner iterations to a small number of tasks.

\end{itemize}


\begin{savenotes}\sphinxattablestart
\sphinxthistablewithglobalstyle
\centering
\begin{tabulary}{\linewidth}[t]{T}
\sphinxtoprule
\sphinxstyletheadfamily 
\sphinxAtStartPar
\sphinxincludegraphics{{partitioned_dsm_to_teared_partitioned_dsm}.svg}
\\
\sphinxmidrule
\sphinxtableatstartofbodyhook
\sphinxAtStartPar
DSM Tearing
\\
\sphinxbottomrule
\end{tabulary}
\sphinxtableafterendhook\par
\sphinxattableend\end{savenotes}


\subsection{How to Create a Task\sphinxhyphen{}Based Design Structure Matrix Model}
\label{\detokenize{SPM/DSM:how-to-create-a-task-based-design-structure-matrix-model}}\begin{enumerate}
\sphinxsetlistlabels{\arabic}{enumi}{enumii}{}{.}%
\item {} 
\sphinxAtStartPar
Select a project to model.

\item {} 
\sphinxAtStartPar
Identify the tasks of the project, who is responsible for each one, and the outputs created by each task.

\item {} 
\sphinxAtStartPar
Lay out the square matrix with the tasks in the order they are nominally executed.

\item {} 
\sphinxAtStartPar
Ask the process (task) experts what inputs are used for each task.

\item {} 
\sphinxAtStartPar
Insert marks representing the information inputs to each task.

\item {} 
\sphinxAtStartPar
Optional: Analyze the DSM model by re\sphinxhyphen{}sequencing the tasks to suggest a new process.

\item {} 
\sphinxAtStartPar
Identify coupled tasks representing the planned iterations. We call these the \sphinxstylestrong{meta\sphinxhyphen{}tasks}.

\item {} 
\sphinxAtStartPar
Identify groups of parallel (uncoupled) tasks.

\item {} 
\sphinxAtStartPar
Highlight the unplanned iterations.

\end{enumerate}


\section{Iterations}
\label{\detokenize{SPM/DSM:iterations}}
\sphinxAtStartPar
Iterations = repetition of tasks due to:
\begin{itemize}
\item {} 
\sphinxAtStartPar
changes in input information (upstream),

\item {} 
\sphinxAtStartPar
update of shared assumptions (concurrent), or

\item {} 
\sphinxAtStartPar
discovery of errors (downstream).

\end{itemize}


\subsection{Types}
\label{\detokenize{SPM/DSM:types}}

\begin{savenotes}\sphinxattablestart
\sphinxthistablewithglobalstyle
\centering
\begin{tabulary}{\linewidth}[t]{TTTT}
\sphinxtoprule
\sphinxstyletheadfamily 
\sphinxAtStartPar
Type
&\sphinxstyletheadfamily 
\sphinxAtStartPar
Cause
&\sphinxstyletheadfamily 
\sphinxAtStartPar
Predictable
&\sphinxstyletheadfamily 
\sphinxAtStartPar
Aim
\\
\sphinxmidrule
\sphinxtableatstartofbodyhook
\sphinxAtStartPar
Planned
&
\sphinxAtStartPar
Needs to “get it right the first time”
&
\sphinxAtStartPar
Yes (when), No (how much)
&
\sphinxAtStartPar
Facilitated
\\
\sphinxhline
\sphinxAtStartPar
Unplanned
&
\sphinxAtStartPar
Errors/unforseen problems
&
\sphinxAtStartPar
No
&
\sphinxAtStartPar
Minimized
\\
\sphinxbottomrule
\end{tabulary}
\sphinxtableafterendhook\par
\sphinxattableend\end{savenotes}


\subsection{Styles}
\label{\detokenize{SPM/DSM:styles}}

\begin{savenotes}\sphinxattablestart
\sphinxthistablewithglobalstyle
\centering
\begin{tabulary}{\linewidth}[t]{TTTT}
\sphinxtoprule
\sphinxstyletheadfamily 
\sphinxAtStartPar
Style
&\sphinxstyletheadfamily 
\sphinxAtStartPar
\#Activities
&\sphinxstyletheadfamily 
\sphinxAtStartPar
Assumption
&\sphinxstyletheadfamily 
\sphinxAtStartPar
Model
\\
\sphinxmidrule
\sphinxtableatstartofbodyhook
\sphinxAtStartPar
Sequential
&
\sphinxAtStartPar
=1
&
\sphinxAtStartPar
Next action\(\leftarrow\) Probabilities
&
\sphinxAtStartPar
Signal Flow Graph
\\
\sphinxhline
\sphinxAtStartPar
Parallel
&
\sphinxAtStartPar
>=2
&
\sphinxAtStartPar
Rework created for other coupled activities
&
\sphinxAtStartPar
Work Transformation
\\
\sphinxbottomrule
\end{tabulary}
\sphinxtableafterendhook\par
\sphinxattableend\end{savenotes}


\begin{savenotes}\sphinxattablestart
\sphinxthistablewithglobalstyle
\centering
\begin{tabulary}{\linewidth}[t]{T}
\sphinxtoprule
\sphinxstyletheadfamily 
\sphinxAtStartPar
\sphinxincludegraphics{{iteration_style}.svg}
\\
\sphinxmidrule
\sphinxtableatstartofbodyhook
\sphinxAtStartPar
Iteration Styles
\\
\sphinxbottomrule
\end{tabulary}
\sphinxtableafterendhook\par
\sphinxattableend\end{savenotes}


\subsubsection{Signal Flow Graph Model}
\label{\detokenize{SPM/DSM:signal-flow-graph-model}}
\sphinxAtStartPar
A \(- pz^t \rightarrow \) B
where
\begin{itemize}
\item {} 
\sphinxAtStartPar
\(p\) = probability

\item {} 
\sphinxAtStartPar
\(z^t\) = duration

\end{itemize}


\begin{savenotes}\sphinxattablestart
\sphinxthistablewithglobalstyle
\centering
\begin{tabulary}{\linewidth}[t]{T}
\sphinxtoprule
\sphinxstyletheadfamily 
\sphinxAtStartPar
\sphinxincludegraphics{{signal_flow_graph_model}.svg}
\\
\sphinxmidrule
\sphinxtableatstartofbodyhook
\sphinxAtStartPar
Signal Flow Graph Model
\\
\sphinxbottomrule
\end{tabulary}
\sphinxtableafterendhook\par
\sphinxattableend\end{savenotes}


\subsubsection{Work Transformation Model}
\label{\detokenize{SPM/DSM:work-transformation-model}}

\subsubsection{Assumptions}
\label{\detokenize{SPM/DSM:assumptions}}\begin{itemize}
\item {} 
\sphinxAtStartPar
All coupled tasks are attempted simultaneously.

\item {} 
\sphinxAtStartPar
Off\sphinxhyphen{}diagonal elements correspond to fractions of each task’s work which must be repeated during subsequent iterations.

\item {} 
\sphinxAtStartPar
Objective is to characterize the nature of design iteration.

\end{itemize}


\subsubsection{Mathematics}
\label{\detokenize{SPM/DSM:mathematics}}

\begin{savenotes}\sphinxattablestart
\sphinxthistablewithglobalstyle
\centering
\begin{tabulary}{\linewidth}[t]{TT}
\sphinxtoprule
\sphinxstyletheadfamily 
\sphinxAtStartPar
Math
&\sphinxstyletheadfamily 
\sphinxAtStartPar
Name
\\
\sphinxmidrule
\sphinxtableatstartofbodyhook
\sphinxAtStartPar
\(u_{t+1} = A u_t\)
&
\sphinxAtStartPar
Work vector
\\
\sphinxhline
\sphinxAtStartPar
\(U = \sum_{t=0}^\infty u_t = (\sum_{t=0}^\infty A^t)u_0\)
&
\sphinxAtStartPar
Total work vector
\\
\sphinxhline
\sphinxAtStartPar
\(A= S \land S^{-1}\)
&
\sphinxAtStartPar
Eigenvalue decomposition
\\
\sphinxhline
\sphinxAtStartPar
\(U=S(\sum_{t=0}^\infty \land^t)S^{-1}u_0\)
&
\sphinxAtStartPar
Substitution
\\
\sphinxhline
\sphinxAtStartPar
\(\sum_{t=0}^\infty \land^t = (I-\land)^{-1}\)
&
\sphinxAtStartPar
Diagonal matrix of\(1/(1-\lambda)\) terms
\\
\sphinxhline
\sphinxAtStartPar
\(U=S[(I-\land)^{-1}S^{-1}u_0]\)
&
\sphinxAtStartPar
Total work is a scaling of the eigenvectors
\\
\sphinxhline
\sphinxAtStartPar
where
&
\sphinxAtStartPar

\\
\sphinxbottomrule
\end{tabulary}
\sphinxtableafterendhook\par
\sphinxattableend\end{savenotes}
\begin{itemize}
\item {} 
\sphinxAtStartPar
\(U\) = total work

\item {} 
\sphinxAtStartPar
\(S\) = eigenvector matrix

\item {} 
\sphinxAtStartPar
\([(I-\land)^{-1}S^{-1}u_0]\) = scaling vector

\end{itemize}


\subsection{Discover Loops}
\label{\detokenize{SPM/DSM:discover-loops}}\begin{enumerate}
\sphinxsetlistlabels{\arabic}{enumi}{enumii}{}{.}%
\item {} 
\sphinxAtStartPar
Replace X \(\rightarrow\) 1 and “” \(\rightarrow\) 0

\item {} 
\sphinxAtStartPar
Square binary matrix

\item {} 
\sphinxAtStartPar
Find non\sphinxhyphen{}zero diagonals

\end{enumerate}


\begin{savenotes}\sphinxattablestart
\sphinxthistablewithglobalstyle
\centering
\begin{tabulary}{\linewidth}[t]{T}
\sphinxtoprule
\sphinxstyletheadfamily 
\sphinxAtStartPar
\sphinxincludegraphics{{find_loops}.svg}
\\
\sphinxmidrule
\sphinxtableatstartofbodyhook
\sphinxAtStartPar
Find Loop Process
\\
\sphinxbottomrule
\end{tabulary}
\sphinxtableafterendhook\par
\sphinxattableend\end{savenotes}

\sphinxstepscope


\part{Misc}

\sphinxstepscope


\chapter{Modified PERT Bistribution}
\label{\detokenize{Misc/pert-mod:modified-pert-bistribution}}\label{\detokenize{Misc/pert-mod::doc}}\begin{equation*}
\begin{split}
PERT(min,Mo,max) = Beta(\alpha,\beta)\cdot(max-min)+ min = \frac{(x-min)^{\alpha-1}(max-x)^{\beta-1}}{B(\alpha,\beta)(max-min)^{\alpha+\beta-1}}
\end{split}
\end{equation*}
\sphinxAtStartPar
where


\begin{savenotes}\sphinxattablestart
\sphinxthistablewithglobalstyle
\centering
\begin{tabulary}{\linewidth}[t]{TT}
\sphinxtoprule
\sphinxstyletheadfamily 
\sphinxAtStartPar
Term
&\sphinxstyletheadfamily 
\sphinxAtStartPar
Definition
\\
\sphinxmidrule
\sphinxtableatstartofbodyhook
\sphinxAtStartPar
\(min\)
&
\sphinxAtStartPar
Lower bound
\\
\sphinxhline
\sphinxAtStartPar
\(Mo\)
&
\sphinxAtStartPar
Mode
\\
\sphinxhline
\sphinxAtStartPar
\(max\)
&
\sphinxAtStartPar
Upper bound
\\
\sphinxhline
\sphinxAtStartPar
\(\alpha\)
&
\sphinxAtStartPar
first shape parameter
\\
\sphinxhline
\sphinxAtStartPar
\(\beta\)
&
\sphinxAtStartPar
second shape parameter
\\
\sphinxhline
\sphinxAtStartPar
\(B\)
&
\sphinxAtStartPar
Beta function
\\
\sphinxbottomrule
\end{tabulary}
\sphinxtableafterendhook\par
\sphinxattableend\end{savenotes}

\sphinxAtStartPar
The \(B\) function is defined as follows:
\begin{equation*}
\begin{split}
B(\alpha,\beta) = \int_{0}^{1} u^{\alpha-1}\left(1-u\right)^{\beta-1}\textrm{d}u =  \frac{\Gamma(\alpha)\Gamma(\beta)}{\Gamma(\alpha+\beta)}
\end{split}
\end{equation*}

\begin{savenotes}\sphinxattablestart
\sphinxthistablewithglobalstyle
\centering
\begin{tabulary}{\linewidth}[t]{TT}
\sphinxtoprule
\sphinxstyletheadfamily 
\sphinxAtStartPar
Term
&\sphinxstyletheadfamily 
\sphinxAtStartPar
Definition
\\
\sphinxmidrule
\sphinxtableatstartofbodyhook
\sphinxAtStartPar
\(\Gamma\)
&
\sphinxAtStartPar
Gamma distribution
\\
\sphinxbottomrule
\end{tabulary}
\sphinxtableafterendhook\par
\sphinxattableend\end{savenotes}

\sphinxAtStartPar
The shape parameters, \(\alpha\) and \(\beta\), are obtained as follows:
\begin{equation*}
\begin{split}
\alpha = \frac{\gamma Mo + max-(\gamma+1)min}{max-min}=1+\gamma \frac{Mo-min}{max-min}
\end{split}
\end{equation*}\begin{equation*}
\begin{split}
\beta = \frac{(\gamma+1)max-min-\gamma Mo}{max-min}=1+\gamma \frac{max-Mo}{max-min}
\end{split}
\end{equation*}
\sphinxAtStartPar
so that
\begin{equation*}
\begin{split}
\mu = \frac{min+\gamma Mo+max}{\gamma+2}
\end{split}
\end{equation*}
\sphinxAtStartPar
and
\begin{equation*}
\begin{split}
\sigma = \frac{(\mu-min)(max-\mu}{\gamma+3}
\end{split}
\end{equation*}
\sphinxstepscope


\chapter{Putnam Model}
\label{\detokenize{Misc/putnam:putnam-model}}\label{\detokenize{Misc/putnam::doc}}

\section{The Norden\sphinxhyphen{}Rayleigh Curve}
\label{\detokenize{Misc/putnam:the-norden-rayleigh-curve}}
\sphinxAtStartPar
The curve is modeled by differential equation
\begin{equation*}
\begin{split}
m{(t)} = \frac{dy}{dt} = 2Kate^{-at^2}
\end{split}
\end{equation*}


\sphinxAtStartPar
where


\begin{savenotes}\sphinxattablestart
\sphinxthistablewithglobalstyle
\centering
\begin{tabulary}{\linewidth}[t]{TT}
\sphinxtoprule
\sphinxstyletheadfamily 
\sphinxAtStartPar
Term
&\sphinxstyletheadfamily 
\sphinxAtStartPar
Definition
\\
\sphinxmidrule
\sphinxtableatstartofbodyhook
\sphinxAtStartPar
\(\frac{dy}{dt}\)
&
\sphinxAtStartPar
manpower utilization rate per unit time
\\
\sphinxhline
\sphinxAtStartPar
\(a\)
&
\sphinxAtStartPar
acceleration factor \sphinxhyphen{} curve sharpness parameter
\\
\sphinxhline
\sphinxAtStartPar
\(K\)
&
\sphinxAtStartPar
total project effort in staff years \sphinxhyphen{} area underneath the curve in\(\left[0,\infty\right]\)
\\
\sphinxhline
\sphinxAtStartPar
\(t\)
&
\sphinxAtStartPar
elapsed time
\\
\sphinxbottomrule
\end{tabulary}
\sphinxtableafterendhook\par
\sphinxattableend\end{savenotes}

\sphinxAtStartPar
Integrating \(m{(t)}\) on \([0,\infty]\), we obtain \(y{(t)} = K\left[1-e^{-at^2}\right]\).

\sphinxAtStartPar
If \(y(0) = 0\) and \(y(\infty) = K\), then \(\frac{d^2y}{dt^2} = 2Kae^{-at^2}\left[1-2at^2\right]=0\), and \(t_d^2 = \frac{1}{2a}\), where \(t_d\) is the time at which the maximum effort rate occurs.

\sphinxAtStartPar
Replacing \(t_d\) leads to \(E=y{(t)}=K\left(1-e^{\frac{t_d^2}{2t_d^2}}\right) = K\left(1-e^{-.5}\right)\), \(E=y(t)=.3935K\), and \(a=\frac{1}{2t_d^2}\).

\sphinxAtStartPar
Replacing \(a\) with \(\frac{1}{2t_d^2}\) in the N/R model, we obtain
\begin{equation*}
\begin{split}
m(t)=\frac{2K}{2t_d^2}te^{-\frac{t^2}{2t_d^2}} = \frac{K}{t_d^2}te^{-\frac{t^2}{2t_d^2}}.
\end{split}
\end{equation*}
\sphinxAtStartPar
The peak manning is denoted by \(m_0\) and is obtained \(m(t_d) = m_0 = \frac{K}{t_d \sqrt{e}}\), where


\begin{savenotes}\sphinxattablestart
\sphinxthistablewithglobalstyle
\centering
\begin{tabulary}{\linewidth}[t]{TT}
\sphinxtoprule
\sphinxstyletheadfamily 
\sphinxAtStartPar
Term
&\sphinxstyletheadfamily 
\sphinxAtStartPar
Definition
\\
\sphinxmidrule
\sphinxtableatstartofbodyhook
\sphinxAtStartPar
\(K\)
&
\sphinxAtStartPar
total project cost/effort (person\sphinxhyphen{}years)
\\
\sphinxhline
\sphinxAtStartPar
\(t_d\)
&
\sphinxAtStartPar
delivery time (years)
\\
\sphinxhline
\sphinxAtStartPar
\(m_0\)
&
\sphinxAtStartPar
staffing level
\\
\sphinxbottomrule
\end{tabulary}
\sphinxtableafterendhook\par
\sphinxattableend\end{savenotes}


\section{Difficulty Metric}
\label{\detokenize{Misc/putnam:difficulty-metric}}
\sphinxAtStartPar
Slope of manpower distribution curve at \(t=0\) has some useful properties.
\begin{equation*}
\begin{split}
m'{(t)} = \frac{d^2y}{dt^2} = 2Kae^{-at^2}\left(1-2at^2\right)
\end{split}
\end{equation*}
\sphinxAtStartPar
For \(t=0\),
\begin{equation*}
\begin{split}
m'{(0)} = 2Ka = \frac{2K}{2t_d^2} = \frac{K}{t_d^2}
\end{split}
\end{equation*}
\sphinxAtStartPar
The ratio
\begin{equation*}
\begin{split}
D = \frac{K}{t_d^2}
\end{split}
\end{equation*}
\sphinxAtStartPar
is called difficulty metric \(D\) (person/year)


\section{The Software Equation}
\label{\detokenize{Misc/putnam:the-software-equation}}
\sphinxAtStartPar
The average productivity
\begin{equation*}
\begin{split}
\overline{PR} = \frac{\text{total end product code}}{\text{total effort to produce code}}
\end{split}
\end{equation*}
\sphinxAtStartPar
The general size of the product in source statements is
\begin{equation*}
\begin{split}
S_s = C_k K^{1/3} t_d^{4/3}
\end{split}
\end{equation*}
\sphinxAtStartPar
where \(C_k\) is a measure of the state of technology of the human\sphinxhyphen{}machine system.

\sphinxAtStartPar
This is a reasonable relation linking the output (\(S_s\)) to the input (\(K\), \(t_d\) \sphinxhyphen{} management parameters) and a constant (\(C_k\)) which is somehow a measure of the state of technology being applied to the project.
Adding people to accelerate a project can accomplish this until the gradient condition is reached, but only at a very high cost.


\subsection{The Effort\sphinxhyphen{}Time Tradeoff Law}
\label{\detokenize{Misc/putnam:the-effort-time-tradeoff-law}}
\sphinxAtStartPar
The tradeoff law between effort and time can be obtained
explicity from the software equation
\begin{equation*}
\begin{split}
S_s = C_k K^{1/3} t_d^{4/3}
\end{split}
\end{equation*}
\sphinxAtStartPar
A constant number of source statements (\(S_s = \text{const}\)) implies \(K t_d^4 = \text{const}\).
So, \(K = \text{constant}/t_d^4\), or proportionally, development effort = constat/\(t_d^4\), is the effort\sphinxhyphen{}development tradeoff law.


\subsection{Effect of Constant Productivity}
\label{\detokenize{Misc/putnam:effect-of-constant-productivity}}
\sphinxAtStartPar
One other relation is worth obtaining; the one where the average productivity remains constant.

\sphinxAtStartPar
\(\overline{PR} = C_n D^{-2/3}\) implies that \(D = \text{const}\)
So the productivity for different projects will be the same only if the difficulty is the same.
This does not seem reasonable to expect very frequently since the difficulty is a measure of the software work to be done, i.e., \(K/t_d^2 = D\)  which is a function of the number of files, the number of reports, and the number of programs the system has.
Thus, planning a new project based on using the same productivity a previous project had, is fallacious unless the difficulty is the same.







\renewcommand{\indexname}{Index}
\printindex
\end{document}